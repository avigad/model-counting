\documentclass[letterpaper,USenglish,cleveref, autoref, thm-restate]{lipics-v2021}
%This is a template for producing LIPIcs articles. 
%See lipics-v2021-authors-guidelines.pdf for further information.
%for A4 paper format use option "a4paper", for US-letter use option "letterpaper"
%for british hyphenation rules use option "UKenglish", for american hyphenation rules use option "USenglish"
%for section-numbered lemmas etc., use "numberwithinsect"
%for enabling cleveref support, use "cleveref"
%for enabling autoref support, use "autoref"
%for anonymousing the authors (e.g. for double-blind review), add "anonymous"
%for enabling thm-restate support, use "thm-restate"
%for enabling a two-column layout for the author/affilation part (only applicable for > 6 authors), use "authorcolumns"
%for producing a PDF according the PDF/A standard, add "pdfa"

%\pdfoutput=1 %uncomment to ensure pdflatex processing (mandatatory e.g. to submit to arXiv)
%\hideLIPIcs  %uncomment to remove references to LIPIcs series (logo, DOI, ...), e.g. when preparing a pre-final version to be uploaded to arXiv or another public repository

%\graphicspath{{./graphics/}}%helpful if your graphic files are in another directory

\usepackage{amsmath}
\usepackage{tikz}
\usepackage{pgfplots}
\usepackage{booktabs}
\usepackage{hyperref}

\newcommand{\pand}{\mathbin{\land^{\sf p}}}
\newcommand{\por}{\mathbin{\lor^{\sf p}}}
\DeclareMathOperator*{\Pand}{\bigwedge^{\sf v}}
\DeclareMathOperator*{\Por}{\bigvee^{\sf a}}
\newcommand{\boolnot}{\neg}
\newcommand{\tautology}{\top}
\newcommand{\nil}{\bot}
\newcommand{\obar}[1]{\overline{#1}}
\newcommand{\oneminus}{{\sim}}
\newcommand{\lit}{\ell}

\newcommand{\varset}{X}
\newcommand{\exvarset}{Z}
\newcommand{\dependencyset}{{\cal D}}
\newcommand{\litset}{{\cal L}}
\newcommand{\ring}{{\cal R}}
\newcommand{\dset}{{\cal A}}
\newcommand{\rep}{\textbf{R}}
\newcommand{\radd}{+}
\newcommand{\rmul}{\times}
\newcommand{\addident}{\textbf{0}}
\newcommand{\mulident}{\textbf{1}}
\newcommand{\imply}{\Rightarrow}

\newcommand{\assign}{\alpha}
\newcommand{\passign}{\rho}
\newcommand{\uassign}{{\cal U}}
\newcommand{\modelset}{{\cal M}}

\newcommand{\indegree}{\textrm{indegree}}
\newcommand{\outdegree}{\textrm{outdegree}}
\newcommand{\validate}{\textsf{validate}}

\newcommand{\makenode}[1]{\mathbf{#1}}
\newcommand{\nodeu}{\makenode{u}}
\newcommand{\nodev}{\makenode{v}}
\newcommand{\nodes}{\makenode{s}}
\newcommand{\nodep}{\makenode{p}}
\newcommand{\noder}{\makenode{r}}

\newcommand{\simplify}[2]{#1|_{#2}}
\newcommand{\prov}{\mathit{Prov}}



\bibliographystyle{plainurl}% the mandatory bibstyle

\title{Certified Knowledge Compilation \\ with Application to Verified Model Counting}

\titlerunning{Certified Knowledge Compilation}

\author{Randal E. Bryant}{Computer Science Department, Carnegie Mellon University, Pittsburgh, PA 15213 USA}{rebryant@cmu.edu}{https://orcid.org/0000-0001-5024-6613}{Supported by NSF grant CCF-2108521}
\author{Wojciech Nawrocki}{Department of Philosophy, Carnegie Mellon University}{wjnawrock@cmu.edu}{https://orcid.org/0000-0002-8839-0618}{}
\author{Jeremy Avigad}{Department of Philosophy, Carnegie Mellon University}{avigad@cmu.edu}{https://orcid.org/0000-0003-1275-315X}{}
\author{Marijn J. H. Heule}{Computer Science Department, Carnegie Mellon University}{marijn@cmu.edu}{https://orcid.org/0000-0002-5587-8801}{Supported by NSF grant CCF-2108521}

\authorrunning{R. E. Bryant, et al} %TODO mandatory. First: Use abbreviated first/middle names. Second (only in severe cases): Use first author plus 'et al.'

%\Copyright{Jane Open Access and Joan R. Public} %TODO mandatory, please use full first names. LIPIcs license is "CC-BY";  http://creativecommons.org/licenses/by/3.0/

%\ccsdesc[100]{\textcolor{red}{Replace ccsdesc macro with valid one}} %TODO mandatory: Please choose ACM 2012 classifications from https://dl.acm.org/ccs/ccs_flat.cfm 

%\keywords{Dummy keyword} %TODO mandatory; please add comma-separated list of keywords

%\category{} %optional, e.g. invited paper

%\funding{(Optional) general funding statement \dots}%optional, to capture a funding statement, which applies to all authors. Please enter author specific funding statements as fifth argument of the \author macro.

%\nolinenumbers %uncomment to disable line numbering

%Editor-only macros:: begin (do not touch as author)%%%%%%%%%%%%%%%%%%%%%%%%%%%%%%%%%%
%\EventEditors{John Q. Open and Joan R. Access}
%\EventNoEds{2}
%\EventLongTitle{42nd Conference on Very Important Topics (CVIT 2016)}
%\EventShortTitle{CVIT 2016}
%\EventAcronym{CVIT}
%\EventYear{2016}
%\EventDate{December 24--27, 2016}
%\EventLocation{Little Whinging, United Kingdom}
%\EventLogo{}
%\SeriesVolume{42}
%\ArticleNo{23}
%%%%%%%%%%%%%%%%%%%%%%%%%%%%%%%%%%%%%%%%%%%%%%%%%%%%%%

\newcommand{\program}[1]{\textsc{#1}}
\newcommand{\lean}{\program{lean4}}

\begin{document}

\maketitle

\definecolor{redorange}{rgb}{0.878431, 0.235294, 0.192157}
\definecolor{lightblue}{rgb}{0.552941, 0.72549, 0.792157}
\definecolor{clearyellow}{rgb}{0.964706, 0.745098, 0}
\definecolor{clearorange}{rgb}{0.917647, 0.462745, 0}
\definecolor{mildgray}{rgb}{0.54902, 0.509804, 0.47451}
\definecolor{softblue}{rgb}{0.643137, 0.858824, 0.909804}
\definecolor{bluegray}{rgb}{0.141176, 0.313725, 0.603922}
\definecolor{lightgreen}{rgb}{0.709804, 0.741176, 0}
\definecolor{darkgreen}{rgb}{0.152941, 0.576471, 0.172549}
\definecolor{redpurple}{rgb}{0.835294, 0, 0.196078}
\definecolor{midblue}{rgb}{0, 0.592157, 0.662745}
\definecolor{clearpurple}{rgb}{0.67451, 0.0784314, 0.352941}
\definecolor{browngreen}{rgb}{0.333333, 0.313725, 0.145098}
\definecolor{darkestpurple}{rgb}{0.396078, 0.113725, 0.196078}
\definecolor{greypurple}{rgb}{0.294118, 0.219608, 0.298039}
\definecolor{darkturquoise}{rgb}{0, 0.239216, 0.298039}
\definecolor{darkbrown}{rgb}{0.305882, 0.211765, 0.160784}
\definecolor{midgreen}{rgb}{0.560784, 0.6, 0.243137}
\definecolor{darkred}{rgb}{0.576471, 0.152941, 0.172549}
\definecolor{darkpurple}{rgb}{0.313725, 0.027451, 0.470588}
\definecolor{darkestblue}{rgb}{0, 0.156863, 0.333333}
\definecolor{lightpurple}{rgb}{0.776471, 0.690196, 0.737255}
\definecolor{softgreen}{rgb}{0.733333, 0.772549, 0.572549}
\definecolor{offwhite}{rgb}{0.839216, 0.823529, 0.768627}
\definecolor{medgreen}{rgb}{0.15, 0.6, 0.15}



\section*{Benchmark machine}
\begin{itemize}
\item 2021 MacBook Pro 18,4
  \begin{itemize}
  \item 3.2 Ghz Apple M1 processor
  \item 64 GB Ram
  \end{itemize}
\item  Samsung T7 solid-state disk
  \begin{itemize}
  \item 2TB capacity
  \item Up to 1 GB/s R/W
  \item Critical for performance.  Wrote and read files of over 162 GB.
  \end{itemize}
\end{itemize}

\section*{Methodology}
\begin{enumerate}
   \item Retrieved 200 public model counting benchmarks from 2022 Model Counting Competition website
     \begin{center}
       \url{https://mccompetition.org/2022/mc_description.html}
     \end{center}
     \begin{itemize}
     \item 100 public instances from Track 1 (unweighted model counting)
     \item 100 public instances from Track 2 (weighted model counting)
     \end{itemize}
   \item Removed duplicates
     \begin{itemize}
    \item  20 instances were duplicated in the two tracks
     \item Reduces set of benchmarks to 180
     \end{itemize}
   \item Ran D4 on 180 instances with time limit of 4000 seconds
     \begin{itemize}
     \item Completed for 124 instances
     \end{itemize}
   \item Generated POG representations from dDNNF graphs
     \begin{itemize}
     \item Simplifications via constant propagation
       \begin{itemize}
       \item Number of POG nodes around $0{.}90\times$ number of dDNNF nodes
       \end{itemize}
     \item Used heuristics to identify root node, and then put in topological order with root node at end.
     \item Measure complexity by number of defining clauses $D$
       \begin{itemize}
       \item Product or sum node with degree $k$ requires $k+1$ clauses
       \item $D$ also equals sum of number of edges and number of nodes
       \item For 124 instances, Min = 304, Median = 848,784, Mean = 93,107,937, Max = 2,761,457,765
       \end{itemize}
     \end{itemize}
   \item Ran program {\sc d2p} to generate CRAT file for maximum of 10,000 seconds
     \begin{itemize}
     \item Completed for 108 instances
     \item Exceeded clause limit for 1 instance (Our programs represent a clause ID as a 32-bit signed value)
     \item Ran out of memory for 1 instance
     \item Timed out for 14 instances
     \item Was able to do one-sided verification for 9 of these.
     \end{itemize}
   \item For completed CRAT files, ran {\sc crat-check} CRAT checker
     \begin{itemize}
     \item Tests all conditions for correct CRAT
     \item Most time spent performing RUP checks for clause addition and deletion
     \item Completed successfully for all 108 instances
     \item Time generally faster than {\sc d2p}. (min = 0.005, max = 1.4, harmonic mean = 0.10)
     \end{itemize}
\end{enumerate}

\section*{Time to Generate and Validate CRAT}

\begin{itemize}
\item Min 0.01, Average 1318.3, Max 13,144.8, Median 71.64
\item Ratio to time for D4: Min 0.50, Harmonic Mean 5.50, Max 457.69
\end{itemize}



\begin{figure}
\centering{%
\begin{tikzpicture}[scale = 0.90]
  \begin{axis}[mark options={scale=0.55},grid=both, grid style={black!10}, ymode=log,
      legend style={at={(0.30,0.96)}},
      legend cell align={left},
                              x post scale=1.8, y post scale=2.4,
                              xmode=log,xmin=0.01,xmax=4000,
                              xtick={0.01, 0.1,1.0,10,100,1000,10000}, xticklabels={0.01, 0.1, 1.0, 10, 100, {1,000}, {10,000}},
                              ymin=0.01, ymax=29000,
                              ytick={0.01, 0.1,1.0,10,100,1000,10000,100000}, yticklabels={0.01, 0.1, 1.0, 10, 100, {1,000},{10,000},{100,000}},
                              xlabel={D4 runtime (seconds)}, ylabel={CRAT generation and checking runtime (seconds)},
%                              title={D4 Defining Clause Generation}
            ]
    \input{data-formatted/time-d4-verify}
    \input{data-formatted/time-d4-verify-onesided-only}
    \input{data-formatted/time-d4-failure}
    \legend{
      \scriptsize \textsf{Full validation},
      \scriptsize \textsf{One-sided validation},
      \scriptsize \textsf{Resource limit exceeded},
    }
    \addplot[mark=none, dashed] coordinates{(0.01,25000) (4000, 25000)};
    \addplot[mark=none, color=lightblue] coordinates{(0.1,0.01) (10000.0,1000.0)};
    \addplot[mark=none, color=lightblue] coordinates{(0.01,0.01) (10000.0,10000.0)};
    \addplot[mark=none, color=lightblue] coordinates{(0.01,0.1) (2000,20000)};
    \addplot[mark=none, color=lightblue] coordinates{(0.01,1.0) (200, 20000)};
    \addplot[mark=none, color=lightblue] coordinates{(0.01,10.0) (20, 20000)};
    \node[left] at (axis cs: 0.9,0.12) {$0{.}1\times$};
    \node[left] at (axis cs: 0.09,0.12) {$1\times$};
    \node[left] at (axis cs: 0.09,1.2) {$10\times$};
    \node[left] at (axis cs: 0.09,12.0) {$100\times$};
    \node[left] at (axis cs: 0.09,120.0) {$1000\times$};

          \end{axis}
\end{tikzpicture}
} % centering
\caption{Combined runtime for CRAT proof generation and checking  as function of D4 runtime}
\label{fig:d4:crat}
\end{figure}

\begin{figure}
\centering{%
\begin{tikzpicture}[scale = 0.90]
  \begin{axis}[mark options={scale=0.55},grid=both, grid style={black!10}, ymode=log, ymin=100, ymax=3000000000,
      legend style={at={(0.30,0.98)}},
      legend cell align={left},
                              x post scale=1.8, y post scale=1.8,
                              xmode=log, xmin=0.01,xmax=4000,
                              xtick={0.01, 0.1,1.0,10,100,1000,10000}, xticklabels={0.01, 0.1, 1.0, 10, 100, {1,000}, {10,000}},
                              ytick={100,1000, 10000, 100000, 1000000, 10000000, 100000000, 1000000000, 1e10}, yticklabels={$10^2$,$10^3$,$10^4$,$10^5$,$10^6$,$10^7$,$10^8$,$10^9$,$10^{10}$},
                              xlabel={D4 runtime (seconds)}, ylabel={Defining Clauses},
%                              title={D4 Defining Clause Generation}
            ]
    \input{data-formatted/time+defining-success}
    \input{data-formatted/time+defining-onesided-only}
    \input{data-formatted/time+defining-failure}
    \legend{
      \scriptsize \textsf{Full validation},
      \scriptsize \textsf{One-sided validation},
      \scriptsize \textsf{Resource limit exceeded},
    }
 \end{axis}
\end{tikzpicture}
} % centering
\caption{Number of defining clauses in generated POG as function of D4 runtime}
\label{fig:generate:d4:log}
\end{figure}

\section*{Generated CRAT Files}

\begin{itemize}
\item Proof Clauses: Min 554, Median 1,719,245, Average 45,001,945, Max 770,382,773
\item Ratio to number of defining clauses: Min 1.54, Harmonic Mean 3.13, Max 6072.96
\end{itemize}


\begin{figure}
\centering{%
\begin{tikzpicture}[scale = 0.90]
  \begin{axis}[mark options={scale=0.55},grid=both, grid style={black!10}, ymode=log, ymin=100, ymax=1000000000, 
                              x post scale=1.8, y post scale=2.15,
                              xmode=log,xmin=100,xmax=1000000000, 
                              xtick={100,1000, 10000, 100000, 1000000, 10000000, 100000000, 1000000000}, xticklabels={$10^2$,$10^3$,$10^4$,$10^5$,$10^6$,$10^7$,$10^8$,$10^9$},
                              ytick={100,1000, 10000, 100000, 1000000, 10000000, 100000000, 1000000000}, yticklabels={$10^2$,$10^3$,$10^4$,$10^5$,$10^6$,$10^7$,$10^8$,$10^9$},
                              xlabel={Defining Clauses}, ylabel={Proof Clauses},
%                              title={D4 Defining Clause Generation}
            ]
    \addplot[mark=none, color=lightblue] coordinates{(100,100) (1e9,1e9)};
    \addplot[mark=none, color=lightblue] coordinates{(100,1000) (1e8,1e9)};
    \addplot[mark=none, color=lightblue] coordinates{(100,10000) (1e7,1e9)};
    \addplot[mark=none, color=lightblue] coordinates{(100,100000) (1e6,1e9)};
    \addplot[mark=none, color=lightblue] coordinates{(100,1000000) (1e5,1e9)};
    \input{data-formatted/defining+total}
    \node[left] at (axis cs: 900,1200) {$1\times$};
    \node[left] at (axis cs: 900,12000) {$10\times$};
    \node[left] at (axis cs: 900,120000) {$100\times$};
    \node[left] at (axis cs: 900,1200000) {$1{,}000\times$};
    \node[left] at (axis cs: 900,12000000) {$10{,}000\times$};

          \end{axis}
\end{tikzpicture}
} % centering
\caption{Total number of clauses in CRAT file as function of number of defining clauses}
\label{fig:defining:total}
\end{figure}



\section*{Computing Counts and Weighted Counts}

\begin{itemize}
\item Implemented package to represent numbers of form $a \cdot 2^b \cdot 5^c$.
  \begin{itemize}
  \item For unweighted counts, first compute density and then scale by $2^n$
  \item For unweighted counts, perform exact decimal arithmetic
  \item The set of numbers of this form is not closed under a reciprocal operation, but the weights in the competion are either of the form
    $w(x) + w(\overline{x}) = 1.0$ or $w(x) = w(\overline{x}) = 1$.  The latter case sums to 2, with reciprocal $1/2$.
  \end{itemize}
\item Can implement having $a$ be integer of arbitrary size and $b$ and $c$ being 32-bit integers.
\item Weighted counts ranged up to 260,909 digits.
\end{itemize}

\section*{Benefits of Optimizations}

Two optimizations were implemented within the CRAT framework:
\begin{description}
\item[Lemmas:] Nodes with indegree greater than one were validated with separate lemmas, avoiding the possibly exponential expansion of the POG into a tree during the proof construction.

\item[Literal grouping:] For a product node with multiple literal
  arguments, those that cannot be validated by BCP are combined as
  arguments to a newly defined product node.  The extension variable
  associated with this node is validated, and the proof is then used
  to validate the literals.  This reduces the number of
  calls to the SAT solver to at most one per product node.
\end{description}

\begin{figure}
\begin{minipage}{0.49\textwidth}
\centering{%
\begin{tikzpicture}[scale = 0.80]
  \begin{axis}[mark options={scale=0.55},grid=both, grid style={black!10}, ymode=log, ymin=1000, ymax=3e8, 
      legend style={at={(0.50,0.98)}},
      legend cell align={left},
                              x post scale=1.1, y post scale=1.8,
                              xmode=log,xmin=1000,xmax=1e7, 
                              xtick={1000, 10000, 100000, 1000000, 10000000}, xticklabels={$10^3$,$10^4$,$10^5$,$10^6$,$10^7$},
                              ytick={1000, 10000, 100000, 1000000, 10000000, 100000000, 1e9}, yticklabels={$10^3$,$10^4$,$10^5$,$10^6$,$10^7$,$10^8$, $10^9$},
%                              xlabel={Optimized Proof Clauses},
%                              ylabel={Proof Clauses},
            ]
    \input{data-formatted/clauses-lemma-nolemma}
    \input{data-formatted/clauses-nosplit-split}
    \legend{
      \scriptsize \textsf{No lemmas},
      \scriptsize \textsf{No literal grouping},
    }
    \addplot[mark=none, color=lightblue] coordinates{(1000,1000) (1e8,1e8)};
    \addplot[mark=none, color=lightblue] coordinates{(1000,10000) (1e8,1e9)};
    \addplot[mark=none, color=lightblue] coordinates{(1000,100000) (1e7,1e9)};

    \node[left] at (axis cs: 5000,7000) {$1\times$};
    \node[left] at (axis cs: 5000,70000) {$10\times$};
    \node[left] at (axis cs: 5000,700000) {$100\times$};


          \end{axis}
\end{tikzpicture}
} % centering
\end{minipage}
\begin{minipage}{0.49\textwidth}
\centering{%
\begin{tikzpicture}[scale = 0.80]
  \begin{axis}[mark options={scale=0.55},grid=both, grid style={black!10}, ymode=log, ymin=1000, ymax=3e8, 
      legend style={at={(0.50,0.98)}},
      legend cell align={left},
                              x post scale=1.1, y post scale=1.8,
                              xmode=log,xmin=1000,xmax=1e7, 
                              xtick={1000, 10000, 100000, 1000000, 10000000}, xticklabels={$10^3$,$10^4$,$10^5$,$10^6$,$10^7$},
                              ytick={1000, 10000, 100000, 1000000, 10000000, 100000000, 1e9}, yticklabels={$10^3$,$10^4$,$10^5$,$10^6$,$10^7$,$10^8$, $10^9$},
%                              xlabel={Optimized Proof Clauses},y
%                              ylabel={Proof Clauses},
            ]
    \input{data-formatted/clauses-optimized-unoptimized}    
    \legend{
      \scriptsize \textsf{Neither optimization},
    }
    \addplot[mark=none, color=lightblue] coordinates{(1000,1000) (1e8,1e8)};
    \addplot[mark=none, color=lightblue] coordinates{(1000,10000) (1e8,1e9)};
    \addplot[mark=none, color=lightblue] coordinates{(1000,100000) (1e7,1e9)};

    \node[left] at (axis cs: 5000,7000) {$1\times$};
    \node[left] at (axis cs: 5000,70000) {$10\times$};
    \node[left] at (axis cs: 5000,700000) {$100\times$};
  \end{axis}
\end{tikzpicture}
} % centering
\end{minipage}
\caption{Proof clauses when one (left) or both (right) optimizations are disabled, versus when both optimizations are enabled}
\label{fig:optimized:lessoptimized}
\end{figure}


We conducted a set of experiments using the 80 benchmark instances for
which the total number of proof clauses when both optimizations are
performed was limited to $10^7$.  We then made three additional runs
of the CRAT generator and checker, with each of the two optimizations
disabled individually and in combination.  Figures
\ref{fig:optimized:partially-optimized}--\ref{fig:optimized:unoptimized} illustrates the effect of the two
optimizations.  Each set of points shows the number of clauses when
both optimizations are performed for the $X$ value and the number with
one or both of these optimizations disabled as the $Y$ value.

The ratios between the number of clauses when either or both optimizations is disabled versus when both are enabled can be summarized as:
\begin{description}
\item[Without lemmas:] Min 1.00, Max 52.54, Harmonic mean 2.11
\item[Without literal grouping:] Min 0.84, Max 39.60, Harmonic mean 1.18
\item[Without either optimization:] Min 0.99, Max 52.54, Harmonic mean 2.95
\end{description}

Here are other ways to summarize the impact on proof length:
\begin{itemize}
\item None of the 80 benchmarks require more than $10^7$
proof clauses when both optimizations are used.  Without lemmas, 19
rise above that threshold.  Without literal merging, 7 do.  Without
either, 25 rise above the threshold.  Two of the benchmarks require
more than $10^8$ proof clauses without lemmas.
\item
  Of the 80 benchmark problems evaluated, and comparing with both
  optimizations disabled, 47 had the number of proof clauses reduced
  by a factor of 2 by using lemmas, 14 by using literal merging, and
  60 by enabling both optimizations.
\end{itemize}

The time performance is less dramatic, but still significant.
Comparing the sum of CRAT generation and checking times for each of
the four combinations of options, we get the following times relative
to the one with both optimizations enabled:
\begin{description}
\item[Without lemmas:] Min 0.85, Max 37.98, Harmonic mean 1.80
\item[Without literal grouping:] Min 0.83, Max 3.92, Harmonic mean 1.12
\item[Without either optimization:] Min 0.91, Max 41.97, Harmonic mean 2.19
\end{description}

Here is another way to summarize the impact on the combined time to generate and validate the CRAT file:
\begin{itemize}
\item Of the 80 benchmark problems evaluated, and comparing with both
  optimizations disabled, 40 had the time reduced by at least a factor of 2
  using lemmas, 7 by using literal merging, and 49 by enabling both
  optimizations.
\end{itemize}

\section{Comparing Monolithic and Structured Validation}

\begin{figure}
\centering{%
\begin{tikzpicture}[scale = 0.90]
  \begin{axis}[mark options={scale=0.55},grid=both, grid style={black!10}, ymode=log,
      legend style={at={(0.22,0.98)}},
      legend cell align={left},
                              x post scale=1.8, y post scale=2.2,
                              xmode=log,xmin=0.01,xmax=1900,
                              xtick={0.01, 0.1,1.0,10,100,1000}, xticklabels={0.01, 0.1, 1.0, 10, 100,{1,000}},
                              ymin=0.01, ymax=1900,
                              ytick={0.01, 0.1,1.0,10,100,1000}, yticklabels={0.01, 0.1, 1.0, 10, 100, {1,000}},
                              xlabel={Structured generation + checking (seconds)}, ylabel={Monolithic generation + checking (seconds)},
            ]
    \input{data-formatted/time-structured+monolithic}
    \input{data-formatted/time-structured+monolithic_timeout}
    \addplot[mark=none, color=lightblue] coordinates{(1.0,0.01) (1000.0,10.0)};
    \addplot[mark=none, color=lightblue] coordinates{(0.1,0.01) (1000.0,100.0)};
    \addplot[mark=none, color=lightblue] coordinates{(0.01,0.01) (1000.0,1000.0)};
    \addplot[mark=none, color=lightblue] coordinates{(0.01,0.1) (100, 1000)};
    \addplot[mark=none, color=lightblue] coordinates{(0.01,1.0) (10, 1000)};
    \addplot[color=black,dashed] coordinates{(0.01,1500) (1500,1500)};
    \addplot[color=black,dashed] coordinates{(1500,0.01) (1500,1500)};
    \node[left] at (axis cs: 5.0,0.06) {$0{.}01\times$};
    \node[left] at (axis cs: 0.5,0.06) {$0{.}1\times$};
    \node[left] at (axis cs: 0.05,0.06) {$1\times$};
    \node[left] at (axis cs: 0.05,0.6) {$10\times$};
    \node[left] at (axis cs: 0.05,6.0) {$100\times$};

          \end{axis}
\end{tikzpicture}
} % centering
\caption{Monolithic versus Structured Validation: Time}
\label{fig:monolithic:time}
\end{figure}

\begin{figure}
\centering{%
\begin{tikzpicture}[scale = 0.90]
  \begin{axis}[mark options={scale=0.55},grid=both, grid style={black!10},
      legend style={at={(0.22,0.98)}},
      legend cell align={left},
                              x post scale=1.8, y post scale=2.2,
                              xmode=log,xmin=1e3,xmax=1e7,
                              ymode=log, ymin=1e3, ymax=1e7,
                              xlabel={Structured (total clauses)}, ylabel={Monolithic (total clauses)},
            ]
    \input{data-formatted/clauses-total-structured+monolithic}
%    \addplot[mark=none, color=black] coordinates{(0.01,1500) (1500,1500)};
%    \addplot[mark=none, color=black] coordinates{(1500,0.01) (1500,1500)};
%    \addplot[mark=none, color=lightblue] coordinates{(10.0,0.01) (10000.0,10.0)};
    \addplot[mark=none, color=lightblue] coordinates{(1e3,1e4) (1e7,1e8)};
    \addplot[mark=none, color=lightblue] coordinates{(1e3,1e3) (1e7,1e7)};
    \addplot[mark=none, color=lightblue] coordinates{(1e3,1e2) (1e7,1e6)};
%    \addplot[mark=none, color=lightblue] coordinates{(0.01,10.0) (1000,1000000)};
%    \node[left] at (axis cs: 50,0.06) {$0{.}001\times$};
    \node[left] at (axis cs: 3500,40000) {$10\times$};
    \node[left] at (axis cs: 3500,4000) {$1\times$};
    \node[left] at (axis cs: 35000,4000) {$0.1\times$};
%    \node[left] at (axis cs: 0.05,60.0) {$1000\times$};

          \end{axis}
\end{tikzpicture}
} % centering
\caption{Monolithic versus Structured Validation: Clauses}
\label{fig:monolithic:clauses}
\end{figure}

Ran all benchmarks for which generation + checking could complete in 1000 seconds.  Total = 82.  All completed in less than 1000 seconds with structured.
Compare both runtimes and total number of clauses for structured vs. monolithic:
\begin{itemize}
\item Structured complete 82.
\item Monolithic completed  69, timed out on 13.
\item Of the 69 completed, 23 were faster with monolithic, 36 were slower (min ratio = 0.015, max ratio = 9.66, hmean = 0.40)
\item Of the 69 completed, 59 had fewer total clauses with monolithic, 10 had more (min ratio = 0.07, max ratio = 1.09, hmean = 0.49)
\end{itemize}



\section{Comparison with MICE}

\begin{figure}
\centering{%
\begin{tikzpicture}[scale = 0.90]
  \begin{axis}[mark options={scale=0.55},grid=both, grid style={black!10},
      legend style={at={(0.22,0.97)}},
      legend cell align={left},
                              x post scale=1.8, y post scale=2.2,
                              xmode=log,xmin=0.01,xmax=1900,
                              xtick={0.01, 0.1,1.0,10,100,1000}, xticklabels={0.01, 0.1, 1.0, 10, 100,{1,000}},
                              ymode=log, ymin=0.01, ymax=1900,
                              ytick={0.01, 0.1,1.0,10,100,1000}, yticklabels={0.01, 0.1, 1.0, 10, 100, {1,000}},
                              xlabel={CRAT total time (seconds)}, ylabel={MICE total time (seconds)},
            ]
    \input{data-formatted/time-mice-compare}
    \input{data-formatted/time-mice-error}
    \legend{
      \scriptsize \textsf{Correct or timeout},
      \scriptsize \textsf{MICE error},
    }
    \input{data-formatted/time-mice-timeout}

    \addplot[color=black,dashed] coordinates{(0.01,1500) (1500,1500)};
    \addplot[color=black,dashed] coordinates{(1500,0.01) (1500,1500)};

%    \addplot[mark=none, color=black] coordinates{(0.01,1500) (1500,1500)};
%    \addplot[mark=none, color=black] coordinates{(1500,0.01) (1500,1500)};
    \addplot[mark=none, color=lightblue] coordinates{(10.0,0.01) (1000.0,1.0)};
    \addplot[mark=none, color=lightblue] coordinates{(1.0,0.01) (1000.0,10.0)};
    \addplot[mark=none, color=lightblue] coordinates{(0.1,0.01) (1000.0,100.0)};
    \addplot[mark=none, color=lightblue] coordinates{(0.01,0.01) (1000.0,1000.0)};
    \addplot[mark=none, color=lightblue] coordinates{(0.01,0.1) (100,1000)};
    \addplot[mark=none, color=lightblue] coordinates{(0.01,1.0) (10,1000)};
    \addplot[mark=none, color=lightblue] coordinates{(0.01,10.0) (1,1000)};
    \node[left] at (axis cs: 50,0.06) {$0{.}001\times$};
    \node[left] at (axis cs: 5.0,0.06) {$0{.}01\times$};
    \node[left] at (axis cs: 0.5,0.06) {$0{.}1\times$};
    \node[left] at (axis cs: 0.05,0.06) {$1\times$};
    \node[left] at (axis cs: 0.05,0.6) {$10\times$};
    \node[left] at (axis cs: 0.05,6.0) {$100\times$};
    \node[left] at (axis cs: 0.05,60.0) {$1000\times$};

          \end{axis}
\end{tikzpicture}
} % centering
\caption{Running Time for MICE versus CRAT proof chains}
\label{fig:mice}
\end{figure}

For comparison, we also ran the MICE tool chain for certifying model
counters~\cite{fichte:sat:2022}.  Figure~\ref{fig:mice} compares the
runtimes of our toolchain versus theirs.  The X-axis shows the sum of
the elapsed times for our generator, checker, and counter (since the MICE checker also provides a count), while the
Y-axis shows the sum of the elapsed times for \textsc{nnf2trace},
generating a file in their trace format from a d-DNNF graph generated
by D4, and for \textsc{sharptrace}, a checker for the trace format.
The plot shows data points for those benchmarks for at least one of
the two tools completed in less than 1000 seconds.

We encountered a significant number of errors with both
\textsc{nnf2trace} and \textsc{sharptrace}.  Based on the error
messages, these appear to be mostly due to features of some of the
CNF files or d-DNNF graphs they had not anticipated.  We provided a bug report to
the authors, but they were unable to fix them on short notice.

In the figure, red dots indicate cases where neither program had an
error, but it possibly timed out.  The time-out cases are shown along
the upper edge (MICE) or the righthand edge (CRAT).

As can be seen, there is not a strong correlation between the program
runtimes.  This is, in part, because the two tools perform different
roles: one validates the algorithmic steps used by a top-down model
counting problem, while the other verifies the logical equivalence of
a Boolean formula and a representation for which model counting can be
done efficiently.

Overall, we can summarize the results as:
\begin{itemize}
\item 84 total data points
\item 67 were completed by both programs in under 1000 seconds
\item Of those 21 were faster with MICE, 46 with CRAT
\item 7 completed with CRAT under 1000 seconds but exceeded that threshold for MICE
\item 1 completed with MICE under 1000 seconds but exceeded that threshold for CRAT
\item 9 completed with CRAT but generated errors for MICE.
\end{itemize}


\bibliography{references}


\end{document}


\begin{figure}
\centering{%
\begin{tikzpicture}[scale = 0.90]
  \begin{axis}[mark options={scale=0.55},grid=both, grid style={black!10}, ymode=log, ymin=100, ymax=1000000000, 
                              x post scale=1.8, y post scale=1.8,
                              xmin=0,xmax=1000,
                              xtick = {0,100,200,300,400,500,600,700,800,900,1000},
                              ytick={100,1000, 10000, 100000, 1000000, 10000000, 100000000, 1000000000}, yticklabels={$10^2$,$10^3$,$10^4$,$10^5$,$10^6$,$10^7$,$10^8$,$10^9$},
                              xlabel={D4 runtime (seconds)}, ylabel={Defining Clauses},
%                              title={D4 Defining Clause Generation}
            ]
    \input{data-formatted/time+defining-success}
    \input{data-formatted/time+defining-failure}
          \end{axis}
\end{tikzpicture}
} % centering
\caption{Number of defining clauses in generated POG as function of D4 runtime (X linear scale)}
\label{fig:generate:d4}
\end{figure}

\begin{description}
\item[Lemmas:] The justification proof must show that the conjunction
  of the input clauses implies the extension variable associated with
  the root node.  It does so by showing that this property holds for
  every node and for every path from the root that node, relative to
  the partial assignment given by that path.  This could potentially
  require enumerating exponential numbers of paths to some nodes,
  effectively converting the graph into a tree.  Instead, the proof
  can contain a {\em lemma} for each node having indegree greater than
  one, showing that the root node is implied by a set of clauses.
  Some of these are the original input clauses, while some are {\em
    synthetic clauses} representing input clauses that have been
  reduced by eliminating false literals.  Given such a proof, the
  lemma can be instantiated for each parent node.  Proofs with lemmas
  incur some overhead---the synthetic clauses are declared as POG
  product nodes, and additional clauses must be added to show that,
  under a given context, the synthetic clauses are implied by some
  input or containing synthetic clause.  On the other hand, they
  guarantee that no node in the graph is the root of more than one
  justication proof.
\item[Literal grouping:] The generated POG representation will
  typically have a number of product nodes forming the conjunction of
  tens, or even hundreds, of literals.  These literals represent ones
  that we either implied by unit propagation or were added when
  assigning the complement to the corresponding variable led to a
  conflict.  Justifying this product node requires showing that each
  of these literals must hold under the given context.  When this
  cannot be done by unit propagation, the program must invoke a SAT
  solver to prove that the complemented value yields a conflict.
  Rather than doing so for each literal individually, the prover can
  create a new product node consisting of the literals that have not
  yet been validated and then prove that falsifying the extension
  variable associated with the product node yields a conflict.  
\end{description}
