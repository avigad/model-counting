\documentclass[letterpaper,USenglish,cleveref, autoref, thm-restate]{lipics-v2021}
%This is a template for producing LIPIcs articles.
%See lipics-v2021-authors-guidelines.pdf for further information.
%for A4 paper format use option "a4paper", for US-letter use option "letterpaper"
%for british hyphenation rules use option "UKenglish", for american hyphenation rules use option "USenglish"
%for section-numbered lemmas etc., use "numberwithinsect"
%for enabling cleveref support, use "cleveref"
%for enabling autoref support, use "autoref"
%for anonymousing the authors (e.g. for double-blind review), add "anonymous"
%for enabling thm-restate support, use "thm-restate"
%for enabling a two-column layout for the author/affilation part (only applicable for > 6 authors), use "authorcolumns"
%for producing a PDF according the PDF/A standard, add "pdfa"

%\pdfoutput=1 %uncomment to ensure pdflatex processing (mandatatory e.g. to submit to arXiv)
%\hideLIPIcs  %uncomment to remove references to LIPIcs series (logo, DOI, ...), e.g. when preparing a pre-final version to be uploaded to arXiv or another public repository

%\graphicspath{{./graphics/}}%helpful if your graphic files are in another directory

\usepackage{amsmath}
\usepackage{amssymb}
\usepackage{tikz}
\usepackage{pgfplots}
\usepackage{booktabs}
\usepackage{hyperref}

\newcommand{\pand}{\mathbin{\land^\textsf{p}}}
\newcommand{\por}{\mathbin{\lor^\textsf{p}}}
\DeclareMathOperator*{\Pand}{\bigwedge^\textsf{p}}
\DeclareMathOperator*{\Por}{\bigvee^\textsf{p}}
\newcommand{\boolnot}{\neg}
\newcommand{\tautology}{\top}
\newcommand{\nil}{\bot}
\newcommand{\obar}[1]{\overline{#1}}
\newcommand{\oneminus}{{\sim}}
\newcommand{\lit}{\ell}

\newcommand{\varset}{X}
\newcommand{\exvarset}{Z}
\newcommand{\dependencyset}{{\cal D}}
\newcommand{\litset}{{\cal L}}
\newcommand{\ring}{{\cal R}}
\newcommand{\dset}{{\cal A}}
\newcommand{\rep}{\textbf{R}}
\newcommand{\radd}{+}
\newcommand{\rmul}{\times}
\newcommand{\addident}{\textbf{0}}
\newcommand{\mulident}{\textbf{1}}
\newcommand{\imply}{\Rightarrow}
\newcommand{\ifandonlyif}{\Leftrightarrow}
%\newcommand{\drational}{\mathbb{Q}_{2,5}}
\newcommand{\drational}{\textbf{Q}_{2,5}}
\newcommand{\entail}{\vDash}
\newcommand{\entaildrat}{\entail_{\textrm{DRAT}}}

\newcommand{\assign}{\alpha}
\newcommand{\passign}{\rho}
\newcommand{\lassign}{\beta}
\newcommand{\uassign}{{\cal U}}
\newcommand{\modelset}{{\cal M}}

\newcommand{\indegree}{\textrm{indegree}}
\newcommand{\outdegree}{\textrm{outdegree}}
\newcommand{\validate}{\textsf{validate}}
\newcommand{\prov}{\textrm{Prov}}
\newcommand{\inputformula}{\phi_I}
\newcommand{\pogformula}{\theta_P}

\newcommand{\makenode}[1]{\mathbf{#1}}
\newcommand{\nodeu}{\makenode{u}}
\newcommand{\nodev}{\makenode{v}}
\newcommand{\nodes}{\makenode{s}}
\newcommand{\nodep}{\makenode{p}}
\newcommand{\noder}{\makenode{r}}

\newcommand{\simplify}[2]{#1|_{#2}}

\newcommand{\progname}[1]{\textsc{#1}}
\newcommand{\dfour}{\progname{D4}}
\newcommand{\cdfour}{\progname{CD4}}
\newcommand{\Dfour}{\progname{D4}}
\newcommand{\cadical}{\progname{CaDiCal}}
\newcommand{\dtrim}{\progname{drat-trim}}
\newcommand{\Dtrim}{\progname{Drat-trim}}

\definecolor{redorange}{rgb}{0.878431, 0.235294, 0.192157}
\definecolor{lightblue}{rgb}{0.552941, 0.72549, 0.792157}
\definecolor{clearyellow}{rgb}{0.964706, 0.745098, 0}
\definecolor{clearorange}{rgb}{0.917647, 0.462745, 0}
\definecolor{mildgray}{rgb}{0.54902, 0.509804, 0.47451}
\definecolor{softblue}{rgb}{0.643137, 0.858824, 0.909804}
\definecolor{bluegray}{rgb}{0.141176, 0.313725, 0.603922}
\definecolor{lightgreen}{rgb}{0.709804, 0.741176, 0}
\definecolor{darkgreen}{rgb}{0.152941, 0.576471, 0.172549}
\definecolor{redpurple}{rgb}{0.835294, 0, 0.196078}
\definecolor{midblue}{rgb}{0, 0.592157, 0.662745}
\definecolor{clearpurple}{rgb}{0.67451, 0.0784314, 0.352941}
\definecolor{browngreen}{rgb}{0.333333, 0.313725, 0.145098}
\definecolor{darkestpurple}{rgb}{0.396078, 0.113725, 0.196078}
\definecolor{greypurple}{rgb}{0.294118, 0.219608, 0.298039}
\definecolor{darkturquoise}{rgb}{0, 0.239216, 0.298039}
\definecolor{darkbrown}{rgb}{0.305882, 0.211765, 0.160784}
\definecolor{midgreen}{rgb}{0.560784, 0.6, 0.243137}
\definecolor{darkred}{rgb}{0.576471, 0.152941, 0.172549}
\definecolor{darkpurple}{rgb}{0.313725, 0.027451, 0.470588}
\definecolor{darkestblue}{rgb}{0, 0.156863, 0.333333}
\definecolor{lightpurple}{rgb}{0.776471, 0.690196, 0.737255}
\definecolor{softgreen}{rgb}{0.733333, 0.772549, 0.572549}
\definecolor{offwhite}{rgb}{0.839216, 0.823529, 0.768627}
\definecolor{medgreen}{rgb}{0.15, 0.6, 0.15}

% Lean code:
\usepackage{listings}
%\definecolor{keywordcolor}{rgb}{0.7, 0.1, 0.1}   % red
\definecolor{keywordcolor}{rgb}{0.0, 0.1, 0.6}   % blue
\definecolor{tacticcolor}{rgb}{0.0, 0.1, 0.6}    % blue
\definecolor{commentcolor}{rgb}{0.4, 0.4, 0.4}   % grey
\definecolor{symbolcolor}{rgb}{0.0, 0.1, 0.6}    % blue
\definecolor{sortcolor}{rgb}{0.1, 0.5, 0.1}      % green
\definecolor{attributecolor}{rgb}{0.7, 0.1, 0.1} % red
\def\lstlanguagefiles{lstlean.tex}
% set default language
\lstset{language=lean, xleftmargin=1em}
\lstset{backgroundcolor=\color{white}}

\bibliographystyle{plainurl}% the mandatory bibstyle

\title{SUPPLEMENT to \\ Certified Knowledge Compilation \\ with Application to Verified Model Counting}
\hideLIPIcs

\titlerunning{Certified Knowledge Compilation}

\author{Randal E. Bryant}{Computer Science Department, Carnegie Mellon University, Pittsburgh, PA 15213 USA}{Randy.Bryant@cs.cmu.edu}{https://orcid.org/0000-0001-5024-6613}{Supported by NSF grant CCF-2108521}
\author{Wojciech Nawrocki}{Department of Philosophy, Carnegie Mellon University}{wjnawrocki@cmu.edu}{https://orcid.org/0000-0002-8839-0618}{Hoskinson Center for Formal Mathematics}
\author{Jeremy Avigad}{Department of Philosophy, Carnegie Mellon University}{avigad@cmu.edu}{https://orcid.org/0000-0003-1275-315X}{Hoskinson Center for Formal Mathematics}
\author{Marijn J. H. Heule}{Computer Science Department, Carnegie Mellon University}{marijn@cmu.edu}{https://orcid.org/0000-0002-5587-8801}{Supported by NSF grant CCF-2108521}

\authorrunning{R. E. Bryant et al.} %TODO mandatory. First: Use abbreviated first/middle names. Second (only in severe cases): Use first author plus 'et al.'

\Copyright{Randal E. Bryant, Wojciech Nawrocki, Jeremy Avigad, and Marijn J. H. Heule} %TODO mandatory, please use full first names. LIPIcs license is "CC-BY";  http://creativecommons.org/licenses/by/3.0/

%\ccsdesc[100]{\textcolor{red}{Replace ccsdesc macro with valid one}} %TODO mandatory: Please choose ACM 2012 classifications from https://dl.acm.org/ccs/ccs_flat.cfm
\begin{CCSXML}
<ccs2012>
   <concept>
       <concept_id>10003752.10003790.10003794</concept_id>
       <concept_desc>Theory of computation~Automated reasoning</concept_desc>
       <concept_significance>500</concept_significance>
       </concept>
 </ccs2012>
\end{CCSXML}

\ccsdesc[500]{Theory of computation~Automated reasoning}


%\keywords{Dummy keyword} %TODO mandatory; please add comma-separated list of keywords
\keywords{Propositional model counting, Proof checking}

%\category{} %optional, e.g. invited paper

%\funding{(Optional) general funding statement \dots}%optional, to capture a funding statement, which applies to all authors. Please enter author specific funding statements as fifth argument of the \author macro.

\nolinenumbers %uncomment to disable line numbering

\EventEditors{Meena Mahajan and Friedrich Slivovsky}
\EventNoEds{2}
\EventLongTitle{26th International Conference on Theory and Applications of Satisfiability Testing (SAT 2023)}
\EventShortTitle{SAT 2023}
\EventAcronym{SAT}
\EventYear{2023}
\EventDate{July 4--8, 2023}
\EventLocation{Alghero, Italy}
\EventLogo{}
\SeriesVolume{271}
\ArticleNo{6}

%Editor-only macros:: begin (do not touch as author)%%%%%%%%%%%%%%%%%%%%%%%%%%%%%%%%%%
%\EventEditors{John Q. Open and Joan R. Access}
%\EventNoEds{2}
%\EventLongTitle{42nd Conference on Very Important Topics (CVIT 2016)}
%\EventShortTitle{CVIT 2016}
%\EventAcronym{CVIT}
%\EventYear{2016}
%\EventDate{December 24--27, 2016}
%\EventLocation{Little Whinging, United Kingdom}
%\EventLogo{}
%\SeriesVolume{42}
%\ArticleNo{23}
%%%%%%%%%%%%%%%%%%%%%%%%%%%%%%%%%%%%%%%%%%%%%%%%%%%%%%

\newcommand{\program}[1]{\textsc{#1}}
\newcommand{\lean}{Lean~4}

\begin{document}

\maketitle



This document is a supplement to the paper
``Certified Knowledge Compilation with Applications to Verified Model Counting,''
published at the 2023 Conference on Theory and Applications of SAT Testing (SAT).


\section{CPOG Example}
\label{app:cpog:example}

\begin{figure}
\begin{minipage}{0.58\textwidth}
(A)  Input Formula\\[1.2ex]
\begin{tabular}{lll}
\toprule
\makebox[5mm]{ID} & \makebox[15mm]{Clauses} & \\
\midrule
1 & \texttt{-1 3 -4} & \texttt{0} \\
2 & \texttt{-1 -3 4} & \texttt{0} \\
3 & \texttt{3 -4} & \texttt{0}\\
4 & \texttt{1 -3 4} & \texttt{0} \\
5 & \texttt{-1 -2} & \texttt{0} \\
\bottomrule
\end{tabular}
\\[1.8ex]
(C) POG Declaration\\[1.2ex]
\begin{tabular}{llll}
\toprule
\makebox[5mm]{ID} & \multicolumn{2}{l}{CPOG line} & Explanation \\
\midrule
6 & \texttt{p 5 -3 -4} & \texttt{0} & $p_5 = \obar{x}_3 \pand \obar{x}_4$ \\
9 & \texttt{p 6 3 4} & \texttt{0} & $p_6 = x_3 \pand x_4$ \\
12 & \texttt{s 7 5 6 7 10} & \texttt{0} & $s_7 = p_5 \por p_6$ \\
15 & \texttt{p 8 -1 7} & \texttt{0} & $p_8 = \obar{x}_1 \pand s_7$ \\
18 & \texttt{p 9 1 -2 7} & \texttt{0} & $p_9 = x_1 \pand \obar{x}_2 \pand s_7$ \\
22 & \texttt{s 10 8 9 16 19} & \texttt{0} & $s_{10} = p_8 \por p_9$ \\
 & \texttt{r 10} && Root $r = s_{10}$\\
\bottomrule
\end{tabular}
\end{minipage}
\begin{minipage}{0.35\textwidth}
(B) POG Representation \\
\input{dd/eg4}
\end{minipage}
%% Break
\\[2.5ex]
\begin{minipage}{0.42\textwidth}
(D) Defining Clauses\\[1.2ex]
\begin{tabular}{llll}
\toprule
\makebox[5mm]{ID} & \multicolumn{2}{l}{Clauses} & Explanation \\
\midrule
\texttt{6} & \texttt{5 3 4} & \texttt{0} & Define $p_5$ \\
\texttt{7} & \texttt{-5 -3} & \texttt{0} & \\
\texttt{8} & \texttt{-5 -4} & \texttt{0} & \\
\midrule
\texttt{9} & \texttt{6 -3 -4} & \texttt{0} & Define $p_6$ \\
\texttt{10} & \texttt{-6 3} & \texttt{0} & \\
\texttt{11} & \texttt{-6 4} & \texttt{0} & \\
\midrule
\texttt{12} & \texttt{-7 5 6} & \texttt{0} & Define $s_7$ \\
\texttt{13} & \texttt{7 -5} & \texttt{0} & \\
\texttt{14} & \texttt{7 -6} & \texttt{0} & \\
\midrule
\texttt{15} & \texttt{8 1 -7} & \texttt{0} & Define $p_8$ \\
\texttt{16} & \texttt{-8 -1} & \texttt{0} & \\
\texttt{17} & \texttt{-8 7} & \texttt{0} & \\
\midrule
\texttt{18} & \texttt{9 -1 2 -7} & \texttt{0} & Define $p_9$ \\
\texttt{19} & \texttt{-9 1} & \texttt{0} & \\
\texttt{20} & \texttt{-9 -2} & \texttt{0} & \\
\texttt{21} & \texttt{-9 7} & \texttt{0} & \\
\midrule
\texttt{22} & \texttt{-10 8 9} & \texttt{0} & Define $s_{10}$ \\
\texttt{23} & \texttt{10 -8} & \texttt{0} & \\
\texttt{24} & \texttt{10 -9} & \texttt{0} & \\
\bottomrule
\end{tabular}
\end{minipage}
\begin{minipage}{0.49\textwidth}
(E) CPOG Assertions\\[1.2ex]
\begin{tabular}{llllll}
\toprule
\makebox[5mm]{ID} & \multicolumn{2}{l}{Clause} & \multicolumn{2}{l}{Hint} & Explanation \\
\midrule
\texttt{25} & \texttt{a 5 1 3} & \texttt{0} & \texttt{3 6} & \texttt{0} & $\obar{x}_1 \land \obar{x}_3 \imply p_5$ \\
\texttt{26} & \texttt{a 6 1 -3} & \texttt{0} & \texttt{4 9} & \texttt{0} & $\obar{x}_1 \land x_3 \imply p_6$ \\
\texttt{27} & \texttt{a 3 7 1} & \texttt{0} & \texttt{13 25} & \texttt{0} & $\obar{x}_3 \land \obar{x}_1 \imply s_7$  \\
\texttt{28} & \texttt{a 7 1} & \texttt{0} & \texttt{27 14 26} & \texttt{0} & $\obar{x}_1 \imply s_7$  \\
\texttt{29} & \texttt{a 8 1} & \texttt{0} & \texttt{28 15} & \texttt{0} & $\obar{x}_1 \imply p_8$  \\
\texttt{30} & \texttt{a 5 -1 3} & \texttt{0} & \texttt{1 6} & \texttt{0} & $x_1 \land \obar{x}_3 \imply p_5$ \\
\texttt{31} & \texttt{a 6 -1 -3} & \texttt{0} & \texttt{2 9} & \texttt{0} & $x_1 \land x_3 \imply p_6$ \\
\texttt{32} & \texttt{a 3 7 -1} & \texttt{0} & \texttt{13 30} & \texttt{0} & $\obar{x}_3 \land x_1 \imply s_7$  \\
\texttt{33} & \texttt{a 7 -1} & \texttt{0} & \texttt{32 14 31} & \texttt{0} & $x_1 \imply s_7$  \\
\texttt{34} & \texttt{a 9 -1} & \texttt{0} & \texttt{5 33 18} & \texttt{0} & $x_1 \imply p_9$  \\
\texttt{35} & \texttt{a 1 10} & \texttt{0} & \texttt{23 29} & \texttt{0} & $\obar{x}_1 \imply s_{10}$  \\
\texttt{36} & \texttt{a 10} & \texttt{0} & \texttt{35 24 34} & \texttt{0} & $s_{10}$ \\
\bottomrule
\end{tabular}
\\[1.5ex]
(F) Input Clause Deletions\\[1.2ex]
\begin{tabular}{lllll}
  \toprule
 \multicolumn{3}{l}{CPOG line} & Explanation\\
\midrule
 \texttt{d 1} & \texttt{36 8 10 12 16 21 22} & \texttt{0} & Delete clause 1 \\
 \texttt{d 2} & \texttt{36 7 11 12 16 21 22} & \texttt{0} & Delete clause 2 \\
 \texttt{d 3} & \texttt{36 8 10 12 17 19 22} & \texttt{0} & Delete clause 3 \\
 \texttt{d 4} & \texttt{36 7 11 12 17 19 22} & \texttt{0} & Delete clause 4 \\
 \texttt{d 5} & \texttt{36 16 20 22} & \texttt{0} &  Delete clause 5 \\
\bottomrule
\end{tabular}
\end{minipage}
\caption{Example formula (A), its POG representation (B), and its CPOG proof (C), (E), and (F)}
\label{fig:eg4:all}
\end{figure}

Figure \ref{fig:eg4:all} illustrates an example formula and shows how
the CPOG file declares its POG representation.  The input formula (A)
consists of five clauses over variables $x_1$, $x_2$, $x_3$, and
$x_4$.  The generated POG (B) has six nonterminal nodes representing
four products and two sums.  We name these by the node
type (product $\nodep$ or sum $\nodes$), subscripted by the ID of the
extension variable.
% We denote the extension variable for nodes $\nodep_i$ and $\nodes_j$ as $p_i$ and $s_j$, respectively.
  The first part of the CPOG file (C) declares
these nodes using clause IDs that increment by three or four,
depending on whether the node has two children or three.  The last two
nonzero values in the sum declarations are the hint providing the
required mutual exclusion proof.

\subsection{Node Declarations}

We step through portions of the file to provide a better understanding of the CPOG proof framework.
Figure
\ref{fig:eg4:all}(D) shows the defining clauses that are implicitly
defined by the POG operation declarations.  These do not appear in the
CPOG file.  Referring back to the declarations of the sum nodes in
Figure \ref{fig:eg4:all}(C), we can see that the declaration of node
$\nodes_7$ had clause IDs 7 and 10 as the hint.  We can see in Figure
\ref{fig:eg4:all}(A) that these two clauses form a RUP proof for the clause
$\obar{p}_5 \lor \obar{p}_6$, showing that the two children of $\nodes_7$
have disjoint models.  Similarly, node $\nodes_{10}$ is declared as having
clause IDs 16 and 19 as the hint.  These form a RUP proof for the clause
$\obar{p}_8 \lor \obar{p}_9$, showing that the two children of
$\nodes_{10}$ have disjoint models.

\subsection{Forward Implication Proof}

Figure \ref{fig:eg4:all}(E) provides the sequence of assertions
leading to unit clause 36, consisting of the literal $s_{10}$.  This clause indicates that $\nodes_{10}$ is implied by the input clauses, i.e.,
any total assignment $\assign$
satisfying the input clauses must have $\assign(s_{10}) = 1$.
Working backward, we can see that
clause 29 indicates that variable $p_8$ will be implied by the input
clauses when $\assign(x_1) = 0$, while clause 34 indicates that node $p_9$ will
be implied by the input clauses when $\assign(x_1) = 1$.  These serve as the
hint for clause 36.

\subsection{Reverse Implication Proof}

Figure \ref{fig:eg4:all}(F) shows the RUP proof steps required to
delete the input clauses.  Consider the first of these, deleting
input clause $\obar{x}_1 \lor x_3 \lor \obar{x}_4$.  The requirement is to show
that there is no total assignment $\assign$ that falsifies this clause but assigns $\assign(s_{10}) = 1$.
The proof proceeds by first assuming that the clause is false, requiring
$\assign(x_1) = 1$, $\assign(x_3) = 0$, and $\assign(x_4) = 1$.  The hint then consist of unit
clauses (e.g., clause 36 asserting that $\alpha(p_{10}) = 1$) or
clauses that cause unit propagation.  Hint clauses 8 and 10 force the
assignments $\assign(p_5) = \assign(p_6) = 0$.  These, plus hint clause 12 force
$\assign(s_7) = 0$.  This, plus hint clauses 16 and 21 force $\assign(p_8) = \assign(p_9) = 0$, leading
via clause 22 to $\assign(s_{10}) = 0$.  But this contradicts clause 36,
completing the RUP proof.  The deletion hints for the other input
clauses follow similar patterns---they work from the bottom nodes of
the POG upward, showing that any total assignment that falsifies the clause
must assign $\assign(s_{10}) = 0$.

Deleting the asserted clauses is so simple that we do not show it.  It
involves simply deleting the clauses from clause number 35 down to
clause number 25, with each deletion using the same hint as were used
to add that clause.  In the end, therefore, only the defining clauses
for the POG nodes and the unit clause asserting $s_{10}$ remain,
completing a proof that the POG is logically equivalent to the input
formula.

Observe that the forward implication proof shown in Figure~\ref{fig:eg4:all}  must ``visit'' nodes
$\nodep_5$ and $\nodep_6$ twice, separately considering  assignments where $\assign(x_1) = 1$
(clauses 25 and 26) and where $\assign(x_1) = 1$ (clauses 30 and 31).
Section~\ref{app:lemma:eg} shows how to use a lemma such that these nodes are each only visited once.


\section{Optimizations for Forward Implication Proofs}

The two optimizations we have implemented exploit the power of our
general resolution framework to define new structures within the
proof.
They create new product nodes that are not part
of the POG representation.

\label{app:optimizations}

\subsection{Literal Grouping}

A single recursive step of $\validate$ can encounter product nodes
having many literals as children.  The naive formulation of $\validate$
considers each literal $\lit \in \lambda$ separately.
Literal grouping allows all literals to be validated with a single call to a SAT solver.
It collects those literals
$\lit_1, \lit_2, \ldots, \lit_m$ that cannot be validated by BCP and defines a
product node $\nodev$ having these literals as children.  The goal
then becomes to prove that any total assignment must yield 1 for extension
variable $v$.  A single call to the solver can generate this proof by invoking it on the formula
  $\simplify{\psi}{\passign} \cup \simplify{\theta_{v}}{\{ \obar{v} \}}$, which should be unsatisfiable.
  The proof steps can be mapped back into clause addition steps in the CPOG file, incorporating the
  input clauses and the defining clauses for $\nodev$ into the hints.


\subsection{Lemmas}
\label{app:lemma}

As we have noted, the recursive calling of $\validate$ starting at
root $\noder$ effectively expands the POG into a tree, and this can
lead to an exponential number of calls.
These shared subgraphs arise when the knowledge compiler employs \emph{clause caching}
to detect that the simplified set of
clauses arising from one partial assignment to the literals matches that
of a previous partial assignment~\cite{darwiche:aaai:2002}.
When this dec-DNNF node is translated into POG
node $\nodeu$, the proof generator can assume (and also check), that
there is a simplified set of clauses $\gamma_{\nodeu}$
for which the subgraph with root $\nodeu$ is its POG representation.

The proof generator can exploit the sharing of subgraphs
by constructing and proving a \emph{lemma} for each node
$\nodeu$ having $\indegree(\nodeu) > 1$.  This proof shows that any
total assignment $\assign$ that satisfies formula $\gamma_{\nodeu}$ and the defining clauses for the POG must yield
$\assign(u) = 1$.  This lemma is then invoked for every node having
$\nodeu$ as a child.
As a result, the generator will make recursive calls during a call to $\validate$ only once for each node in the POG\@.

The challenge for implementing this strategy is to find a way to
represent the clauses for the simplified formula $\gamma_{\nodeu}$ in the CPOG file.  Some may be
unaltered input clauses, and these can be used directly.  Others,
however will be clauses that do not appear in the input formula.  We
implement these by adding POG product nodes to the CPOG file to create
the appropriate clauses.  Consider an \emph{argument} clause
$C \in \gamma_{\nodeu}$ with $C = \lit_1 \lor \lit_2 \lor \cdots \lor \lit_k$.  If we
define a product node $\nodev$ with arguments
$\obar{\lit}_1, \obar{\lit}_2, \ldots, \obar{\lit}_k$, 
we will introduce a defining clause
$v \lor \lit_1 \lor \lit_2 \lor \cdots \lit_k$.  We call this a {\em
  synthetic} clause having $\obar{v}$ as the \emph{guard literal}.
That is, a partial assignment $\passign$ such that $\passign(v) = 0$ will {\em
  activate} the clause, causing it to represent argument clause $C$.  On the other
hand, a partial assignment with $\passign(v) = 1$ will
cause the clause to become a tautology and therefore have no effect.

Suppose for every clause $C_j \in \gamma_{\nodeu}$ that does not correspond to
an input clause, we generate a synthetic clause $C'_j$ with guard literal
$\obar{v}_j$, for $1 \leq j \leq m$.  Let $\gamma'_{\nodeu}$ be the formula where each clause $C_j$ is replaced by synthetic clause $C'_j$,
while input clauses in $\gamma_{\nodeu}$ are left unchanged.
Let $\lassign = \{ \obar{v}_1, \obar{v}_2, \ldots, \obar{v}_m \}$.
Invoking $\validate(\nodeu, \lassign, \gamma'_{\nodeu})$
 will then prove a lemma, given by the target clause
 $u \lor v_1 \lor v_2 \lor \cdots \lor v_m$,
 showing that any total assignment $\assign$ that activates the synthetic clauses will have $\assign(u) = 1$.

Later, when node $\nodeu$ is encountered by a call to $\validate(\nodeu, \passign, \psi)$, we invoke the lemma
by showing that each synthetic clause
$C_j$ matches some simplified clause in $\simplify{\psi}{\passign}$.  More precisely,
for $1 \leq j \leq m$,
we use clause addition to assert the clause
$\obar{v}_j \lor \bigvee_{\lit \in \passign} \obar{\lit}$,
showing that synthetic clause $C_j$ will be activated.
Combining the lemma with these activations provides a derivation of the target clause for the call to $\validate$.

Observe that the lemma structure can be hierarchical, since a shared
subgraph may contain nodes that are themselves roots of shared
subgraphs.  Nonetheless, the principles described allow the
definition, proof, and applications of a lemma for each shared node in
the graph.  For any node $\nodeu$, the first call to
$\validate(\nodeu, \passign, \psi)$ may require further recursion,
but any subsequent call can simply reuse the lemma proved by the first call.

\subsection{Lemma Example}
\label{app:lemma:eg}

\begin{figure}
(A) Additional nodes\\[1.0em]
\begin{tabular}{llll}
\toprule
\makebox[5mm]{ID} & \multicolumn{2}{l}{CPOG line} & Explanation \\
\midrule
25 & \texttt{p 11 -3 4} & \texttt{0} & $v_{11} = \obar{x}_3 \pand {x}_4$ \\
28 & \texttt{p 12  3 -4} & \texttt{0} & $v_{12} = {x}_3 \pand \obar{x}_4$ \\
\bottomrule
\end{tabular}
\\[1.0em]
(B) Implicit Clauses\\[1.2em]
\begin{tabular}{llll}
\toprule
\makebox[5mm]{ID} & \multicolumn{2}{l}{Clauses} & Explanation \\
\midrule
\texttt{25} & \texttt{11 3 -4} & \texttt{0} & Argument clause $\{x_3 ,\, \obar{x}_4\}$, activated by $\obar{v}_{11}$ \\
\texttt{26} & \texttt{-11 -3} & \texttt{0} & \\
\texttt{27} & \texttt{-11 4} & \texttt{0} & \\
\midrule
\texttt{28} & \texttt{12 -3 4} & \texttt{0} & Argument clause $\{\obar{x}_3,\,  {x}_4\}$, activated by $\obar{v}_{12}$ \\
\texttt{29} & \texttt{-12 3} & \texttt{0} & \\
\texttt{30} & \texttt{-12 -4} & \texttt{0} & \\
\bottomrule
\end{tabular}
\\[1.0em]
(C) CPOG Assertions\\[1.0em]
\begin{tabular}{llllll}
\toprule
\makebox[5mm]{ID} & \multicolumn{2}{l}{Clause} & \multicolumn{2}{l}{Hint} & Explanation \\
\midrule
\multicolumn{6}{l}{Lemma Proof} \\
\texttt{31} & \texttt{a 5 11 12 3} & \texttt{0} & \texttt{25 6} & \texttt{0} & $(\obar{v}_{11} \land \obar{v}_{12}) \land \obar{x}_3 \imply p_{5}$ \\
\texttt{32} & \texttt{a 6 11 12 -3} & \texttt{0} & \texttt{28 9} & \texttt{0} & $(\obar{v}_{11} \land \obar{v}_{12}) \land {x}_3 \imply p_{6}$ \\
\texttt{33} & \texttt{a 3 7 11 12} & \texttt{0} & \texttt{13 31} & \texttt{0} & $(\obar{v}_{11} \land \obar{v}_{12}) \land \obar{x}_3 \imply s_{7}$ \\
\texttt{34} & \texttt{a 7 11 12} & \texttt{0} & \texttt{33 14 32} & \texttt{0} & $(\obar{v}_{11} \land \obar{v}_{12}) \imply s_{7}$ \\
\midrule
\multicolumn{6}{l}{Lemma Application \#1} \\
\texttt{35} & \texttt{a -11 1} & \texttt{0} & \texttt{26 27 3} & \texttt{0} & $\obar{x}_1 \imply \obar{v}_{11}$ \\
\texttt{36} & \texttt{a -12 1} & \texttt{0} & \texttt{29 30 4} & \texttt{0} & $\obar{x}_1 \imply \obar{v}_{12}$ \\
\texttt{37} & \texttt{a 7 1} & \texttt{0} & \texttt{35 36 34} & \texttt{0} & $\obar{x}_1 \imply s_7$ \\
\midrule
\texttt{38} & \texttt{a 8 1} & \texttt{0} & \texttt{37 15} & \texttt{0} & $\obar{x}_1 \imply p_8$ \\
\midrule
\multicolumn{6}{l}{Lemma Application \#2} \\
\texttt{39} & \texttt{a -11 -1} & \texttt{0} & \texttt{26 27 1} & \texttt{0} & ${x}_1 \imply \obar{v}_{11}$ \\
\texttt{40} & \texttt{a -12 -1} & \texttt{0} & \texttt{29 30 2} & \texttt{0} & ${x}_1 \imply \obar{v}_{12}$ \\
\texttt{41} & \texttt{a 7 -1} & \texttt{0} & \texttt{39 40 34} & \texttt{0} & ${x}_1 \imply s_7$ \\
\midrule
\texttt{42} & \texttt{a 9 -1} & \texttt{0} & \texttt{5 41 18} & \texttt{0} & ${x}_1 \imply p_9$ \\
\texttt{43} & \texttt{a 1 10} & \texttt{0} & \texttt{23 38} & \texttt{0} & $\obar{x}_1 \imply s_{10}$  \\
\texttt{44} & \texttt{a 10} & \texttt{0} & \texttt{43 24 42} & \texttt{0} & $s_{10}$ \\
\bottomrule
\end{tabular}
\caption{Example of lemma definition, proof, and application}
\label{fig:eg4:lemmas}
\end{figure}

Figure~\ref{fig:eg4:lemmas} shows an alternate forward implication
proof for the example of Figure~\ref{fig:eg4:all} using a lemma to
represent the shared node $\nodes_7$.  We can see that the POG with
this node as root encodes the Boolean formula $x_3 \leftrightarrow x_4$, having a CNF representation consisting of the clauses
$\{x_3 ,\, \obar{x}_4\}$ and $\{\obar{x}_3 ,\, {x}_4\}$.  The product node
declarations shown in Figure~\ref{fig:eg4:lemmas}(A) create synthetic
clauses 25 and 28 to encode these arguments with activating literals
$\obar{v}_{11}$ and $\obar{v}_{12}$, respectively.  Clauses 31--34
then provide a proof of the lemma, stating that any assignment
$\assign$ that activates these clauses will  assign $1$ to $s_7$.
Clauses 35 and 36 state that an assignment with $\assign(x_1) = 0$
will cause the first synthetic clause to activate due to input clause
3, and it will cause the second synthetic clause to activate due to
input clause 4.  From this, clause 37 can use the lemma to state that
assigning $0$ to $x_1$ will cause $s_7$ to evaluate to $1$.  Similarly,
clauses 39 and 40 serve to activate the synthetic clauses when
$\assign(x_1) = 1$, due to input clauses 1 and 2, and clause 41 then
uses the lemma to state that assigning $1$ to $x_1$ will cause $s_7$ to
evaluate to $1$.

In this example, adding the lemma increases the proof length, but that
is only because it is such a simple formula.


\section{More information about the formally verified checker}
\label{appendix:lean}

We have implemented the rightmost two components of Figure~\ref{fig:chain}, namely the
proof checker and both model and weighted model counters, in the \lean{} programming language
and proof assistant [cite]. \lean{} implements a logical foundation in which expressions
have a computational interpretation, and just as with other proof assistants like Isabelle and Coq,
a function defined within the formal system can be compiled to efficient code.
At the same time, we can state and prove claims about the function within the system, thereby
verifying that the functions compute the intended results.

Verifying our checker and the model counters in this way serves at least two purposes.
When we run them on particular data sets, they provide high confidence
that the particular results we obtain are correct.
More broadly, however, our success also validates our general approach,
showing, in particular, that the CRAT proof format is sound.
Indeed, our formalization helped us discover a bug in our implementation
of the informal proof checker: in the ``\emph{s}'' step described at the end of
Subsection~\ref{subsection:syntax}, it is sound (and necessary) to allow defining
clauses for the new operations in the proof of $\obar{\lit}_1 \lor \obar{\lit}_2$,
but it is generally not sound to allow the input clauses or other added clauses.
[NOTE: let's add this specification to the end of Section 6.1.]
Our verified implementation of the checker tells us that the algorithm is sound and
provides a reference for other implementations.

In this section, we describe the implementation in Lean and the specifications we have proved.
\lean{}'s logical foundation describes a functional programming language with
inductive data types,
but our procedures also make use of calculations with natural numbers, integers, and rationals,
and data structures for arrays and hashmaps.
Lean's code extraction uses efficient but unverified versions of these,
and so in Subsection~\ref{subsection:verification:specifics} we clarify
what exactly we have verified and what we currently trust.

\subsection{Logical objects and operations}

Our specifications are built on a generic library for the syntax and semantics of
propositional logic. The data type of propositional formulas with variables indexed by
a datatype \lstinline{ν} is defined inductively as follows:
\begin{lstlisting}
inductive PropForm (ν : Type)
  | var (x : ν)
  | tr
  | fls
  | neg (φ : PropForm ν)
  | conj (φ₁ φ₂ : PropForm ν)
  | disj (φ₁ φ₂ : PropForm ν)
  | impl (φ₁ φ₂ : PropForm ν)
  | biImpl (φ₁ φ₂ : PropForm ν)
\end{lstlisting}
The last two constructors denote implication and bi-implication, which are not used in this
project. The usual syntactic and semantic notions are defined in the expected ways.
It is convenient to take a truth assignment \lstinline{PropAssignment ν} to a function that
assigns a boolean value to every assignment, and later restrict attention to subsets of the
variables. In Lean, we introduce the notation \lstinline{τ ⊨ φ} for the statement that
\lstinline{τ} satisfies \lstinline{φ}. We write \lstinline{entails φ₁ φ₂} to say that
\lstinline{φ₁} and \lstinline{φ₂} and \lstinline{equivalent φ₁ φ₂}.
For the purpose of model counting, if \lstinline{s} is any finite set of variables,
we define \lstinline{models φ s} to be the finite set of truth assignments \lstinline{τ}
satisfying \lstinline{φ} that assign arbitrary values on \lstinline{s} and
\lstinline{false} outside \lstinline{s}.

In logic, the set of propositional formulas modulo equivalence is sometimes called the
\emph{Lindenbaum--Tarski algebra}. We have found it convenient to define this quotient,
which we denote \lstinline{PropTerm ν}, and lift the boolean operations and the entailment
relation to this new type. The advantage is that equivalent formulas give
rise to equal elements in the quotient.
This makes easier to prove equivalences as we go, since we can calculate with the associated
elements of the quotient and substitute one for another when the corresponding formulas are equivalent.

In our application, we instantiate \lstinline{ν} to a type \lstinline{Var} consisting
of positive natural numbers, corresponding to the indexing of variables in the DIMACS
format for SAT. It may be helpful to think of the elements of \lstinline{PropForm Var},
\lstinline{PropTerm Var}, and \lstinline{PropAssignment Var} as mathematical objects
rather than computational data. Our proof checker and model counters do not compute
with them; rather, we use them to specify what the checker and model counters are supposed to do.
The starting point for the proof checker is, instead, a CNF formula, which \emph{is}
presented as data.
Specifically, a literal is a nonzero integer, with the understanding that $-\ell$ represents
the negation of $\ell$. A clause is an array of literals, and a CNF formula is an array
of clauses:
\begin{lstlisting}
def ILit := { i : Int // i ≠ 0 }
abbrev IClause := Array ILit
abbrev ICnf := Array IClause
\end{lstlisting}
We define operations that translate these objects to the elements of \lstinline{PropForm Var}
that they represent. Lean provides us with helpful ``anonymous projection'' syntax that allows
us to write, for example, \lstinline{C.toPropForm} for the expression
\lstinline{Clause.toPropForm C}.
Similarly, we have a representation \lstinline{C.toPropTerm}
of \lstinline{C} as an equivalence class of propositional formulas, and we can speak of the
value of \lstinline{C} under any truth assignment.
Our code represents partial truth assignments as hashmaps:
\begin{lstlisting}
abbrev PartPropAssignment := HashMap Var Bool
\end{lstlisting}
We have written a procedure that reduces a clause \lstinline{C} with respect to a partial
assignment, and another procedure that uses a hashset to test whether a clause \lstinline{C} is a
tautology.
We have also verified that these low-level procedures do what they are supposed to do.

\subsection{The Proof Checker}
\label{appendix:proof:checker}

Our proof checker interprets the input clauses listed in a CRAT file as a CNF formula $\varphi$.
It then processes and checks each rule, throwing an exception if a precondition is not met.
If it terminates without raising an exception, it outputs a POG, $P$, with distinguished root
$\bf r$.
We have formally proved the claim that in this case the formula corresponding to node
$\bf r$ is a partitioned formula and is
logically equivalent to $\varphi$.
The POG $P$ can be saved to a file [TODO: implement this?] or passed on to one of the model counting
procedures described in the next section.

In our code, each element of a POG  is either a variable, a binary disjunction,
or an arbitrary conjunction:
\begin{lstlisting}
inductive PogElt where
  | var  : Var → PogElt
  | disj : Var → ILit → ILit → PogElt
  | conj : Var → Array ILit → PogElt
\end{lstlisting}
In the first case, the argument \lstinline{x} in the expression
\lstinline{var x} is the index
of the variable; in \lstinline{disj x left right} and \lstinline{conj x args}
it is the definition number in the CRAT file.
The CRAT generator described in Section~\ref{section:generating:crat}
declares new variables sequentially, so we currently make the simplifying assumption
that the $i$th element of the array has the node with index $i + 1$,
but it will not be hard to generalize this.
Notice that the arguments to \lstinline{disj} and \lstinline{conj} are literals,
corresponding to positive or negative instances of variables declared earlier in the POG.
So our POG data structure is as follows:
\begin{lstlisting}
structure Pog where
  elts : Array PogElt
  wf : ∀ i : Fin elts.size, elts[i].args_decreasing
  inv : ∀ i : Fin elts.size, i = elts[i].varNum.natPred
\end{lstlisting}
The invariant \lstinline{wf} says that the graph is well founded, in the
sense that if \lstinline{elts[i]} is a disjunction
or a conjunction, then the variables occurring in the arguments have smaller indices.
Note that \lstinline{Fin elts.size} is the data type consisting of natural numbers
less than the size of the array \lstinline{elts}; our code enforces statically
that the array accesses are within bounds.

Note that we have not yet said that the corresponding formula is partitioned.
For each POG \lstinline{P} and literal \lstinline{l}, we define
\lstinline{P.toPropForm l} to be the
propositional formula that arises from interpreting \lstinline{l} as a propositional
formula, unfolding all the defined conjunctions and disjunctions. We define
what it means for such a formula to be partitioned:
\begin{lstlisting}
def partitioned : PropForm ν → Prop
  | tr         => True
  | fls        => True
  | var _      => True
  | neg φ      => φ.partitioned
  | disj φ ψ   => φ.partitioned ∧ ψ.partitioned ∧
                    ∀ τ, ¬ (φ.eval τ ∧ ψ.eval τ)
  | conj φ ψ   => φ.partitioned ∧ ψ.partitioned ∧ (φ.vars ∩ ψ.vars = ∅)
  | impl _ _   => False
  | biImpl _ _ => False
\end{lstlisting}
In other words, we rule out formulas with implication and bi-implication, and otherwise
we follow the straightforward recursive definition.

With these definitions in place, we can explain the specification for our checker.
Our parser reads a CRAT file and returns an input formula \lstinline{cnf : ICnf} and a
CRAT proof {pf : Array CratStep}, that is, an array of single proof steps:
\begin{lstlisting}
inductive CratStep (α ν β : Type)
  | /-- Add asymmetric tautology. -/
    addAt (idx : α) (C : Array β) (upHints : Array α)
  | /-- Delete asymmetric tautology. -/
    delAt (idx : α) (upHints : Array α)
  | /-- Declare product operation. -/
    prod (idx : α) (x : ν) (ls : Array β)
  | /-- Declare sum operation. -/
    sum (idx : α) (x : ν) (l₁ l₂ : β) (upHints : Array α)
  | /-- Delete operation. -/
    delOp (x : ν)
\end{lstlisting}
The core checker takes those as input, as well as a specification of the final node \lstinline{r},
and either throws an exception, indicating that the proof is not well formed,
or returns a POG \lstinline{P}.
The specification asserts that, in the latter case, \lstinline{P.toPropForm r} is partitioned
and is equivalent to \lstinline{cnf.toPropForm}.

\subsection{The Model Counters}

We have formalized the central quantity (\ref{eqn:rep}) in the ring evaluation problem,
Definition~\ref{def:ring_evaluation}, as follows:
\begin{lstlisting}
def weightCount {R : Type} [CommRing R]
    (weight : ν → R) (φ : PropForm ν) (s : Finset ν) : R :=
  ∑ τ in models φ s, ∏ x in s, if τ x then weight x else 1 - weight x
\end{lstlisting}
Here \lstinline{R} is assumed to be a commutative ring. The curly brackets around \lstinline{R} indicate that
this argument is meant ot be left implicit, because it can inferred from the return type of
the weight function. Similarly, the square brackets indicate that the commutative ring structure
is a type class argument, generally inferred from the context.

More precisely, the quantity \lstinline{weightCount weight φ s} adds up the weights of the models
of \lstinline{φ} over the set of variables \lstinline{s}. We prove that as long as \lstinline{s}
contains the variables of \lstinline{φ}, adding extra variables does not change the sum, that is, \lstinline{vars φ ⊆ s} implies
\begin{lstlisting}
  weightCount weight φ s = weightCount weight φ (vars φ)
\end{lstlisting}
The counting scheme of Proposition~\ref{prop:ring:eval} for partitioned formulas is expressed as follows:
\begin{lstlisting}
 def countWeight (weight : ν → Rat) : PropForm ν → Rat
  | tr       => 1
  | fls      => 0
  | var x    => weight x
  | neg φ    => 1 - countWeight weight φ
  | disj φ ψ => countWeight weight φ + countWeight weight ψ
  | conj φ ψ => countWeight weight φ * countWeight weight ψ
\end{lstlisting}
We arbitrarily assign 0 to implications and bi-implications;
recall that, by definition, these do not appear in partitioned formulas.
Proposition~\ref{prop:ring:eval} is then formalized as follows:
\begin{lstlisting}
theorem countWeight_eq_weightCount (weight : ν → R) {φ : PropForm ν}
    (hdec : partitioned φ) :
  countWeight weight φ = weightCount weight φ (vars φ)
\end{lstlisting}

The functions just described are not meant to be computed, but,
rather, serve as our mathematical reference.
We have implemented an efficient function to calculate the weighted model count
for an arbitrary node of a Pog, and we have proved that it computes the function above on the
associated formula:
\begin{lstlisting}
theorem countWeight_eq_countWeight (pog : Pog) (weight : Var → Rat)
    (x : Var) :
  pog.countWeight weight x = (pog.toPropForm (.mkPos x)).countWeight weight
\end{lstlisting}
Here \lstinline{.mkPos x} denotes the positive literal for the variable \lstinline{x}.
Applying this to the output of our verified CRAT proof checker,
we obtain a proof that
our toolchain computes the correct weighted model count of the input CNF.

\lean{} does not yet have an efficient implementation of rational arithmetic, and
we have not yet implemented a verified version of $\drational$,
defined in Section~\ref{sect:experimental}.
We have therefore implemented a separate procedure that carries out an integer
calculation of the number of models.
The number of models that a propositional formula has depends on what we take the
variables to be. For example, there is only one assignment to the variables
$\{ x, y \}$ that makes $x \wedge y$ true, but there are two assignments to the
variables $\{ x, y, z \}$ that make $x \wedge y$ true.
It is not hard to show that for partitioned formulas \lstinline{PropForm ν} whose
variables are among a finite set \lstinline{s} of variables of cardinality \lstinline{numVars},
we can count the number of models of the formula on the variables in \lstinline{s}
recursively as follows:
\begin{lstlisting}
def countModels (nVars : Nat) : PropForm ν → Nat
  | tr       => 2^nVars
  | fls      => 0
  | var _    => 2^(nVars - 1)
  | neg φ    => 2^nVars - countModels nVars φ
  | disj φ ψ => countModels nVars φ + countModels nVars ψ
  | conj φ ψ => countModels nVars φ * countModels nVars ψ / 2^nVars
\end{lstlisting}
We have also implemented an efficient counting function for POGs and proved that
it computes that same quantity on the corresponding formulas.
Finally,
We have proved that the calculation above really does
return the number of models:
\begin{lstlisting}
theorem countModels_eq_card_models {φ : PropForm ν} {s : Finset ν}
    (hvars : vars φ ⊆ s) (hpar :  partitioned φ) :
  countModels (card s) φ = card (models φ s)
\end{lstlisting}
In particular, taking \lstinline{s} to be exactly the variables of \lstinline{φ},
we have that the number of models on its variables is \lstinline{countModels φ (card (vars φ))}.

\subsection{Implementation of the Proof Checker}

We have specified the behavior of the proof checker in Section~\ref{appendix:proof:checker};
in this section we sketch the implementation.
Establishing its correctness was, by far, the most challenging part of the formalization
process.
For clarity, we describe the algorithm with ordinary mathematical notation rather than
Lean snippets.

We begin by reading the input CNF formula $\varphi$ and adding its clauses to a clause
database $C$. We initialize that POG graph $G$ with the variables of $\varphi$,
and then iteratively process each line of the CRAT proof as follows:
\begin{itemize}
  \item (Summarize what each step does.)
\end{itemize}
We establish that all the following invariants hold at each step:
\begin{itemize}
  \item (Wojciech has a long list of invariants)
\end{itemize}
At the end \ldots [Explain how the invariants at the end imply the desired conclusion.]

\subsection{Verification specifics}
\label{subsection:verification:specifics}

In this subsection, we clarify what has been verified and what has been trusted.
Recall that our first step is to parse a CRAT file to read in the initial CNF formula
and the CRAT proof. We do not verify this step.
If one is only interested in obtaining a model count or a verified model count,
there is no strong need to verify the parsing of the proof,
because if the checker succeeds in constructing an equivalent POG,
it does not really matter whether the proof was the one the user intended.
But it would be nice to have reassurance that the CNF formula that serves
as input to our verified checker matches the one described in the file.
An easy and fairly standard way of addressing this is to print out the CNF formula
right after parsing it and then do a diff with the original input.
This involves trusting only the correctness of the print procedure and diff.
Similarly, if one wants to establish the correctness of the POG contained in the CRAT file,
one can print out the POG that is constructed by the checker and compare.

Lean's code extraction replaces calculations with natural numbers and integers with
efficient but unverified arbitrary precision versions.
Lean also uses an efficient implementation of arrays; within the
formal system, these are defined in terms of lists, but code extraction replaces them
with dynamic arrays and uses reference counting to allow destructive updates when it is safe
to do so [cite counting immutable beans paper].
Code extraction also provides efficient implementations of tail recursive functions and loops
[cite ``do unchained'' paper].
Finally, Lean's standard library implements hashmaps in terms of array.
Many of the basic properties of hashmaps have been formally verified, but not all.
In particular, we make use of a fold operation whose verification is not yet complete.
Thus our proofs depend on the assumption that the fold operation has the
expected properties.

In sum, in addition to trusting Lean's foundation and kernel checker,
we also have to trust that code extraction respects that foundation,
that the implementation of arrays and hashmaps satisfy their internal descriptions,
and that, after parsing, the computation has the correct input formula.
All of our specifications have been completely proved and verified relative to that.

\subsection{Evaluation}
\label{app:lean:evaluation}

\begin{figure}
\centering{%
\begin{tikzpicture}[scale = 0.70]
  \begin{axis}[mark options={scale=0.55},grid=both, grid style={black!10},
      legend style={at={(0.22,0.97)}},
      legend cell align={left},
                              x post scale=2.4, y post scale=2.2,
                              xmode=log,xmin=1000,xmax=1e7, 
                              xtick={1000, 10000, 100000, 1000000, 10000000}, xticklabels={$10^3$,$10^4$,$10^5$,$10^6$,$10^7$},
                              ymode=log, ymin=0.01, ymax=1900,
                              ytick={0.01, 0.1,1.0,10,100,1000}, yticklabels={0.01, 0.1, 1.0, 10, 100, {1,000}},
                              xlabel={CRAT Size (Total clauses)}, ylabel={Checking Time (seconds)},
            ]
    \input{data-formatted/time-check-lean}
    \input{data-formatted/time-check-regular}
    \legend{
      \scriptsize \textsf{\lean{} checker},
      \scriptsize \textsf{Regular checker},
    }

    \addplot[color=black,dashed] coordinates{(0.01,1500) (1e7,1500)};
    \addplot[mark=none, color=lightblue] coordinates{(0.01,1000) (1e7,1000)};
          \end{axis}
\end{tikzpicture}
} % centering
\caption{Times for Regular and \lean{} Checker, as Functions of Total Clauses}
\label{fig:lean}
\end{figure}


We ran the \lean{} checker on the 80 benchmark problems for which the
generated CRAT file had at most 10~million total clauses.  These are the
same benchmarks used in evaluating the optimizations (Section
\ref{app:experiment:optimize}).  Figure~\ref{fig:lean} shows the
comparative performance of the \lean{} checker versus our unverified prototype
checker on the Y axis, with the total number of clauses on the X axis.

As would be expected, the \lean{} checker consistently runs slower
than the prototype checker.  All of the checks completed and confirmed the outcome of the prototype checker.
The ratio of the runtime for \lean{} versus the runtime for
the prototype checker had a harmonic mean of $5.94$.

Importantly, however, it can be seen that both checkers show the same
overall performance trends.  The ratio of the two runtimes was less
than $8.0$ for all but three small benchmarks.  This seems like a
reasonable price to pay for a rigorous guarantee of correctness, and
we are confident that we can reduce this gap with more effort in
optimizing the checker.





\section{Detailed Experimental Results}
\label{app:experiments}

Our experiments are designed to evaluate the following:
\begin{itemize}
\item
  whether our toolchain can handle challenging benchmark problems,
\item
the effectiveness of some design choices and optimizations,
\item
the comparative performance of the prototype and verified checkers, and
\item
how the performance of our toolchain compares to that of the MICE model counter verifier.
\end{itemize}

\subsection{Experimental Setup}

As a test machine, we used a 2021~MacBook~Pro laptop having a 3.2~Ghz
Apple M1 processor and with 64~GB of physical memory.  We also used an
external 3~TB solid-state disk drive with an advertised read/write
speed of one gigabit per second.  Despite being a laptop, this
configuration is somewhat more powerful than the nodes typically
found in server clusters.  The fast access to storage is especially
important due to the very large files generated and processed by the
tools.  Our tools generated individual files with over 160 gigabytes
of data.

For benchmarks, we downloaded 100 files each from the 2022 standard and weighted model counting competitions:
\begin{center}
       \url{https://mccompetition.org/2022/mc_description.html}
\end{center}
We found that 20 of these were duplicates across the two tracks,
yielding a total of 180 unique benchmark problems, ranging in size
from 250 to 2,753,207 clauses.

Our default configuration for the proof generator used the structural
method for the forward implication proof, with the two optimizations
of literal grouping and lemmas.
Running with a time limit of 4000 seconds\footnote{All measured times
in this document are actual elapsed times, not CPU times.}, \dfour{}
was able to generate dec-DNNF representations for 124 of these, and it timed out on the
rest.  Our proof generator was able to convert all of the generated
dec-DNNF graphs into POGs and determine how many defining clauses they
would generate.  A POG operation with $k$ arguments requires $k+1$
defining clauses, and so the total number of defining clauses in POG $P$ equals
the number of nonterminal nodes plus the number of edges in the graph, corresponding to our definition of $|P|$.
The number of defining clauses ranged from 304 to 2,761,457,765 with a median of 848,784.

\subsection{Performance of the Proof Generator and Checker}

\begin{figure}
\centering{%
\begin{tikzpicture}[scale = 0.90]
  \begin{axis}[mark options={scale=0.55},grid=both, grid style={black!10}, ymode=log,
      legend style={at={(0.30,0.96)}},
      legend cell align={left},
                              x post scale=1.8, y post scale=2.4,
                              xmode=log,xmin=0.01,xmax=4000,
                              xtick={0.01, 0.1,1.0,10,100,1000,10000}, xticklabels={0.01, 0.1, 1.0, 10, 100, {1,000}, {10,000}},
                              ymin=0.01, ymax=29000,
                              ytick={0.01, 0.1,1.0,10,100,1000,10000,100000}, yticklabels={0.01, 0.1, 1.0, 10, 100, {1,000},{10,000},{100,000}},
                              xlabel={D4 runtime (seconds)}, ylabel={CRAT generation and checking runtime (seconds)},
%                              title={D4 Defining Clause Generation}
            ]
    \input{data-formatted/time-d4-verify}
    \input{data-formatted/time-d4-verify-onesided-only}
    \input{data-formatted/time-d4-failure}
    \legend{
      \scriptsize \textsf{Full validation},
      \scriptsize \textsf{One-sided validation},
      \scriptsize \textsf{Resource limit exceeded},
    }
    \addplot[mark=none, dashed] coordinates{(0.01,25000) (4000, 25000)};
    \addplot[mark=none, color=lightblue] coordinates{(0.01,20000) (4000,20000)};
    \addplot[mark=none, color=lightblue] coordinates{(0.1,0.01) (10000.0,1000.0)};
    \addplot[mark=none, color=lightblue] coordinates{(0.01,0.01) (10000.0,10000.0)};
    \addplot[mark=none, color=lightblue] coordinates{(0.01,0.1) (2000,20000)};
    \addplot[mark=none, color=lightblue] coordinates{(0.01,1.0) (200, 20000)};
    \addplot[mark=none, color=lightblue] coordinates{(0.01,10.0) (20, 20000)};
    \node[left] at (axis cs: 0.9,0.12) {$0{.}1\times$};
    \node[left] at (axis cs: 0.09,0.12) {$1\times$};
    \node[left] at (axis cs: 0.09,1.2) {$10\times$};
    \node[left] at (axis cs: 0.09,12.0) {$100\times$};
    \node[left] at (axis cs: 0.09,120.0) {$1000\times$};

          \end{axis}
\end{tikzpicture}
} % centering
\caption{Combined runtime for CRAT proof generation and checking  as function of D4 runtime.  Timeouts are shown as points on the dashed line.}
\label{fig:d4:crat}
\end{figure}

\begin{figure}
\centering{%
\begin{tikzpicture}[scale = 0.70]
  \begin{axis}[mark options={scale=0.55},grid=both, grid style={black!10}, ymode=log, ymin=100, ymax=1000000000, 
                              x post scale=1.8, y post scale=2.15,
                              xmode=log,xmin=100,xmax=1000000000, 
                              xtick={100,1000, 10000, 100000, 1000000, 10000000, 100000000, 1000000000}, xticklabels={$10^2$,$10^3$,$10^4$,$10^5$,$10^6$,$10^7$,$10^8$,$10^9$},
                              ytick={100,1000, 10000, 100000, 1000000, 10000000, 100000000, 1000000000}, yticklabels={$10^2$,$10^3$,$10^4$,$10^5$,$10^6$,$10^7$,$10^8$,$10^9$},
                              xlabel={Defining Clauses}, ylabel={Proof Clauses},
%                              title={D4 Defining Clause Generation}
            ]
    \addplot[mark=none, color=lightblue] coordinates{(100,100) (1e9,1e9)};
    \addplot[mark=none, color=lightblue] coordinates{(100,1000) (1e8,1e9)};
    \addplot[mark=none, color=lightblue] coordinates{(100,10000) (1e7,1e9)};
    \addplot[mark=none, color=lightblue] coordinates{(100,100000) (1e6,1e9)};
    \addplot[mark=none, color=lightblue] coordinates{(100,1000000) (1e5,1e9)};
    \input{data-formatted/defining+total}
    \node[left] at (axis cs: 900,1200) {$1\times$};
    \node[left] at (axis cs: 900,12000) {$10\times$};
    \node[left] at (axis cs: 900,120000) {$100\times$};
    \node[left] at (axis cs: 900,1200000) {$1{,}000\times$};
    \node[left] at (axis cs: 900,12000000) {$10{,}000\times$};

          \end{axis}
\end{tikzpicture}
} % centering
\caption{Total number of clauses in CRAT file as function of number of defining clauses}
\label{fig:defining:total}
\end{figure}


For each of the 124 dec-DNNF graphs generated by \dfour{}, we ran
our proof generator and prototype checker, limiting proof
generation to 10,000 seconds.  We ran the programs to generate and check
one-sided proofs for the graphs, again with a time limit of 10,000
seconds for proof generation.

Figure~\ref{fig:d4:cpog} summarizes our results in terms of the time
required by \dfour{} (X axis) versus the sum of the times for the
proof generator and checker (Y axis).
We were able to complete a full validation for 108 of the 124
benchmarks, with times ranging from fractions of a second to 13,144
seconds, with a median of 71.6 seconds.

Relative to the runtime for \dfour{}, two problems ran faster with the
generator and checker.  These were for problems having small numbers
of models (relative to the number of variables), and so most of the
time spent by both \dfour{} and our proof generator was in running a
CDCL solver to search the very sparse solution space.  \cadical{}
generally outperforms the miniSAT-based solver used by \dfour{}.  At
the other extreme, one problem that required only 19 seconds for
\dfour{} required 8657 seconds to generate a proof and 45 seconds to
check it.  This benchmark appears to stem from weak performance
by our implementation of BCP\@.  Overall the ratios
between the combined generation plus checking times versus the time
for \dfour{} had a harmonic mean of 5.5.

Of the 16 benchmarks that could not be fully validated, one had so
many defining clauses that it overflowed the 32-bit signed integers
our programs use for clause identifiers.  Another exited due to the
virtual memory limit imposed by the operating system, and the other 14
timed out.  Of the 15 that did not overflow the clause counter, we
were able to complete a one-sided verification for 9, but the other 6
timed out.  Overall, we were able to provide some level of
verification for 94\% of the benchmark problems.


Figure~\ref{fig:defining:total} compares the number of defining
clauses (X axis) with the total number of clauses (Y axis) for the 108
problems that were fully verified.  These totals include the defining
clauses for the POG, the additional defining clauses introduced to
support literal grouping and lemmas, and the clauses added by RUP
steps in the forward implication proof.  These totals ranged from 554
to 770,382,773, with a median of 1,719,245.  The ratios of the total
clauses versus defining clauses ranged from 1.54 to 6073, with a
harmonic mean of 3.13.  The high numbers were for problems having very
few models, and so the proof clauses must encode large proofs of
unsatisfiability.

\subsection{Monolithic Forward Implication Proofs}
\label{app:experiment:monolithic}


To assess the relative merits of the two approaches to forward
implication proof generation, we ran the proof generator in monolithic
mode on the 92 problems for which the combination of structural proof generation
and checking required at most 3000 seconds and then kept only those
measurements for which at least one of the two approaches had combined
times (generation plus checking) of at most 1000 seconds.  There were
a total of 83 such problems.  Figures~\ref{fig:monolithic:time} and
\ref{fig:monolithic:clauses} show comparisons between the two
approaches in terms of time and total clauses.

Examining Figure~\ref{fig:monolithic:time}, we can see that the
monolithic approach ran faster than the structural approach for 23 of
the problems, including a case where the generator and checker
required only one second, versus 71 seconds for the structural
approach.  On the other hand, the monolithic approach timed out for 13
of the cases and was slower for the other 37.  In general, we can see
the monolithic approach faring worse on larger problems.

Figure~\ref{fig:monolithic:clauses} shows the total number of clauses
for the two approaches for the 70 cases where both approaches completed within 1000 seconds.
As can be seen, the monolithic approach consistently produces
shorter proofs, perhaps due to the benefits of the proof trimming by
\dtrim{}.

\begin{figure}
\centering{%
\begin{tikzpicture}[scale = 0.70]
  \begin{axis}[mark options={scale=0.55},grid=both, grid style={black!10}, ymode=log,
      legend style={at={(0.22,0.98)}},
      legend cell align={left},
                              x post scale=1.8, y post scale=2.2,
                              xmode=log,xmin=0.01,xmax=1000,
                              xtick={0.01, 0.1,1.0,10,100,1000}, xticklabels={0.01, 0.1, 1.0, 10, 100,{1,000}},
                              ymin=0.01, ymax=1900,
                              ytick={0.01, 0.1,1.0,10,100,1000}, yticklabels={0.01, 0.1, 1.0, 10, 100, {1,000}},
                              xlabel={Structural generation + checking (seconds)}, ylabel={Monolithic generation + checking (seconds)},
            ]
    \input{data-formatted/time-structured+monolithic}
    \input{data-formatted/time-structured+monolithic_timeout}
    \addplot[mark=none, color=lightblue] coordinates{(1.0,0.01) (1000.0,10.0)};
    \addplot[mark=none, color=lightblue] coordinates{(0.1,0.01) (1000.0,100.0)};
    \addplot[mark=none, color=lightblue] coordinates{(0.01,0.01) (1000.0,1000.0)};
    \addplot[mark=none, color=lightblue] coordinates{(0.01,0.1) (100, 1000)};
    \addplot[mark=none, color=lightblue] coordinates{(0.01,1.0) (10, 1000)};
    \addplot[color=black,dashed] coordinates{(0.01,1500) (1500,1500)};
    \addplot[color=black,dashed] coordinates{(1500,0.01) (1500,1500)};
    \addplot[mark=none, color=lightblue] coordinates{(0.01,1000) (1000,1000)};
    \addplot[mark=none, color=lightblue] coordinates{(1000,0.01) (1000,1000)};
    \node[left] at (axis cs: 5.0,0.06) {$0{.}01\times$};
    \node[left] at (axis cs: 0.5,0.06) {$0{.}1\times$};
    \node[left] at (axis cs: 0.05,0.06) {$1\times$};
    \node[left] at (axis cs: 0.05,0.6) {$10\times$};
    \node[left] at (axis cs: 0.05,6.0) {$100\times$};

          \end{axis}
\end{tikzpicture}
} % centering
\caption{Monolithic versus Structural Validation: Time.  Timeouts are shown as points on the dashed lines.}
\label{fig:monolithic:time}
\end{figure}

\begin{figure}
\centering{%
\begin{tikzpicture}[scale = 0.90]
  \begin{axis}[mark options={scale=0.55},grid=both, grid style={black!10},
      legend style={at={(0.22,0.98)}},
      legend cell align={left},
                              x post scale=1.8, y post scale=2.2,
                              xmode=log,xmin=1e3,xmax=1e7,
                              ymode=log, ymin=1e3, ymax=1e7,
                              xlabel={Structural (total clauses)}, ylabel={Monolithic (total clauses)},
            ]
    \input{data-formatted/clauses-total-structured+monolithic}
%    \addplot[mark=none, color=black] coordinates{(0.01,1500) (1500,1500)};
%    \addplot[mark=none, color=black] coordinates{(1500,0.01) (1500,1500)};
%    \addplot[mark=none, color=lightblue] coordinates{(10.0,0.01) (10000.0,10.0)};
    \addplot[mark=none, color=lightblue] coordinates{(1e3,1e4) (1e7,1e8)};
    \addplot[mark=none, color=lightblue] coordinates{(1e3,1e3) (1e7,1e7)};
    \addplot[mark=none, color=lightblue] coordinates{(1e3,1e2) (1e7,1e6)};
%    \addplot[mark=none, color=lightblue] coordinates{(0.01,10.0) (1000,1000000)};
%    \node[left] at (axis cs: 50,0.06) {$0{.}001\times$};
    \node[left] at (axis cs: 3500,40000) {$10\times$};
    \node[left] at (axis cs: 3500,4000) {$1\times$};
    \node[left] at (axis cs: 35000,4000) {$0.1\times$};
%    \node[left] at (axis cs: 0.05,60.0) {$1000\times$};

          \end{axis}
\end{tikzpicture}
} % centering
\caption{Monolithic versus Structural Validation: Clauses}
\label{fig:monolithic:clauses}
\end{figure}



These experiments suggest that a hybrid of the two approaches might
yield the best results in terms of consistency, runtime, and proof
size.  It would begin the recursive descent of $\validate$ according
to a structural approach, but monolithically generate
the proof once it encounters a sufficiently small subgraph.

\subsection{Impact of Optimizations}
\label{app:experiment:optimize}

\begin{figure}
\begin{minipage}{0.49\textwidth}
\centering{%
\begin{tikzpicture}[scale = 0.80]
  \begin{axis}[mark options={scale=0.55},grid=both, grid style={black!10}, ymode=log, ymin=1000, ymax=3e8, 
      legend style={at={(0.50,0.98)}},
      legend cell align={left},
                              x post scale=1.1, y post scale=1.8,
                              xmode=log,xmin=1000,xmax=1e7, 
                              xtick={1000, 10000, 100000, 1000000, 10000000}, xticklabels={$10^3$,$10^4$,$10^5$,$10^6$,$10^7$},
                              ytick={1000, 10000, 100000, 1000000, 10000000, 100000000, 1e9}, yticklabels={$10^3$,$10^4$,$10^5$,$10^6$,$10^7$,$10^8$, $10^9$},
                              xlabel={Optimized Proof Clauses},
%                              ylabel={Proof Clauses},
            ]
    \input{data-formatted/clauses-lemma-nolemma}
    \input{data-formatted/clauses-nosplit-split}
    \legend{
      \scriptsize \textsf{No lemmas},
      \scriptsize \textsf{No literal grouping},
    }
    \addplot[mark=none, color=lightblue] coordinates{(1000,1000) (1e8,1e8)};
    \addplot[mark=none, color=lightblue] coordinates{(1000,10000) (1e8,1e9)};
    \addplot[mark=none, color=lightblue] coordinates{(1000,100000) (1e7,1e9)};

    \node[left] at (axis cs: 5000,7000) {$1\times$};
    \node[left] at (axis cs: 5000,70000) {$10\times$};
    \node[left] at (axis cs: 5000,700000) {$100\times$};


          \end{axis}
\end{tikzpicture}
} % centering
\end{minipage}
\begin{minipage}{0.49\textwidth}
\centering{%
\begin{tikzpicture}[scale = 0.80]
  \begin{axis}[mark options={scale=0.55},grid=both, grid style={black!10}, ymode=log, ymin=1000, ymax=3e8, 
      legend style={at={(0.50,0.98)}},
      legend cell align={left},
                              x post scale=1.1, y post scale=1.8,
                              xmode=log,xmin=1000,xmax=1e7, 
                              xtick={1000, 10000, 100000, 1000000, 10000000}, xticklabels={$10^3$,$10^4$,$10^5$,$10^6$,$10^7$},
                              ytick={1000, 10000, 100000, 1000000, 10000000, 100000000, 1e9}, yticklabels={$10^3$,$10^4$,$10^5$,$10^6$,$10^7$,$10^8$, $10^9$},
                              xlabel={Optimized Proof Clauses},
%                              ylabel={Unoptimized Proof Clauses},
            ]
    \input{data-formatted/clauses-optimized-unoptimized}    
    \legend{
      \scriptsize \textsf{Neither optimization},
    }
    \addplot[mark=none, color=lightblue] coordinates{(1000,1000) (1e8,1e8)};
    \addplot[mark=none, color=lightblue] coordinates{(1000,10000) (1e8,1e9)};
    \addplot[mark=none, color=lightblue] coordinates{(1000,100000) (1e7,1e9)};

    \node[left] at (axis cs: 5000,7000) {$1\times$};
    \node[left] at (axis cs: 5000,70000) {$10\times$};
    \node[left] at (axis cs: 5000,700000) {$100\times$};
  \end{axis}
\end{tikzpicture}
} % centering
\end{minipage}
\caption{Proof clauses when one (left) or both (right) optimizations are disabled, versus when both optimizations are enabled}
\label{fig:optimized:lessoptimized}
\end{figure}


To assess the impact of the two optimizations: literal grouping and
lemmas, we ran the proof generator on the 80 problems for which the
combination of the two optimizations yielded proofs totaling at most
10~million clauses.  Compared to the problems used to assess monolithic
proof generation (Section~\ref{app:experiment:monolithic}), this set included one that had a short proof but
long runtime and excluded four that had short runtimes but long
proofs.  We then ran the proof generator on all three combinations of
the optimizations being partially or totally disabled.
Figure~\ref{fig:optimized:lessoptimized} shows how disabling these optimizations affected the
total number of clauses in the proofs, where the X axis indicates the
total number with both optimizations enabled, while the Y axis
indicates the number with one (left) or both (right) optimizations
disabled.

As the figures illustrated, each optimization on its own can shorten
the proofs, sometimes substantially, and the two combine to have even
greater effect.  Of the 80 benchmark problems evaluated, and comparing
with both optimizations disabled, 47 had the number of proof clauses
reduced by a factor of 2 by using lemmas, 14 by using literal merging,
and 60 by enabling both optimizations.  In the extreme cases, a lack
of lemmas caused one proof to grow by a factor of 52.5, while a lack
of literal grouping caused another proof to grow by a factor of 39.6.

The time performance of these optimizations is less dramatic, but
still significant.  Lemmas sometimes made proof generation slower, but
in one case ran $38\times$ faster.  Similarly, literal grouping could
take longer, but in one case ran $3.9\times$ faster.

Overall, it is clear that these two optimizations are worthwhile, and
sometimes critical, to success for some benchmarks.

\subsection{Certifying Preprocessed Formulas}

\begin{figure}
\centering{%
\begin{tikzpicture}[scale = 0.70]
  \begin{axis}[mark options={scale=0.55},grid=both, grid style={black!10}, ymode=log,
      legend style={at={(0.22,0.98)}},
      legend cell align={left},
                              x post scale=1.8, y post scale=2.2,
                              xmode=log,xmin=0.01,xmax=1000,
                              xtick={0.01, 0.1,1.0,10,100,1000}, xticklabels={0.01, 0.1, 1.0, 10, 100,{1,000}},
                              ymin=0.01, ymax=1900,
                              ytick={0.01, 0.1,1.0,10,100,1000}, yticklabels={0.01, 0.1, 1.0, 10, 100, {1,000}},
                              xlabel={Total time without preprocessing (seconds)}, ylabel={Total time with preprocessing (seconds)},
            ]
    \input{data-formatted/time-monolithic+preprocess}
%    \input{data-formatted/timeout-monolithic+time-preprocess}
%    \addplot[mark=none, color=lightblue] coordinates{(1.0,0.01) (1000.0,10.0)};
    \addplot[mark=none, color=lightblue] coordinates{(0.1,0.01) (1000.0,100.0)};
    \addplot[mark=none, color=lightblue] coordinates{(0.01,0.01) (1000.0,1000.0)};
    \addplot[mark=none, color=lightblue] coordinates{(0.01,0.1) (100, 1000)};
%    \addplot[mark=none, color=lightblue] coordinates{(0.01,1.0) (10, 1000)};
    \addplot[color=black,dashed] coordinates{(0.01,1500) (1500,1500)};
    \addplot[color=black,dashed] coordinates{(1500,0.01) (1500,1500)};
    \addplot[mark=none, color=lightblue] coordinates{(0.01,1000) (1000,1000)};
    \addplot[mark=none, color=lightblue] coordinates{(1000,0.01) (1000,1000)};
%    \node[left] at (axis cs: 5.0,0.06) {$0{.}01\times$};
    \node[left] at (axis cs: 0.5,0.06) {$0{.}1\times$};
    \node[left] at (axis cs: 0.05,0.06) {$1\times$};
    \node[left] at (axis cs: 0.05,0.6) {$10\times$};
%    \node[left] at (axis cs: 0.05,6.0) {$100\times$};

          \end{axis}
\end{tikzpicture}
} % centering
\caption{Runtime impact of preprocessing.  All use monolithic generation.  Times include running {\sc D4}}
\label{fig:preprocess:time}
\end{figure}

\begin{figure}
\centering{%
\begin{tikzpicture}[scale = 0.70]
  \begin{axis}[mark options={scale=0.55},grid=both, grid style={black!10},
      legend style={at={(0.22,0.98)}},
      legend cell align={left},
                              x post scale=1.8, y post scale=2.2,
                              xmode=log,xmin=1e3,xmax=1e7,
                              ymode=log, ymin=1e3, ymax=1e7,
                              xlabel={Total clauses without preprocessing}, ylabel={Total clauses with preprocessing},
            ]
    \input{data-formatted/clauses-total-monolithic+preprocess}
%    \addplot[mark=none, color=black] coordinates{(0.01,1500) (1500,1500)};
%    \addplot[mark=none, color=black] coordinates{(1500,0.01) (1500,1500)};
%    \addplot[mark=none, color=lightblue] coordinates{(10.0,0.01) (10000.0,10.0)};
    \addplot[mark=none, color=lightblue] coordinates{(1e3,1e4) (1e7,1e8)};
    \addplot[mark=none, color=lightblue] coordinates{(1e3,1e3) (1e7,1e7)};
    \addplot[mark=none, color=lightblue] coordinates{(1e3,1e2) (1e7,1e6)};
%    \addplot[mark=none, color=lightblue] coordinates{(0.01,10.0) (1000,1000000)};
%    \node[left] at (axis cs: 50,0.06) {$0{.}001\times$};
    \node[left] at (axis cs: 3500,40000) {$10\times$};
    \node[left] at (axis cs: 3500,4000) {$1\times$};
    \node[left] at (axis cs: 35000,4000) {$0.1\times$};
%    \node[left] at (axis cs: 0.05,60.0) {$1000\times$};

          \end{axis}
\end{tikzpicture}
} % centering
\caption{Impact of preprocessing on total clauses.  All use monolithic generation}
\label{fig:preprocess:clauses}
\end{figure}


The \dfour{} knowledge compiler can optionally perform three different
forms of preprocessing on the input formula.  These transform the
formula while preserving logical equivalence~\cite{lagniez:aaai:2014}.
Proving that the combination of preprocessing plus knowledge
compilation preserved logical equivalence would be a valuable
capability.  However, proof generators that operate based on the
structure of the input clauses, including our structural approach,
will generally fail in this case, since the compilation was not based
on the original input clauses.

One strength of monolithic proof generation is that it makes no
assumptions about how the compiled representation.  A complete
proof-generating SAT solver could, in principle, always generate the
forward implication proof, as long as forward implication holds.  To
test this hypothesis, we ran our proof framework with preprocessing
enabled on the same 83 benchmarks as in
Section~\ref{app:experiment:monolithic}.  We measured the time for two
entire toolchains: one with compilation, proof generation, and
checking, and the other starting with preprocessing and then having
the same steps as the other.  The results are summarized in
Figures~\ref{fig:preprocess:time} and \ref{fig:preprocess:clauses}.
Setting a time limit of 1000 seconds, 70 of the formulas completed
without preprocessing, while 69 did so with preprocessing.  The one
that timed out with preprocessing did so due to the excessive time
required for preprocessing.  Of the 69, 53 ran faster with
preprocessing, while 15 ran slower.  The sizes of the generated 
CPOG proofs were more variable: 31 had fewer clauses with preprocessing, 21 had more, and 17 were identical.


\subsection{Evaluation of the Verified Checker}
\label{app:lean:evaluation}

\begin{figure}
\centering{%
\begin{tikzpicture}[scale = 0.70]
  \begin{axis}[mark options={scale=0.55},grid=both, grid style={black!10},
      legend style={at={(0.22,0.97)}},
      legend cell align={left},
                              x post scale=2.4, y post scale=2.2,
                              xmode=log,xmin=1000,xmax=1e7, 
                              xtick={1000, 10000, 100000, 1000000, 10000000}, xticklabels={$10^3$,$10^4$,$10^5$,$10^6$,$10^7$},
                              ymode=log, ymin=0.01, ymax=1900,
                              ytick={0.01, 0.1,1.0,10,100,1000}, yticklabels={0.01, 0.1, 1.0, 10, 100, {1,000}},
                              xlabel={CRAT Size (Total clauses)}, ylabel={Checking Time (seconds)},
            ]
    \input{data-formatted/time-check-lean}
    \input{data-formatted/time-check-regular}
    \legend{
      \scriptsize \textsf{\lean{} checker},
      \scriptsize \textsf{Regular checker},
    }

    \addplot[color=black,dashed] coordinates{(0.01,1500) (1e7,1500)};
    \addplot[mark=none, color=lightblue] coordinates{(0.01,1000) (1e7,1000)};
          \end{axis}
\end{tikzpicture}
} % centering
\caption{Times for Regular and \lean{} Checker, as Functions of Total Clauses}
\label{fig:lean}
\end{figure}


We ran the \lean{} checker on the 80 benchmark problems for which the
generated CPOG file had at most 10~million total clauses.  These are the
same benchmarks used in evaluating the optimizations (Section
\ref{app:experiment:optimize}).
All of the checks completed and confirmed the outcome of the prototype checker.

Figure~\ref{fig:lean} shows the
performance of both the \lean{} checker and the prototype
checker on the Y axis, as functions of the total number of clauses on the X axis.
As would be expected, the \lean{} checker consistently runs slower
than the prototype checker.
The ratio of the runtime for \lean{} versus the runtime for
the prototype checker had a harmonic mean of $5.94$.

Importantly, however, it can be seen that both checkers show the same
overall performance trends.  The ratio of the two runtimes was less
than $8.0$ for all but three small benchmarks.  This seems like a
reasonable price to pay for a rigorous guarantee of correctness, and
we are confident that we can reduce this gap with more effort in
optimizing the verified checker.


\section{Comparison with other Certification Frameworks}
\label{app:comparison}

This section provides more insights into how our proof framework (POG)
compares with other work in certified knowledge compilation and model counting.


\subsection{Comparison with \cdfour{} Framework}

\begin{figure}
\centering{%
\begin{tikzpicture}[scale = 0.70]
  \begin{axis}[mark options={scale=0.55},grid=both, grid style={black!10}, ymode=log,
      legend style={at={(0.22,0.98)}},
      legend cell align={left},
                              x post scale=1.8, y post scale=2.2,
                              xmode=log,xmin=0.01,xmax=1900,
                              xtick={0.01, 0.1,1.0,10,100,1000}, xticklabels={0.01, 0.1, 1.0, 10, 100,{1,000}},
                              ymin=0.01, ymax=1900,
                              ytick={0.01, 0.1,1.0,10,100,1000}, yticklabels={0.01, 0.1, 1.0, 10, 100, {1,000}},
                              xlabel={CPOG Toolchain (seconds)}, ylabel={CD4 Toolchain (seconds)},
            ]
    \input{data-formatted/time-toolchain-compare-cd4-timeout}
    \input{data-formatted/time-toolchain-compare-pog-fail}
    \input{data-formatted/time-toolchain-compare}
    \addplot[mark=none, color=lightblue] coordinates{(1.0,0.01) (1000.0,10.0)};
    \addplot[mark=none, color=lightblue] coordinates{(0.1,0.01) (1000.0,100.0)};
    \addplot[mark=none, color=lightblue] coordinates{(0.01,0.01) (1000.0,1000.0)};
    \addplot[mark=none, color=lightblue] coordinates{(0.01,0.1) (100, 1000)};
%%    \addplot[mark=none, color=lightblue] coordinates{(0.01,1.0) (10, 1000)};
    \addplot[color=black,dashed] coordinates{(0.01,1500) (1500,1500)};
    \addplot[color=black,dashed] coordinates{(1500,0.01) (1500,1500)};
    \addplot[mark=none, color=lightblue] coordinates{(0.01,1000) (1000,1000)};
    \addplot[mark=none, color=lightblue] coordinates{(1000,0.01) (1000,1000)};
    \node[left] at (axis cs: 5.0,0.06) {$0{.}01\times$};
    \node[left] at (axis cs: 0.5,0.06) {$0{.}1\times$};
    \node[left] at (axis cs: 0.05,0.06) {$1\times$};
    \node[left] at (axis cs: 0.05,0.6) {$10\times$};

          \end{axis}
\end{tikzpicture}
} % centering
\caption{Times for CD4 Toolchain versus CPOG Toolchain.  Times include knowledge compilation, proof generation, and checking}
\label{fig:cd4:cpog}
\end{figure}


Figure~\ref{fig:cd4:cpog} shows the results of a set of experiments
comparing the performance of our toolchain (CPOG) versus one based on
\cdfour{}.  It shows the timings for the 110 benchmarks for which
either the CPOG toolchain or the \cdfour{} toolchain completes in less
than 1000 seconds.  Both toolchains include compilation, in addition
to proof generation and checking.  Of these, 80 completed for both
toolchains, 28 timed out or failed with the CPOG toolchain, and 2
timed out for the \cdfour{} toolchain.  The data clearly show that the
\cdfour{} toolchain can handle larger benchmarks and generally runs
faster.  Only 7 of the 80 that completed with both toolchains ran
faster with the CPOG toolchain.  For these 80, the toolchain based on
\cdfour{} ran faster by a harmonic mean factor of $19.8\times$.

On the other hand, the \cdfour{} proof generation and checking is
tightly connected to the operation of \dfour{}, and every change to
the program could require changes to the proof steps and possibly
changes to the proof framework.  For example, we have found that
\cdfour{} does not work properly when the compiler makes use of its
internal preprocessor.


\subsection{Comparison with the MICE Framework}

To compare the performance of our toolchain versus that of
MICE~\cite{fichte:sat:2022}, we consider how the combination of proof
generation, proof checking, and ring evaluation in the CPOG toolchain compares to the time to
run their proof generator \progname{nnf2trace} and checker
\progname{sharptrace}.  We add the time for ring evaluation on our
side, since their checker provides a count, although only an
unweighted one.  We used the prototype versions of the checker and ring evaluator.
We ran their tools on the 92 benchmark problems for
which our toolchain requires at most 3000 seconds and retained
the 84 data points for which at least one of the toolchains completes in under 1000 seconds.

We can summarize the results as:
\begin{itemize}
\item 84 total data points
\item 76 were completed by both programs in under 1000 seconds
\item Of those 21 were faster with MICE, 55 with CPOG
\item 7 completed with CPOG under 1000 seconds but exceeded that threshold for MICE
\item 1 completed with MICE under 1000 seconds but exceeded that threshold for CPOG
\end{itemize}

\begin{figure}
\centering{%
\begin{tikzpicture}[scale = 0.70]
  \begin{axis}[mark options={scale=0.55},grid=both, grid style={black!10},
      legend style={at={(0.22,0.97)}},
      legend cell align={left},
                              x post scale=1.8, y post scale=2.2,
                              xmode=log,xmin=0.01,xmax=1900,
                              xtick={0.01, 0.1,1.0,10,100,1000}, xticklabels={0.01, 0.1, 1.0, 10, 100,{1,000}},
                              ymode=log, ymin=0.01, ymax=1900,
                              ytick={0.01, 0.1,1.0,10,100,1000}, yticklabels={0.01, 0.1, 1.0, 10, 100, {1,000}},
                              xlabel={CPOG total time (seconds)}, ylabel={MICE total time (seconds)},
            ]
    \input{data-formatted/time-mice-compare}
%    \input{data-formatted/time-mice-error}
%%    \legend{
%%      \scriptsize \textsf{Correct or timeout},
%%      \scriptsize \textsf{MICE error},
%%    }
    \input{data-formatted/time-mice-timeout}

    \addplot[color=black,dashed] coordinates{(0.01,1500) (1500,1500)};
    \addplot[color=black,dashed] coordinates{(1500,0.01) (1500,1500)};
    \addplot[mark=none, color=lightblue] coordinates{(0.01,1000) (1000,1000)};
    \addplot[mark=none, color=lightblue] coordinates{(1000,0.01) (1000,1000)};

%    \addplot[mark=none, color=black] coordinates{(0.01,1500) (1500,1500)};
%    \addplot[mark=none, color=black] coordinates{(1500,0.01) (1500,1500)};
    \addplot[mark=none, color=lightblue] coordinates{(10.0,0.01) (1000.0,1.0)};
    \addplot[mark=none, color=lightblue] coordinates{(1.0,0.01) (1000.0,10.0)};
    \addplot[mark=none, color=lightblue] coordinates{(0.1,0.01) (1000.0,100.0)};
    \addplot[mark=none, color=lightblue] coordinates{(0.01,0.01) (1000.0,1000.0)};
    \addplot[mark=none, color=lightblue] coordinates{(0.01,0.1) (100,1000)};
    \addplot[mark=none, color=lightblue] coordinates{(0.01,1.0) (10,1000)};
    \addplot[mark=none, color=lightblue] coordinates{(0.01,10.0) (1,1000)};
    \node[left] at (axis cs: 50,0.06) {$0{.}001\times$};
    \node[left] at (axis cs: 5.0,0.06) {$0{.}01\times$};
    \node[left] at (axis cs: 0.5,0.06) {$0{.}1\times$};
    \node[left] at (axis cs: 0.05,0.06) {$1\times$};
    \node[left] at (axis cs: 0.05,0.6) {$10\times$};
    \node[left] at (axis cs: 0.05,6.0) {$100\times$};
    \node[left] at (axis cs: 0.05,60.0) {$1000\times$};

          \end{axis}
\end{tikzpicture}
} % centering
\caption{Running Time for MICE versus CPOG proof chains.  Times include proof generation, checking, and counting.  Timeouts are shown as points on the dashed lines.}
\label{fig:mice}
\end{figure}


The detailed results are shown in
Figure~\ref{fig:mice}.  As can be seen, the comparative runtimes are
highly variable, reflecting the fact that the two toolchains differ
fundamentally in their objectives and their approaches.

One shortcoming of \progname{nnf2trace} can be seen in
the near-vertical series of points in the upper-lefthand corner.
These correlate with a similar set of points along the left in
Figure~\ref{fig:optimized:lessoptimized}.  Like the naive
implementation of $\validate$, their program recursively traverses the
graph representation of a formula, effectively expanding it into a
tree.  They lack an optimization analogous to lemmas.

We have not tried the MICE toolchain on larger benchmarks, but we
suspect it would encounter significant scaling limitations.

\bibliography{references}

\end{document}
