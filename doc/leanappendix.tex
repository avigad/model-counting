\section{More information about the formally verified checker}
\label{appendix:lean}

We have implemented the rightmost two components of Figure~\ref{fig:chain}, namely the
proof checker and both model and weighted model counters, in the \lean{} programming language
and proof assistant [cite]. \lean{} implements a logical foundation in which expressions
have a computational interpretation, and just as with other proof assistants like Isabelle and Coq,
a function defined within the formal system can be compiled to efficient code.
At the same time, we can state and prove claims about the function within the system, thereby
verifying that the functions compute the intended results.

Verifying our checker and the model counters in this way serves at least two purposes.
When we run them on particular data sets, they provide high confidence
that the particular results we obtain are correct.
More broadly, however, our success also validates our general approach,
showing, in particular, that the CRAT proof format is sound.
Indeed, our formalization helped us discover a bug in our implementation
of the informal proof checker: in the ``\emph{s}'' step described at the end of
Subsection~\ref{subsection:syntax}, it is sound (and necessary) to allow defining
clauses for the new operations in the proof of $\obar{\lit}_1 \lor \obar{\lit}_2$,
but it is generally not sound to allow the input clauses or other added clauses.
[NOTE: let's add this specification to the end of Section 6.1.]
Our verified implementation of the checker tells us that the algorithm is sound and
provides a reference for other implementations.

In this section, we describe the implementation in Lean and the specifications we have proved.
\lean{}'s logical foundation describes a functional programming language with
inductive data types,
but our procedures also make use of calculations with natural numbers, integers, and rationals,
and data structures for arrays and hashmaps.
Lean's code extraction uses efficient but unverified versions of these,
and so in Subsection~\ref{subsection:verification:specifics} we clarify
what exactly we have verified and what we currently trust.

\subsection{Logical objects and operations}

Our specifications are built on a generic library for the syntax and semantics of
propositional logic. The data type of propositional formulas with variables indexed by
a datatype \lstinline{ν} is defined inductively as follows:
\begin{lstlisting}
inductive PropForm (ν : Type)
  | var (x : ν)
  | tr
  | fls
  | neg (φ : PropForm ν)
  | conj (φ₁ φ₂ : PropForm ν)
  | disj (φ₁ φ₂ : PropForm ν)
  | impl (φ₁ φ₂ : PropForm ν)
  | biImpl (φ₁ φ₂ : PropForm ν)
\end{lstlisting}
The last two constructors denote implication and bi-implication, which are not used in this
project. The usual syntactic and semantic notions are defined in the expected ways.
It is convenient to take a truth assignment \lstinline{PropAssignment ν} to a function that
assigns a boolean value to every assignment, and later restrict attention to subsets of the
variables. In Lean, we introduce the notation \lstinline{τ ⊨ φ} for the statement that
\lstinline{τ} satisfies \lstinline{φ}. We write \lstinline{entails φ₁ φ₂} to say that
\lstinline{φ₁} and \lstinline{φ₂} and \lstinline{equivalent φ₁ φ₂}.
For the purpose of model counting, if \lstinline{s} is any finite set of variables,
we define \lstinline{models φ s} to be the finite set of truth assignments \lstinline{τ}
satisfying \lstinline{φ} that assign arbitrary values on \lstinline{s} and
\lstinline{false} outside \lstinline{s}.

In logic, the set of propositional formulas modulo equivalence is sometimes called the
\emph{Lindenbaum--Tarski algebra}. We have found it convenient to define this quotient,
which we denote \lstinline{PropTerm ν}, and lift the boolean operations and the entailment
relation to this new type. The advantage is that equivalent formulas give
rise to equal elements in the quotient.
This makes easier to prove equivalences as we go, since we can calculate with the associated
elements of the quotient and substitute one for another when the corresponding formulas are equivalent.

In our application, we instantiate \lstinline{ν} to a type \lstinline{Var} consisting
of positive natural numbers, corresponding to the indexing of variables in the DIMACS
format for SAT. It may be helpful to think of the elements of \lstinline{PropForm Var},
\lstinline{PropTerm Var}, and \lstinline{PropAssignment Var} as mathematical objects
rather than computational data. Our proof checker and model counters do not compute
with them; rather, we use them to specify what the checker and model counters are supposed to do.
The starting point for the proof checker is, instead, a CNF formula, which \emph{is}
presented as data.
Specifically, a literal is a nonzero integer, with the understanding that $-\ell$ represents
the negation of $\ell$. A clause is an array of literals, and a CNF formula is an array
of clauses:
\begin{lstlisting}
def ILit := { i : Int // i ≠ 0 }
abbrev IClause := Array ILit
abbrev ICnf := Array IClause
\end{lstlisting}
We define operations that translate these objects to the elements of \lstinline{PropForm Var}
that they represent. Lean provides us with helpful ``anonymous projection'' syntax that allows
us to write, for example, \lstinline{C.toPropForm} for the expression
\lstinline{Clause.toPropForm C}.
Similarly, we have a representation \lstinline{C.toPropTerm}
of \lstinline{C} as an equivalence class of propositional formulas, and we can speak of the
value of \lstinline{C} under any truth assignment.
Our code represents partial truth assignments as hashmaps:
\begin{lstlisting}
abbrev PartPropAssignment := HashMap Var Bool
\end{lstlisting}
We have written a procedure that reduces a clause \lstinline{C} with respect to a partial
assignment, and another procedure that uses a hashset to test whether a clause \lstinline{C} is a
tautology.
We have also verified that these low-level procedures do what they are supposed to do.

\subsection{The Proof Checker}
\label{appendix:proof:checker}

Our proof checker interprets the input clauses listed in a CRAT file as a CNF formula $\varphi$.
It then processes and checks each rule, throwing an exception if a precondition is not met.
If it terminates without raising an exception, it outputs a POG, $P$, with distinguished root
$\bf r$.
We have formally proved the claim that in this case the formula corresponding to node
$\bf r$ is a partitioned formula and is
logically equivalent to $\varphi$.
The POG $P$ can be saved to a file [TODO: implement this?] or passed on to one of the model counting
procedures described in the next section.

In our code, each element of a POG  is either a variable, a binary disjunction,
or an arbitrary conjunction:
\begin{lstlisting}
inductive PogElt where
  | var  : Var → PogElt
  | disj : Var → ILit → ILit → PogElt
  | conj : Var → Array ILit → PogElt
\end{lstlisting}
In the first case, the argument \lstinline{x} in the expression
\lstinline{var x} is the index
of the variable; in \lstinline{disj x left right} and \lstinline{conj x args}
it is the definition number in the CRAT file.
The CRAT generator described in Section~\ref{section:generating:crat}
declares new variables sequentially, so we currently make the simplifying assumption
that the $i$th element of the array has the node with index $i + 1$,
but it will not be hard to generalize this.
Notice that the arguments to \lstinline{disj} and \lstinline{conj} are literals,
corresponding to positive or negative instances of variables declared earlier in the POG.
So our POG data structure is as follows:
\begin{lstlisting}
structure Pog where
  elts : Array PogElt
  wf : ∀ i : Fin elts.size, elts[i].args_decreasing
  inv : ∀ i : Fin elts.size, i = elts[i].varNum.natPred
\end{lstlisting}
The invariant \lstinline{wf} says that the graph is well founded, in the
sense that if \lstinline{elts[i]} is a disjunction
or a conjunction, then the variables occurring in the arguments have smaller indices.
Note that \lstinline{Fin elts.size} is the data type consisting of natural numbers
less than the size of the array \lstinline{elts}; our code enforces statically
that the array accesses are within bounds.

Note that we have not yet said that the corresponding formula is partitioned.
For each POG \lstinline{P} and literal \lstinline{l}, we define
\lstinline{P.toPropForm l} to be the
propositional formula that arises from interpreting \lstinline{l} as a propositional
formula, unfolding all the defined conjunctions and disjunctions. We define
what it means for such a formula to be partitioned:
\begin{lstlisting}
def partitioned : PropForm ν → Prop
  | tr         => True
  | fls        => True
  | var _      => True
  | neg φ      => φ.partitioned
  | disj φ ψ   => φ.partitioned ∧ ψ.partitioned ∧
                    ∀ τ, ¬ (φ.eval τ ∧ ψ.eval τ)
  | conj φ ψ   => φ.partitioned ∧ ψ.partitioned ∧ (φ.vars ∩ ψ.vars = ∅)
  | impl _ _   => False
  | biImpl _ _ => False
\end{lstlisting}
In other words, we rule out formulas with implication and bi-implication, and otherwise
we follow the straightforward recursive definition.

With these definitions in place, we can explain the specification for our checker.
Our parser reads a CRAT file and returns an input formula \lstinline{cnf : ICnf} and a
CRAT proof {pf : Array CratStep}, that is, an array of single proof steps:
\begin{lstlisting}
inductive CratStep (α ν β : Type)
  | /-- Add asymmetric tautology. -/
    addAt (idx : α) (C : Array β) (upHints : Array α)
  | /-- Delete asymmetric tautology. -/
    delAt (idx : α) (upHints : Array α)
  | /-- Declare product operation. -/
    prod (idx : α) (x : ν) (ls : Array β)
  | /-- Declare sum operation. -/
    sum (idx : α) (x : ν) (l₁ l₂ : β) (upHints : Array α)
  | /-- Delete operation. -/
    delOp (x : ν)
\end{lstlisting}
The core checker takes those as input, as well as a specification of the final node \lstinline{r},
and either throws an exception, indicating that the proof is not well formed,
or returns a POG \lstinline{P}.
The specification asserts that, in the latter case, \lstinline{P.toPropForm r} is partitioned
and is equivalent to \lstinline{cnf.toPropForm}.

\subsection{The Model Counters}

We have formalized the central quantity (\ref{eqn:rep}) in the ring evaluation problem,
Definition~\ref{def:ring_evaluation}, as follows:
\begin{lstlisting}
def weightCount {R : Type} [CommRing R]
    (weight : ν → R) (φ : PropForm ν) (s : Finset ν) : R :=
  ∑ τ in models φ s, ∏ x in s, if τ x then weight x else 1 - weight x
\end{lstlisting}
Here \lstinline{R} is assumed to be a commutative ring. The curly brackets around \lstinline{R} indicate that
this argument is meant ot be left implicit, because it can inferred from the return type of
the weight function. Similarly, the square brackets indicate that the commutative ring structure
is a type class argument, generally inferred from the context.

More precisely, the quantity \lstinline{weightCount weight φ s} adds up the weights of the models
of \lstinline{φ} over the set of variables \lstinline{s}. We prove that as long as \lstinline{s}
contains the variables of \lstinline{φ}, adding extra variables does not change the sum, that is, \lstinline{vars φ ⊆ s} implies
\begin{lstlisting}
  weightCount weight φ s = weightCount weight φ (vars φ)
\end{lstlisting}
The counting scheme of Proposition~\ref{prop:ring:eval} for partitioned formulas is expressed as follows:
\begin{lstlisting}
 def countWeight (weight : ν → Rat) : PropForm ν → Rat
  | tr       => 1
  | fls      => 0
  | var x    => weight x
  | neg φ    => 1 - countWeight weight φ
  | disj φ ψ => countWeight weight φ + countWeight weight ψ
  | conj φ ψ => countWeight weight φ * countWeight weight ψ
\end{lstlisting}
We arbitrarily assign 0 to implications and bi-implications;
recall that, by definition, these do not appear in partitioned formulas.
Proposition~\ref{prop:ring:eval} is then formalized as follows:
\begin{lstlisting}
theorem countWeight_eq_weightCount (weight : ν → R) {φ : PropForm ν}
    (hdec : partitioned φ) :
  countWeight weight φ = weightCount weight φ (vars φ)
\end{lstlisting}

The functions just described are not meant to be computed, but,
rather, serve as our mathematical reference.
We have implemented an efficient function to calculate the weighted model count
for an arbitrary node of a Pog, and we have proved that it computes the function above on the
associated formula:
\begin{lstlisting}
theorem countWeight_eq_countWeight (pog : Pog) (weight : Var → Rat)
    (x : Var) :
  pog.countWeight weight x = (pog.toPropForm (.mkPos x)).countWeight weight
\end{lstlisting}
Here \lstinline{.mkPos x} denotes the positive literal for the variable \lstinline{x}.
Applying this to the output of our verified CRAT proof checker,
we obtain a proof that
our toolchain computes the correct weighted model count of the input CNF.

\lean{} does not yet have an efficient implementation of rational arithmetic, and
we have not yet implemented a verified version of $\drational$,
defined in Section~\ref{sect:experimental}.
We have therefore implemented a separate procedure that carries out an integer
calculation of the number of models.
The number of models that a propositional formula has depends on what we take the
variables to be. For example, there is only one assignment to the variables
$\{ x, y \}$ that makes $x \wedge y$ true, but there are two assignments to the
variables $\{ x, y, z \}$ that make $x \wedge y$ true.
It is not hard to show that for partitioned formulas \lstinline{PropForm ν} whose
variables are among a finite set \lstinline{s} of variables of cardinality \lstinline{numVars},
we can count the number of models of the formula on the variables in \lstinline{s}
recursively as follows:
\begin{lstlisting}
def countModels (nVars : Nat) : PropForm ν → Nat
  | tr       => 2^nVars
  | fls      => 0
  | var _    => 2^(nVars - 1)
  | neg φ    => 2^nVars - countModels nVars φ
  | disj φ ψ => countModels nVars φ + countModels nVars ψ
  | conj φ ψ => countModels nVars φ * countModels nVars ψ / 2^nVars
\end{lstlisting}
We have also implemented an efficient counting function for POGs and proved that
it computes that same quantity on the corresponding formulas.
Finally,
We have proved that the calculation above really does
return the number of models:
\begin{lstlisting}
theorem countModels_eq_card_models {φ : PropForm ν} {s : Finset ν}
    (hvars : vars φ ⊆ s) (hpar :  partitioned φ) :
  countModels (card s) φ = card (models φ s)
\end{lstlisting}
In particular, taking \lstinline{s} to be exactly the variables of \lstinline{φ},
we have that the number of models on its variables is \lstinline{countModels φ (card (vars φ))}.

\subsection{Implementation of the Proof Checker}

We have specified the behavior of the proof checker in Section~\ref{appendix:proof:checker};
in this section we sketch the implementation.
Establishing its correctness was, by far, the most challenging part of the formalization
process.
For clarity, we describe the algorithm with ordinary mathematical notation rather than
Lean snippets.

We begin by reading the input CNF formula $\varphi$ and adding its clauses to a clause
database $C$. We initialize that POG graph $G$ with the variables of $\varphi$,
and then iteratively process each line of the CRAT proof as follows:
\begin{itemize}
  \item (Summarize what each step does.)
\end{itemize}
We establish that all the following invariants hold at each step:
\begin{itemize}
  \item (Wojciech has a long list of invariants)
\end{itemize}
At the end \ldots [Explain how the invariants at the end imply the desired conclusion.]

\subsection{Verification specifics}
\label{subsection:verification:specifics}

In this subsection, we clarify what has been verified and what has been trusted.
Recall that our first step is to parse a CRAT file to read in the initial CNF formula
and the CRAT proof. We do not verify this step.
If one is only interested in obtaining a model count or a verified model count,
there is no strong need to verify the parsing of the proof,
because if the checker succeeds in constructing an equivalent POG,
it does not really matter whether the proof was the one the user intended.
But it would be nice to have reassurance that the CNF formula that serves
as input to our verified checker matches the one described in the file.
An easy and fairly standard way of addressing this is to print out the CNF formula
right after parsing it and then do a diff with the original input.
This involves trusting only the correctness of the print procedure and diff.
Similarly, if one wants to establish the correctness of the POG contained in the CRAT file,
one can print out the POG that is constructed by the checker and compare.

Lean's code extraction replaces calculations with natural numbers and integers with
efficient but unverified arbitrary precision versions.
Lean also uses an efficient implementation of arrays; within the
formal system, these are defined in terms of lists, but code extraction replaces them
with dynamic arrays and uses reference counting to allow destructive updates when it is safe
to do so [cite counting immutable beans paper].
Code extraction also provides efficient implementations of tail recursive functions and loops
[cite ``do unchained'' paper].
Finally, Lean's standard library implements hashmaps in terms of array.
Many of the basic properties of hashmaps have been formally verified, but not all.
In particular, we make use of a fold operation whose verification is not yet complete.
Thus our proofs depend on the assumption that the fold operation has the
expected properties.

In sum, in addition to trusting Lean's foundation and kernel checker,
we also have to trust that code extraction respects that foundation,
that the implementation of arrays and hashmaps satisfy their internal descriptions,
and that, after parsing, the computation has the correct input formula.
All of our specifications have been completely proved and verified relative to that.

\subsection{Evaluation}

\begin{figure}
\centering{%
\begin{tikzpicture}[scale = 0.70]
  \begin{axis}[mark options={scale=0.55},grid=both, grid style={black!10},
      legend style={at={(0.22,0.97)}},
      legend cell align={left},
                              x post scale=2.4, y post scale=2.2,
                              xmode=log,xmin=1000,xmax=1e7, 
                              xtick={1000, 10000, 100000, 1000000, 10000000}, xticklabels={$10^3$,$10^4$,$10^5$,$10^6$,$10^7$},
                              ymode=log, ymin=0.01, ymax=1900,
                              ytick={0.01, 0.1,1.0,10,100,1000}, yticklabels={0.01, 0.1, 1.0, 10, 100, {1,000}},
                              xlabel={CRAT Size (Total clauses)}, ylabel={Checking Time (seconds)},
            ]
    \input{data-formatted/time-check-lean}
    \input{data-formatted/time-check-regular}
    \legend{
      \scriptsize \textsf{\lean{} checker},
      \scriptsize \textsf{Regular checker},
    }

    \addplot[color=black,dashed] coordinates{(0.01,1500) (1e7,1500)};
    \addplot[mark=none, color=lightblue] coordinates{(0.01,1000) (1e7,1000)};
          \end{axis}
\end{tikzpicture}
} % centering
\caption{Times for Regular and \lean{} Checker, as Functions of Total Clauses}
\label{fig:lean}
\end{figure}


We ran the \lean{} checker on the 80 benchmark problems for which the
generated CRAT file had at most 10~million total clauses.  These are the
same benchmarks used in evaluating the optimizations (Section
\ref{app:experiment:optimize}).  Figure~\ref{fig:lean} shows the
comparative performance of the \lean{} checker versus our unverified
checker on the Y axis, with the total number of clauses on the X axis.

As would be expected, the \lean{} checker consistently runs slower
than the unverified checker.  Two of the benchmarks exceeded the time
limit of 1000 seconds with the \lean{} checker.  Of those that
completed, the ratio of the runtime for \lean{} versus the runtime for
the unverified checker had a harmonic mean of 5.9.

Importantly, however, it can be seen that both checkers show the same
overall asymptotic performance.  The ratio of the two runtimes was
less than $8.0$ for all but two very small benchmarks.  This seems like a
reasonable price to pay for a rigorous guarantee of correctness.



