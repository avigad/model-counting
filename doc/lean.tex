\section{A Formally Verified Toolchain}
\label{section:formally-verified-toolchain}

We set out to formally verify the system with two goals in mind: first, to ensure that the CRAT framework is mathematically
sound; and second, to implement correct-by-construction proof checking and ring evaluation (the ``Trusted Code'' components of \cref{fig:chain}). These two goals are achieved with a single proof development in the \lean{} programming language~\cite{demoura:cade:2021}. Verification was greatly aided by the Aesop~\cite{23limperg_aesop_white_box_best_first_proof_search_lean} automated tactic. In this section, we briefly describe the functionality we implemented and what we proved about it. More information is provided in \cref{appendix:lean}.

\vspace{1em}\noindent
\textbf{Proof checking.} The goal of a CRAT proof is to construct a POG that is equivalent to the input CNF $\inputformula$. The checker begins by parsing the input formula, initializing the set of active clauses to $\theta \leftarrow \inputformula$, and initializing the POG $P$ to an empty one. It then processes every step of the CRAT proof, either modifying its state by adding/deleting clauses in $\theta$ and adding nodes to $P$, or throwing an exception if a step is incorrect. Afterwards, it carries out the \textsc{final conditions} check of \cref{subsection:semantics}. Throughout the process, we maintain invariants which ensure that $P$ is partitioned and that a successful final check entails the logical equivalence of $\inputformula$ and $\phi_\noder$, where $\noder$ is the final POG root.

The specifications we use to state these invariants are built on a general theory of propositional logic, mirroring \cref{section:logical:foundations}. Following the DIMACS CNF convention, we define the data types \lstinline{Var} of variables being positive natural numbers, \lstinline{ILit} of literals being non-zero integers, and \lstinline{PropForm Var} of propositional formulas. Assignments of truth values are taken to be total functions \lstinline{PropAssignment Var := Var → Bool}. Requiring totality is not a limitation: instead of talking about two equal, partial assignments to a subset $X' \subseteq X$ of variables, we can more conveniently talk about two total assignments that agree on $X'$. We write \lstinline{σ ⊨ φ} when \lstinline{σ : PropAssignment Var} satisfies \lstinline{φ : PropForm Var}.

The invariants refer to the checker state \lstinline{st} with fields \lstinline{st.inputCnf} for $\inputformula$, \lstinline{st.clauseDb} for $\theta$, \lstinline{st.pog} for $P$, \lstinline{st.pogDefsForm} for the POG definitions formula $\bigwedge_{\nodeu\in P}\theta_u$, and \lstinline{st.allVars} for all variables (original and extension) added so far. For any $\nodeu\in P$, \lstinline{st.pog.toPropForm u} computes $\phi_\nodeu$. The first two invariants state that assignments to original variables extend uniquely to extension variables defining the POG nodes. In the formalization, we split this into two parts, namely, extension and uniqueness:
\begin{lstlisting}
/-- Any assignment satisfying φ₁ extends to φ₂ while preserving values on X. -/
def extendsOver (X : Set Var) (φ₁ φ₂ : PropForm Var) :=
  ∀ (σ₁ : PropAssignment Var), σ₁ ⊨ φ₁ → ∃ σ₂, σ₁.agreeOn X σ₂ ∧ σ₂ ⊨ φ₂

invariants.extends_pogDefsForm : extendsOver st.inputCnf.vars ⊤ st.pogDefsForm

/-- Assignments satisfying φ are determined on Y by their values on X. -/
def uniqueExt (X Y : Set Var) (φ : PropForm Var) :=
  ∀ (σ₁ σ₂ : PropAssignment Var), σ₁ ⊨ φ → σ₂ ⊨ φ → σ₁.agreeOn X σ₂ →
    σ₁.agreeOn Y σ₂

invariants.uep_pogDefsForm : uniqueExt st.inputCnf.vars st.allVars st.pogDefsForm
\end{lstlisting}
The next one guarantees that the set of solutions over the original variables is preserved:
\begin{lstlisting}
def equivalentOver (X : Set Var) (φ₁ φ₂ : PropForm Var) :=
  extendsOver X φ₁ φ₂ ∧ extendsOver X φ₂ φ₁

invariants.equivalent_clauseDb : equivalentOver st.inputCnf.vars
  st.inputCnf st.clauseDb
\end{lstlisting}
Finally, for every node $\nodeu\in P$ with corresponding literal $u$ we ensure that $\phi_\nodeu$ is partitioned (\cref{def:partitioned-operation-formula}) and relate $\phi_\nodeu$ to its clausal encoding $\theta_u \doteq u \wedge \bigwedge_{\nodev\in P}\theta_v$:
\begin{lstlisting}
def partitioned : PropForm Var → Prop
  | tr | fls | var _ => True
  | neg φ    => φ.partitioned
  | disj φ ψ => φ.partitioned ∧ ψ.partitioned ∧ ∀ τ, ¬(τ ⊨ φ ∧ τ ⊨ ψ)
  | conj φ ψ => φ.partitioned ∧ ψ.partitioned ∧ φ.vars ∩ ψ.vars = ∅

invariants.partitioned : ∀ u : ILit, (st.pog.toPropForm u).partitioned

invariants.equivalent_lits : ∀ u : ILit, equivalentOver st.inputCnf.vars
    (u ∧ st.pogDefsForm) (st.pog.toPropForm x)
\end{lstlisting}
These are maintained by valid CRAT proofs. Together with a few additional invariants
that ensure the correctness of cached computations, they imply the soundness theorem for $P$ with root
node $\noder$: the \textsc{final conditions} ensure that $\theta$ is logically equivalent to
$\pogformula \doteq \{\{r\}\} \cup \; \bigcup_{\nodeu \in P} \theta_{u}$
(\lstinline{equivalent st.clauseDb (r ∧ st.pogDefsForm)}), and from the invariants we can prove
$\inputformula \iff \phi_\noder$ (\lstinline{equivalent st.inputCnf (st.pog.toPropForm r)}).
After certifying a proof, the checker can pass its in-memory POG representation to the ring evaluator.
% JA: I cut a few words here to save a line.

\vspace{1em}\noindent
\textbf{Ring evaluation}. We formalized the central quantity (\ref{eqn:rep}) in the ring evaluation problem
(\cref{def:ring_evaluation}) in a commutative ring \lstinline{R} as follows:
\begin{lstlisting}
def weightSum {R : Type} [CommRing R]
    (weight : Var → R) (φ : PropForm Var) (s : Finset Var) : R :=
  ∑ τ in models φ s, ∏ x in s, if τ x then weight x else 1 - weight x
\end{lstlisting}
The rules for efficient ring evaluation of partitioned formulas are expressed as:
\begin{lstlisting}
def ringEval (weight : Var → R) : PropForm Var → R
  | tr       => 1
  | fls      => 0
  | var x    => weight x
  | neg φ    => 1 - ringEval weight φ
  | disj φ ψ => ringEval weight φ + ringEval weight ψ
  | conj φ ψ => ringEval weight φ * ringEval weight ψ
\end{lstlisting}
\Cref{prop:ring:eval} is then formalized as follows:
\begin{lstlisting}
theorem ringEval_eq_weightSum (weight : Var → R) {φ : PropForm Var} :
    partitioned φ → ringEval weight φ = weightSum weight φ (vars φ)
\end{lstlisting}
To efficiently compute the ring evaluation of a formula represented by a POG node, we implemented
\lstinline{Pog.ringEval} and then proved that it matches the specification above:
\begin{lstlisting}
theorem ringEval_eq_ringEval (pog : Pog) (weight : Var → R) (x : Var) :
  pog.ringEval weight x = (pog.toPropForm x).ringEval weight
\end{lstlisting}
Applying this to the output of our verified CRAT proof checker, which we know to be partitioned
and equivalent to the input formula $\inputformula$, we obtain a proof that our toolchain computes
the correct ring evaluation of $\inputformula$.

\vspace{1em}\noindent
\textbf{Model counting.} Finally, we established that ring evaluation with the appropriate weights
corresponds to the standard model count. To do so, we defined a function that
% JA: deleted the word "directly" to save a line
carries out an integer calculation of the number of models over a set of variables
of cardinality \lstinline{nVars}:
\begin{lstlisting}
def countModels (nVars : Nat) : PropForm Var → Nat
  | tr       => 2^nVars
  | fls      => 0
  | var _    => 2^(nVars - 1)
  | neg φ    => 2^nVars - countModels nVars φ
  | disj φ ψ => countModels nVars φ + countModels nVars ψ
  | conj φ ψ => countModels nVars φ * countModels nVars ψ / 2^nVars
\end{lstlisting}
We then formally proved that for a partitioned formula whose variables are among a finite set
\lstinline{s}, this computation really does count the number of models over \lstinline{s}:
\begin{lstlisting}
theorem countModels_eq_card_models {φ : PropForm Var} {s : Finset Var} :
  vars φ ⊆ s → partitioned φ → countModels (card s) φ = card (models φ s)
\end{lstlisting}
In particular, taking \lstinline{s} to be exactly the variables of \lstinline{φ},
we have that the number of models on its variables is \lstinline{countModels φ (card (vars φ))}.