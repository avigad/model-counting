\section{A Formally Verified CRAT Checker and POG Model Counter}

We have implemented the rightmost two components of Figure~\ref{fig:chain}, namely the
proof checker and both model and weighted model counters, in the Lean 4 programming language
and proof assistant [cite].
In this section, we briefly describe the functions we have implemented in Lean and
the specifications we have proved.
More information is provided in the Appendix~\ref{appendix:lean}.

\subsection{Logical objects and operations}

Our specifications are built on a generic library for the syntax and semantics of
propositional logic. The data type of propositional formulas with variables indexed by
a datatype \lstinline{ν} is defined inductively as follows:
\begin{lstlisting}
inductive PropForm (ν : Type)
  | var (x : ν)
  | tr
  | fls
  | neg (φ : PropForm ν)
  | conj (φ₁ φ₂ : PropForm ν)
  | disj (φ₁ φ₂ : PropForm ν)
  | impl (φ₁ φ₂ : PropForm ν)
  | biImpl (φ₁ φ₂ : PropForm ν)
\end{lstlisting}
The last two constructors denote implication and bi-implication, which are not used in this
project. The usual syntactic and semantic notions are defined in the expected ways.
It is convenient to take a truth assignment \lstinline{PropAssignment ν} to a function that
assigns a boolean value to every assignment, and later restrict attention to subsets of the
variables. In our application, we instantiate \lstinline{ν} to a type \lstinline{Var} consisting
of positive natural numbers, corresponding to the indexing of variables in the DIMACS
format for SAT.

We define the following representations for CNF formulas.
A literal is represented as a nonzero integer, a clause is an array of literals,
and a CNF formula is an array of clauses.
\begin{lstlisting}
def ILit := { i : Int // i ≠ 0 }
abbrev IClause := Array ILit
abbrev ICnf := Array IClause
\end{lstlisting}
We also define functions \lstinline{ILit.toPropForm}, \lstinline{IClause.toPropForm},
and \lstinline{ICnf.toPropForm} to relate these to propositional formulas and their
semantics.

\subsection{The Proof Checker}
\label{subsection:proof:checker}

Our proof checker starts by parsing the CNF formula and storing it internally.
It then processes and checks each rule of a CRAT proof, throwing an exception if
a precondition is not met. We have proved that if the checker terminates successfully,
it succeeds in constructing a POG that is equivalent to that CNF.

In greater detail, each element of a POG  is either a variable, a binary disjunction,
or an arbitrary conjunction:
\begin{lstlisting}
inductive PogElt where
  | var  : Var → PogElt
  | disj : Var → ILit → ILit → PogElt
  | conj : Var → Array ILit → PogElt
\end{lstlisting}
In the first case, the argument \lstinline{x} in the expression
\lstinline{var x} is the index
of the variable; in \lstinline{disj x left right} and \lstinline{conj x args}
it is the definition number in the CRAT file.
The CRAT generator described in Section~\ref{section:generating:crat}
declares new variables sequentially, so we currently make the simplifying assumption
that the $i$th element of the array has the node with index $i + 1$,
but it will not be hard to generalize this.
Notice that the arguments to \lstinline{disj} and \lstinline{conj} are literals,
corresponding to positive or negative instances of variables declared earlier in the POG.
So our POG data structure is as follows:
\begin{lstlisting}
structure Pog where
  elts : Array PogElt
  wf : ∀ i : Fin elts.size, elts[i].args_decreasing
  inv : ∀ i : Fin elts.size, i = elts[i].varNum.natPred
\end{lstlisting}
The invariant \lstinline{wf} says that the graph is well founded, in the
sense that if \lstinline{elts[i]} is a disjunction
or a conjunction, then the variables occurring in the arguments have smaller indices.
Note that \lstinline{Fin elts.size} is the data type consisting of natural numbers
less than the size of the array \lstinline{elts}; our code enforces statically
that the array accesses are within bounds.

For each POG \lstinline{P} and literal \lstinline{l}, we define
\lstinline{P.toPropForm l} to be the
propositional formula that arises from interpreting \lstinline{l} as a propositional
formula, unfolding all the defined conjunctions and disjunctions. We define
what it means for such a formula to be partitioned:
\begin{lstlisting}
def partitioned : PropForm ν → Prop
  | tr         => True
  | fls        => True
  | var _      => True
  | neg φ      => φ.partitioned
  | disj φ ψ   => φ.partitioned ∧ ψ.partitioned ∧
                    ∀ τ, ¬ (φ.eval τ ∧ ψ.eval τ)
  | conj φ ψ   => φ.partitioned ∧ ψ.partitioned ∧ (φ.vars ∩ ψ.vars = ∅)
  | impl _ _   => False
  | biImpl _ _ => False
\end{lstlisting}
In other words, we rule out formulas with implication and bi-implication, and otherwise
we follow the straightforward recursive definition.

The core checker reads in a CRAT proof, which includes a specification of the final node \lstinline{r}.
It either throws an exception, indicating that the proof is not well formed,
or returns a POG \lstinline{P}.
We have formally proved the claim that in the latter case the propositional formula \lstinline{P.toPropForm r} corresponding to node \lstinline{r} is partitioned
and is equivalent to \lstinline{cnf.toPropForm}.

\subsection{The Model Counters}

We have formalized the central quantity (\ref{eqn:rep}) in the ring evaluation problem,
Definition~\ref{def:ring_evaluation}, as follows:
\begin{lstlisting}
def weightCount {R : Type} [CommRing R]
    (weight : ν → R) (φ : PropForm ν) (s : Finset ν) : R :=
  ∑ τ in models φ s, ∏ x in s, if τ x then weight x else 1 - weight x
\end{lstlisting}
Here \lstinline{R} is assumed to be a commutative ring.
The counting scheme of Proposition~\ref{prop:ring:eval} for partitioned formulas is expressed as follows:
\begin{lstlisting}
 def countWeight (weight : ν → Rat) : PropForm ν → Rat
  | tr       => 1
  | fls      => 0
  | var x    => weight x
  | neg φ    => 1 - countWeight weight φ
  | disj φ ψ => countWeight weight φ + countWeight weight ψ
  | conj φ ψ => countWeight weight φ * countWeight weight ψ
\end{lstlisting}
We arbitrarily assign 0 to implications and bi-implications;
recall that, by definition, these do not appear in partitioned formulas.
Proposition~\ref{prop:ring:eval} is then formalized as follows:
\begin{lstlisting}
theorem countWeight_eq_weightCount (weight : ν → R) {φ : PropForm ν}
    (hdec : partitioned φ) :
  countWeight weight φ = weightCount weight φ (vars φ)
\end{lstlisting}

We have implemented an efficient function \lstinline{Pog.countWeight} to calculate the weighted model count
for an arbitrary node of a Pog, and we have proved that it computes the function above on the
associated formula:
\begin{lstlisting}
theorem countWeight_eq_countWeight (pog : Pog) (weight : Var → Rat)
    (x : Var) :
  pog.countWeight weight x = (pog.toPropForm (.mkPos x)).countWeight weight
\end{lstlisting}
Here \lstinline{.mkPos x} denotes the positive literal for the variable \lstinline{x}.
Applying this to the output of our verified CRAT proof checker,
we obtain a proof that
our toolchain computes the correct weighted model count of the input CNF.

We have also implemented a separate procedure that directly carries out an integer
calculation of the number of models.
For partitioned formulas \lstinline{PropForm ν} whose
variables are among a finite set \lstinline{s} of variables of cardinality \lstinline{numVars},
we can count the number of models of the formula on the variables in \lstinline{s}
recursively as follows:
\begin{lstlisting}
def countModels (nVars : Nat) : PropForm ν → Nat
  | tr       => 2^nVars
  | fls      => 0
  | var _    => 2^(nVars - 1)
  | neg φ    => 2^nVars - countModels nVars φ
  | disj φ ψ => countModels nVars φ + countModels nVars ψ
  | conj φ ψ => countModels nVars φ * countModels nVars ψ / 2^nVars
\end{lstlisting}
We have implemented an efficient counting function for POGs and proved that
it computes that same quantity on the corresponding formulas.
Finally,
we have proved that the calculation above really does
return the number of models:
\begin{lstlisting}
theorem countModels_eq_card_models {φ : PropForm ν} {s : Finset ν}
    (hvars : vars φ ⊆ s) (hpar :  partitioned φ) :
  countModels (card s) φ = card (models φ s)
\end{lstlisting}
In particular, taking \lstinline{s} to be exactly the variables of \lstinline{φ},
we have that the number of models on its variables is \lstinline{countModels φ (card (vars φ))}.
