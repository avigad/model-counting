\documentclass[letterpaper,USenglish,cleveref, autoref, thm-restate]{lipics-v2021}
%This is a template for producing LIPIcs articles. 
%See lipics-v2021-authors-guidelines.pdf for further information.
%for A4 paper format use option "a4paper", for US-letter use option "letterpaper"
%for british hyphenation rules use option "UKenglish", for american hyphenation rules use option "USenglish"
%for section-numbered lemmas etc., use "numberwithinsect"
%for enabling cleveref support, use "cleveref"
%for enabling autoref support, use "autoref"
%for anonymousing the authors (e.g. for double-blind review), add "anonymous"
%for enabling thm-restate support, use "thm-restate"
%for enabling a two-column layout for the author/affilation part (only applicable for > 6 authors), use "authorcolumns"
%for producing a PDF according the PDF/A standard, add "pdfa"

%\pdfoutput=1 %uncomment to ensure pdflatex processing (mandatatory e.g. to submit to arXiv)
%\hideLIPIcs  %uncomment to remove references to LIPIcs series (logo, DOI, ...), e.g. when preparing a pre-final version to be uploaded to arXiv or another public repository

%\graphicspath{{./graphics/}}%helpful if your graphic files are in another directory

\usepackage{amsmath}
\usepackage{tikz}
\usepackage{pgfplots}
\usepackage{booktabs}

\newcommand{\pand}{\mathbin{\land^{\sf v}}}
\newcommand{\por}{\mathbin{\lor^{\sf a}}}
\DeclareMathOperator*{\Pand}{\bigwedge^{\sf v}}
\DeclareMathOperator*{\Por}{\bigvee^{\sf a}}
\newcommand{\boolnot}{\neg}
\newcommand{\tautology}{\top}
\newcommand{\nil}{\bot}
\newcommand{\obar}[1]{\overline{#1}}
\newcommand{\oneminus}{{\sim}}
\newcommand{\lit}{\ell}

\newcommand{\varset}{X}
\newcommand{\exvarset}{Z}
\newcommand{\dependencyset}{{\cal D}}
\newcommand{\litset}{{\cal L}}
\newcommand{\ring}{{\cal R}}
\newcommand{\dset}{{\cal A}}
\newcommand{\rep}{\textbf{R}}
\newcommand{\radd}{+}
\newcommand{\rmul}{\times}
\newcommand{\addident}{\textbf{0}}
\newcommand{\mulident}{\textbf{1}}

\newcommand{\assign}{\alpha}
\newcommand{\uassign}{{\cal U}}
\newcommand{\modelset}{{\cal M}}

\newcommand{\indegree}{\textrm{indegree}}
\newcommand{\outdegree}{\textrm{outdegree}}

\bibliographystyle{plainurl}% the mandatory bibstyle

\title{Certified Knowledge Compilation \\ with Application to Verified Model Counting}

\titlerunning{Certified Knowledge Compilation}

\author{Randal E. Bryant}{Computer Science Department, Carnegie Mellon University, Pittsburgh, PA 15213 USA}{rebryant@cmu.edu}{https://orcid.org/0000-0001-5024-6613}{Supported by NSF grant CCF-2108521}
\author{Wojciech Nawrocki}{Department of Philosophy, Carnegie Mellon University}{wjnawrock@cmu.edu}{https://orcid.org/0000-0002-8839-0618}{}
\author{Jeremy Avigad}{Department of Philosophy, Carnegie Mellon University}{avigad@cmu.edu}{https://orcid.org/0000-0003-1275-315X}{}
\author{Marijn J. H. Heule}{Computer Science Department, Carnegie Mellon University}{marijn@cmu.edu}{https://orcid.org/0000-0002-5587-8801}{Supported by NSF grant CCF-2108521}

%\authorrunning{R. E. Bryant, et al} %TODO mandatory. First: Use abbreviated first/middle names. Second (only in severe cases): Use first author plus 'et al.'

%\Copyright{Jane Open Access and Joan R. Public} %TODO mandatory, please use full first names. LIPIcs license is "CC-BY";  http://creativecommons.org/licenses/by/3.0/

%\ccsdesc[100]{\textcolor{red}{Replace ccsdesc macro with valid one}} %TODO mandatory: Please choose ACM 2012 classifications from https://dl.acm.org/ccs/ccs_flat.cfm 

%\keywords{Dummy keyword} %TODO mandatory; please add comma-separated list of keywords

%\category{} %optional, e.g. invited paper

%\funding{(Optional) general funding statement \dots}%optional, to capture a funding statement, which applies to all authors. Please enter author specific funding statements as fifth argument of the \author macro.

%\nolinenumbers %uncomment to disable line numbering

%Editor-only macros:: begin (do not touch as author)%%%%%%%%%%%%%%%%%%%%%%%%%%%%%%%%%%
%\EventEditors{John Q. Open and Joan R. Access}
%\EventNoEds{2}
%\EventLongTitle{42nd Conference on Very Important Topics (CVIT 2016)}
%\EventShortTitle{CVIT 2016}
%\EventAcronym{CVIT}
%\EventYear{2016}
%\EventDate{December 24--27, 2016}
%\EventLocation{Little Whinging, United Kingdom}
%\EventLogo{}
%\SeriesVolume{42}
%\ArticleNo{23}
%%%%%%%%%%%%%%%%%%%%%%%%%%%%%%%%%%%%%%%%%%%%%%%%%%%%%%

\newcommand{\program}[1]{\textsc{#1}}
\newcommand{\lean}{\program{lean4}}

\begin{document}

\maketitle

\begin{abstract}
Knowledge compilers convert Boolean formulas into a form for which
useful tasks, including model counting and weighted model counting,
can easily be performed.  Existing knowledge compilers, however,
provide no guarantee that the generated representation accurately
describes the original formula.

We present the Partitioned Operation Graph (POG) representation that
generalizes existing knowledge compiler representations, and a proof
framework that can prove the equivalence of a POG representation with
a conjunctive normal form (CNF) Boolean formula.

We have developed a program that can generate POG representations and
equivalence proofs from the dDNNF graphs generated by a leading
knowledge compiler.  It scales to all but the largest formulas from a
recent model counting competition.  We also present two proof
checkers---a fast checker that can handle the large proofs generated,
and a formally verified checker written in the \lean{} proof system.
\end{abstract}

\section{Introduction}

\section{Logical Foundations}

  Let $\varset$ denote a set of Boolean variables, and let $\assign$
  be an {\em assignment} of truth values to the variables, where $0$
  denotes false and $1$ denotes true, i.e., $\assign \colon \varset
  \rightarrow \{0,1\}$.  The set of all possible assignments over
  $\varset$ is denoted $\uassign$.  For each variable $x \in \varset$,
  we define {\em literals} $x$ and $\obar{x}$, where $\obar{x}$ is the
  negation of $x$. An
  assignment $\assign$ can be viewed as a set of literals, where
  we write $\lit \in \assign$ when $\lit = x$ and $\assign(x) = 1$ or when
  $\lit = \obar{x}$ and $\assign(x) = 0$.  We write the negation of literal $\lit$ as $\obar{\lit}$.  That is, $\obar{\lit} = \obar{x}$ when $\lit = x$ and
$\obar{\lit} = x$ when $\lit = \obar{x}$.

  The set of Boolean formulas is defined recursively.  Each
  formula $\phi$ has an associated {\em dependency set}
  $\dependencyset(\phi)  \subseteq \varset$, and a set of models $\modelset(\phi)$,
  consisting of assignments that satisfy the formula:
  \begin{enumerate}
  \item Boolean constants $0$ and $1$ are Boolean formulas,
    with $\dependencyset(0) = \dependencyset(1) = \emptyset$, with $\modelset(0) = \emptyset$, and with $\modelset(1) = \uassign$.
  \item A variable $x$ is a Boolean formula, with $\dependencyset(x) = \{x\}$
    and $\modelset(x) = \{\assign \in \uassign | \assign(x)=1\}$.
  \item For formula $\phi$, its {\em negation}, written $\boolnot \phi$ is a Boolean formula,
    with $\dependencyset(\boolnot \phi) = \dependencyset(\phi)$ and $\modelset(\boolnot \phi) = \uassign - \modelset(\phi)$.
  \item For formulas $\phi_1, \phi_2, \ldots, \phi_k$, their {\em conjunction} $\phi = \bigwedge_{i=1,k} \phi_i$ is a Boolean formula, with
      $\dependencyset(\phi) = \bigcup_{i=1,k} \dependencyset(\phi_i)$ and
      $\modelset(\phi) = \bigcap_{i=1,k} \modelset(\phi_i)$.
  \item For formulas $\phi_1, \phi_2, \ldots, \phi_k$, their {\em disjunction} $\phi = \bigvee_{i=1,k} \phi_i$ is a Boolean formula, with
      $\dependencyset(\phi) = \bigcup_{i=1,k} \dependencyset(\phi_i)$ and
      $\modelset(\phi) = \bigcup_{i=1,k} \modelset(\phi_i)$.
  \end{enumerate}
  
  We highlight the following special classes of Boolean formulas:
  \begin{itemize}
    \item In a {\em normal form} formula, negation is 
      applied only to variables
    \item A normal-form formula is in {\em conjunctive normal form} (CNF) if disjunction is applied only to literals.
    \item A {\em partitioned-operation formula} (or simply a ``partitioned formula'') satisfies the following restrictions on its conjunction and disjunction operations:
      \begin{itemize}
      \item The arguments to a conjunction must have disjoint dependency sets.  That is, operation
        $\phi = \bigwedge_{i=1,k} \phi_i$ requires $\dependencyset(\phi_i) \cap \dependencyset(\phi_j) = \emptyset$ for $1 \leq i < j \leq k$.
      \item The arguments to a disjunction must have disjoint models.  That is, operation
        $\phi = \bigvee_{i=1,k} \phi_i$ requires $\modelset(\phi_i) \cap \modelset(\phi_j) = \emptyset$ for $1 \leq i < j \leq k$.
      \end{itemize}
     We let $\pand$ and $\por$ denote the conjunction and disjunction operations in a partitioned formula.
  \end{itemize}
  
  \section{Ring Evaluation of a Boolean Formula}

We propose a general framework for describing a variety of tasks that
summarize some aspect of Boolean formulas.

\begin{definition}[Commutative Ring]
  A {\em commutative ring} $\ring$ is an algebraic structure
  $\langle \dset, \radd, \rmul, \addident, \mulident \rangle$, 
  with elements in the set $\dset$ and with commutative and
  associative operations $\radd$ (addition) and $\rmul$ (multiplication),
  such that multiplication distributes
  over addition.  $\addident$ is the additive identity and $\mulident$ is
  the multiplicative identity.  Every element $a \in \dset$ has an
  {\em additive inverse} $-a$ such that $a + -a = \addident$.
\label{def:ring}
\end{definition}

\begin{definition}[Ring Evaluation Problem (REP)]
  A {\em ring labeling} associates a value $w(x) \in \dset$ with
  every variable $x \in \varset$.  We define $w(\obar{x})$ to equal $1-w(x)$.

  For Boolean formula $\phi$, the {\em ring evaluation problem} (REP) is to compute:
  \begin{eqnarray}
    \rep(\phi) & = & \sum_{\alpha \in \modelset(\phi)} \prod_{\lit \in \alpha} w(\ell) \label{eqn:rep}
  \end{eqnarray}
\label{def:labeling}
\end{definition}

A wide variety of important tasks for Boolean functions can be
expressed as ring evaluation problems.  The 
{\em model counting} problem can be cast as an REP by having $\radd$ and
$\rmul$ be addition and multiplication over rational numbers and using
label $w(x) = 1/2$ for every variable $x$.  
Ring evaluation of formula $\phi$ will then compute the {\em density} of
the formula, i.e., the fraction of all possible assignments that are
models.  For $n = |\varset|$, scaling the density by $2^n$ will then
yield the number of models.

The {\em weighted model counting} (WMC) problem is also defined over
rational numbers.  Some formulations of WMC allow independent
assignments of weights $W$ to the variables and their negations, such
that $W(x) + W(\obar{x}) \not = 1$.  We can convert this into a
ring evaluation problem by letting $r(x) = W(x) + W(\obar{x})$,
performing algebraic evaluation with label $w(x) = W(x)/r(x)$ for each
variable $x$, and then computing the final result as $\rep(\phi)\;
\rmul\; \prod_{x \in \varset} r(x)$.  Of course, this requires that $r(x) \not = 0$.

The {\em function hashing problem} provides a test
of inequivalence for Boolean formulas.  That is, let $\ring$ be a
finite  field with $|\dset| \geq 2 |\varset|$.  Two formulas
$\phi_1$ and $\phi_2$ will clearly have $\rep(\phi_1) = \rep(\phi_2)$
if they are logically equivalent.  If they are not equivalent, then
the probability that $\rep(\phi_1) \not = \rep(\phi_2)$ will be at
least $\left(1-\frac{1}{z}\right)^n \geq \left(1-\frac{1}{2n}\right)^n > 1/2$.
Therefore, ring evaluation can be used as part of a
randomized algorithm for equivalence testing~\cite{blum:ipl:1980}.

\section{Partitioned-Operation Graphs (POGs)}

Performing ring evaluation on an arbitrary Boolean formula could be intractable, but it is straightforward for a partitioned formula:
\begin{proposition}
Ring evaluation with operations $\boolnot$, $\pand$, and $\por$ satisfies the following:
\begin{description}
\item[Negation] $\rep(\boolnot \phi) = \mulident - \rep(\phi)$
\item[Partitioned Conjunction] $\rep\left(\Pand_{i=1,k} \phi_i\right) = \prod_{i=1,k} \rep(\phi_i)$
\item[Partitioned Disjunction] $\rep\left(\Por_{i=1,k} \phi_i\right) = \sum_{i=1,k} \rep(\phi_i)$
\end{description}
\end{proposition}

A {\em partitioned-operation graph} (POG) is a directed, acyclic graph
with nodes $N$ and edges $E \subseteq N \times N$.  When $(u,v) \in E$,
node $v$ is said to be a {\em child} of node $u$.
The in- and out-degrees of node $u$ are defined as $\indegree(u) = | E \cap (N \times \{u\}) |$, and
$\outdegree(u) = | E \cap (\{u\} \times N) |$.
Node $u$ is said to be {\em terminal} if $\outdegree(u) = 0$.  A terminal node is labeled by a Boolean constant or variable.
Node $u$ is said to be {\em nonterminal} if $\outdegree(u) > 0$.  A nonterminal node is labeled by one of three Boolean operations:
$\boolnot$, $\pand$, or $\por$.  Node $u$ can be labeled with $\boolnot$ only if $\outdegree(u) = 1$.
It can be labeled with operation $\pand$ or $\por$ only if it satisfies the partitioning restriction for that operation.
There is a unique {\em root node} $r$ such that $\indegree(r) = 0$.

As can be seen, a POG is simply a way to represent a partitioned
formula with a sharing of common subformulas, and so we write
$\phi(P)$ as the formula denoted by $P$.

We define the {\em size} of POG $P$, written $|P|$ to be the sum of
the number of nonterminal nodes and the number of edges.  Ring
evaluation of $\phi(P)$ can be performed with at most $|P|$ ring
operations by traversing the graph from the terminal nodes up to
the root, computing a value $\rep(u)$ for each node $u$.

POGs are inspired by the d-DNNF graphs devised by Darwiche for
representing Boolean functions \cite{darwiche:jair:2002}.
POGs generalize d-DNNF in two ways:
\begin{itemize}
\item They allow negation of arbitrary nodes in the graph, not just
  variables.  As we have seen, ring evaluation can readily handle negation, and so we allow it in the interest of generality.
  Negation could be useful in some contexts, such as when
  converting binary decision diagrams (BDDs) having complemented
  edges~\cite{brace-dac-1990,minato-dac-1990} into POGs.

\item It allows arbitrary arguments to a $\por$ operation, as long as
  they have disjoint models.  By contrast, each
  disjunction node in a d-DNNF can have at most two children, and there must be a variable $x$ such that
  the satisfying assignments for one child have $x=1$ while those for the
  other child have $x=0$.  Our generalization is provided to allow an
  encoding of Sentential Decision Diagrams (SDDs)~\cite{darwiche:ijcai:2011} into
  POGs.  This is discussed in Section~\ref{sect:future}, describing possible future research.
\end{itemize}
  Both of these generalizations allow more flexibility in the POG
  representation while maintaining the ability to efficiently perform ring evaluation.


\section{The CRAT Proof System}



The CRAT clausal proof framework provides a means to express a
checkable proof that a Boolean formula, given in conjunctive normal
form, is logically equivalent to a POG\@, thus enabling certified
knowledge compilation.  It is based on extended
resolution~\cite{Tseitin:1983} with reverse unit propagation
(RUP)~\cite{goldberg,vangelder08_verifying_rup_proofs} as its core method for
determining that a clause is implication-redundant with respect to other clauses.

Two programs are involved in producing a certified compilation of a formula:
\begin{itemize}
\item The {\em compiler} generates a CRAT file that both describes a POG and provides a proof that this POG is equivalent to the input formula.
  This compiler might start with the output of another knowledge compiler.
\item The {\em checker} ensures that all of the proof conditions are satisfied.  
\end{itemize}

The CRAT format draws its inspiration from the LRAT format for Boolean
formulas~\cite{lrat} and the QRAT format for quantified Boolean
formulas (QBF)~\cite{heule:JAR2014}.  The following are its key properties:
\begin{itemize}
  \item
  In addition to explicit clause additions and deletions, the proof contains declarations of $\pand$ and $\por$ operations.
\begin{itemize}
\item 
  Each such declaration implicitly adds an extension variable representing that operation and a set of {\em defining} clauses
  encoding the conjunction or disjunction operation.
  This is the only means for generating extension variables or adding blocked clauses to the proof.
\item The checker must track the dependency set for every input and
  extension variable.  When an extension variable is introduced based
  on the $\pand$ operation, the dependency sets of its arguments must
  be disjoint.  The resulting extension variable has a dependency set equal to the union of those of its arguments.
\item Declaring a $\por$ operation requires a sequence of clauses
  providing a RUP proof that the arguments are mutually exclusive.
  Only binary $\por$ operations are allowed to avoid requiring
  $k\,(k-1)/2$ proofs of disjointness for larger values of $k$.
\item Boolean negation is provided implicitly by allowing the
  arguments of the $\por$ and $\pand$ operations to be literals and not just
  variables.
\end{itemize}
\item
  Assertion clauses can can added when they satisfy the RUP
  property, with respect to a sequence of existing clauses.  This set must be provided explicitly as the {\em hints} for the assertion.
\item
  Deleting clauses requires proving that the resulting formula is not weaker.
  That is, it should not allow any additional satisfying assignment.
  Only the input clauses and those declared by an assertion can be deleted.
      The deletion must be accompanied by a sequence of remaining clauses
      providing a RUP proof of the clause.  This sequence must be provided as the {\em hints} for the deletion.
\end{itemize}

\subsection{Syntax}

\begin{table}
  \caption{CRAT Step Types.  $C$: clause identifier, $L$: literal, $V$: variable}
  \label{tab:crat:syntax}
\centering{
  \begin{tabular}{lllll}
    \multicolumn{4}{c}{Rule} & \multicolumn{1}{c}{Description} \\
    \midrule
    \makebox[5mm][l]{$C$} & \makebox[10mm][l]{\tt a}  & \makebox[15mm][l]{$L^{*}$ {\tt 0}} & \makebox[15mm][l]{$C^{+}$ {\tt 0}}  & \makebox[20mm][l]{Add RUP clause} \\
     & {\tt dc} & $C$             & $C^{+}$  {\tt 0} & Delete RUP clause \\
    \midrule
    $C$    & {\tt p} & $V \; L^{*}$ {\tt 0}    &                  & Declare $\pand$ operation \\
    $C$    & {\tt s} & $V \; L \; L$    & $C^{+}$ {\tt 0}  & Declare $\por$ operation \\ 
    \midrule
     & {\tt r} & $L$             &            & Declare root literal\\
  \end{tabular}
  }
\end{table}

Table~\ref{tab:crat:syntax} shows the declarations that can occur in a CRAT file.
The checker is provided with the input formula as a separate file.
As with other clausal proof formats, a variable is
represented by a positive integer $v$, with the first ones being input
variables and successive ones being extension variables.  Literal $\lit$
is represented by a signed integer, with $-v$ being the complement of
variable $v$.  Each clause is indicated by a positive integer
identifier $C$, with the first ones being the IDs of the input clauses and successive
ones being the IDs of added clauses.  Clause identifiers must be totally ordered,
such that clause $C$ can only reference clauses $C'$ such that $C' < C$.

The first set of proof rules are similar to those in other clausal
proofs.
Clauses can be added via RUP addition
(command {\tt a}), with a sequence of antecedent clauses (the
``hint'').
Similarly for clause deletion (command {\tt dc}).

The declaration of a $\pand$ operation has the form:
\begin{center}
\begin{tabular}{ccccccccc}
  \makebox[5mm]{$i$} & \makebox[5mm]{{\tt p}} & \makebox[5mm]{$v$} & \makebox[5mm]{$\lit_1$} & \makebox[5mm]{$\lit_2$} &
  \makebox[5mm]{$\cdots$} & \makebox[5mm]{$\lit_k$} & \makebox[5mm]{\tt 0} \\
\end{tabular}
\end{center}
where $i$ is a new clause ID, $v$ is a positive integer that does not
correspond to any previous variable, and $\lit_1, \lit_2, \ldots, \lit_k$ is a sequence of $k$  
integers representing literals of existing variables.  
This declaration implicitly causes $k+1$ clauses to be added to the proof:
\begin{center}
\begin{tabular}{cccccc}
\makebox[10mm]{ID} & \multicolumn{5}{c}{Clause} \\
  $i$ & $v$ & $-\lit_1$ & $-\lit_2$ & $\cdots$ & $-\lit_k$\\
  $i\!+\!1$ & $-v$ & $\lit_1$  \\
  $i\!+\!2$ & $-v$ & $\lit_2$  \\
  & $\ldots$ \\
  $i\!+\!k$ & $-v$ & $\lit_k$  \\
\end{tabular}
\end{center}
The dependency sets for the arguments represented by each pair of
literals $\lit_i$
and $\lit_{j}$ must
be disjoint, for $1 \leq i < j \leq k$.  A $\pand$ operation may have no arguments,
representing Boolean constant $1$.  The only clause added to the proof will be
the unit literal $v$.  A reference to literal $-v$ then becomes a way
of representing constant $0$.


The declaration of a $\por$ operation has the form:
\begin{center}
\begin{tabular}{ccccccc}
  \makebox[5mm]{$i$} & \makebox[5mm]{{\tt s}} & \makebox[5mm]{$v$} & \makebox[5mm]{$\lit_1$} & \makebox[5mm]{$\lit_2$} 
\makebox[5mm]{$H$} & \makebox[5mm]{$\texttt{0}$} \\
\end{tabular}
\end{center}
where $i$ is a new clause ID, $v$ is a positive integer that does
not correspond to any previous variable, and $\lit_1$ and $\lit_2$ are
signed integers representing literals of existing variables.  Hint $H$
consists of a
sequence of clause IDs.
This declaration implicitly causes three clauses to be added to the proof:
\begin{center}
\begin{tabular}{cccc}
\makebox[10mm]{ID} & \multicolumn{3}{c}{Clause} \\
  $i$ & $-v$ & $\lit_1$ & $\lit_2$ \\
  $i\!+\!1$ & $v$ & $-\lit_1$ \\
  $i\!+\!2$ & $v$ & $-\lit_2$ \\
\end{tabular}
\end{center}
The hints must provide a RUP proof of the clause $\obar{\lit}_1 \lor \obar{\lit}_2$.

Finally, the literal denoting the root of the POG is declared with the
{\tt r} command.  It can occur anywhere in the file.  Typically, it
will be the extension variable representing the root of a graph, but in
degenerate cases it can be the positive or negative literal of an
input variable.

\subsection{Semantics}

Observe that the defining clauses for the conjunction and disjunction
operations provide a full Tseitin encoding of the operations.  Letting
$\exvarset$ denote the set of added extension variables,  any
assignment $\assign\colon \varset \rightarrow \{0,1\}$ has a unique extension,
which we also call $\assign$, with $\assign\colon (\varset \cup \exvarset) \rightarrow \{0,1\}$.

A CRAT proof follows the same general form as a QRAT dual
proof~\cite{bryant:cade:2021}---one that ensures that each clause
addition and each clause deletion preserves equivalence.  With CRAT,
however, clauses are defined both explicitly and implicitly.  Starting
with the set of input clauses, the proof consists of a sequence of
steps that both add and delete clauses.  Each addition must be truth
preserving, that is, any satisfying assignment for the set of clauses
before clause addition should still be a satisfying assignment afterwards.
Each deletion must be falsehood preserving.  That is, there can be no
new satisfying assignments when the clause is deleted.

At the completion of the proof, the following conditions must hold:
\begin{itemize}
\item All of the input clauses have been deleted.
\item There should be exactly one clause left that was added via RUP
  addition, and this should be a unit clause consisting of the root literal $r$.
\end{itemize}
The sequence of clause addition steps provide a proof that any
assignment $\assign$ to the input variables that satisfies all of the
clauses must, when extended, assign $\assign(r) = 1$.  Conversely,
each proof step that deletes an input clause proves that any
assignment $\alpha$ that falsifies the clause must, when extended,
assign $\assign(r) = 0$.

The sequence of operator declarations, asserted clauses, and deleted
clauses represents a systematic transformation of the input formula
into a POG\@.  Validating each of these steps serves to prove that the
POG is logically equivalent to the input formula.

\subsection{CRAT Example}

\begin{figure}

  \centering{
\begin{minipage}{0.58\textwidth}
(A)  Input Formula\\[1.2ex]
\begin{tabular}{lll}
\toprule
\makebox[5mm]{ID} & \makebox[15mm]{Literals} & \\
\midrule
1 & \texttt{-1 3 -4} & \texttt{0} \\
2 & \texttt{-1 -3 4} & \texttt{0} \\
3 & \texttt{3 -4} & \texttt{0}\\
4 & \texttt{1 -3 4} & \texttt{0} \\
5 & \texttt{-1 -2} & \texttt{0} \\
\bottomrule
\end{tabular}
\\[1.8ex]
(B) CRAT Declaration of POG\\[1.2ex]
\begin{tabular}{llll}
\toprule
\makebox[5mm]{ID} & \multicolumn{2}{l}{Proof line} & Explanation \\
\midrule
6 & \texttt{p 5 -3 -4} & \texttt{0} & $p_5 = \obar{x}_3 \pand \obar{x}_4$ \\
9 & \texttt{p 6 3 4} & \texttt{0} & $p_6 = x_3 \pand x_4$ \\
12 & \texttt{s 7 5 6 7 10} & \texttt{0} & $s_7 = p_5 \por p_6$ \\
15 & \texttt{p 8 -1 7} & \texttt{0} & $p_8 = \obar{x}_1 \pand s_7$ \\ 
18 & \texttt{p 9 1 -2 7} & \texttt{0} & $p_9 = x_1 \pand \obar{x}_2 \pand s_7$ \\
22 & \texttt{s 10 8 9 16 19} & \texttt{0} & $s_{10} = p_8 \por p_9$ \\
 & \texttt{r 10} && Root $r = s_{10}$\\
\bottomrule
\end{tabular}
\end{minipage}
\begin{minipage}{0.40\textwidth}
(C) POG Visualization\\
\input{dd/eg4}
\end{minipage}
}
\caption{Example CNF Formula and POG Representation}
\label{fig:eg4:pog}
\end{figure}

Figure \ref{fig:eg4:pog} illustrates an example formula and shows how
the CRAT file declares its POG representation.  The input formula (A)
consists of five clauses over variables $x_1$, $x_2$, $x_3$, and
$x_4$.  The generated POG has six nonterminal nodes representing four
conjunctions and two disjunctions.  We name these by the node type (product or sum) and the extension variable.
The first part of the CRAT file
(B) declares these nodes using clause IDs that increment by three or
four, depending on whether the node has two children or three.
The last two nonzero values in the sum declarations are hints providing the required mutual exclusion proof.


\begin{figure}
\begin{minipage}{0.49\textwidth}
(A) Implicit clauses\\[1.2ex]
\begin{tabular}{llll}
\toprule
\makebox[5mm]{ID} & \multicolumn{2}{l}{Proof line} & Explanation \\
\midrule
\texttt{6} & \texttt{5 3 4} & \texttt{0} & Define $p_5$ \\
\texttt{7} & \texttt{-5 -3} & \texttt{0} & \\
\texttt{8} & \texttt{-5 -4} & \texttt{0} & \\
\midrule
\texttt{9} & \texttt{6 -3 -4} & \texttt{0} & Define $p_6$ \\
\texttt{10} & \texttt{-6 3} & \texttt{0} & \\
\texttt{11} & \texttt{-6 4} & \texttt{0} & \\
\midrule
\texttt{12} & \texttt{-7 5 6} & \texttt{0} & Define $s_7$ \\
\texttt{13} & \texttt{7 -5} & \texttt{0} & \\
\texttt{14} & \texttt{7 -6} & \texttt{0} & \\
\midrule
\texttt{15} & \texttt{8 1 -7} & \texttt{0} & Define $p_8$ \\
\texttt{16} & \texttt{-8 -1} & \texttt{0} & \\
\texttt{17} & \texttt{-8 7} & \texttt{0} & \\
\midrule
\texttt{18} & \texttt{9 -1 2 -7} & \texttt{0} & Define $p_9$ \\
\texttt{19} & \texttt{-9 1} & \texttt{0} & \\
\texttt{20} & \texttt{-9 -2} & \texttt{0} & \\
\texttt{21} & \texttt{-9 7} & \texttt{0} & \\
\midrule
\texttt{22} & \texttt{-10 8 9} & \texttt{0} & Define $s_{10}$ \\
\texttt{23} & \texttt{10 -8} & \texttt{0} & \\
\texttt{24} & \texttt{10 -9} & \texttt{0} & \\
\bottomrule
\end{tabular}
\end{minipage}
\begin{minipage}{0.49\textwidth}
(B) Assertions\\[1.2ex]
\begin{tabular}{llllll}
\toprule
\makebox[5mm]{ID} & \multicolumn{2}{l}{Proof line} & \multicolumn{2}{l}{Hints} & Explanation \\
\midrule
\texttt{25} & \texttt{5 1 3} & \texttt{0} & \texttt{3 6} & \texttt{0} & $\obar{x}_1 \land \obar{x}_3 \rightarrow p_5$ \\
\texttt{26} & \texttt{6 1 -3} & \texttt{0} & \texttt{4 9} & \texttt{0} & $\obar{x}_1 \land x_3 \rightarrow p_6$ \\
\texttt{27} & \texttt{3 7 1} & \texttt{0} & \texttt{13 25} & \texttt{0} & $\obar{x}_3 \land \obar{x}_1 \rightarrow s_7$  \\
\texttt{28} & \texttt{7 1} & \texttt{0} & \texttt{27 14 26} & \texttt{0} & $\obar{x}_1 \rightarrow s_7$  \\
\texttt{29} & \texttt{8 1} & \texttt{0} & \texttt{28 15} & \texttt{0} & $\obar{x}_1 \rightarrow p_8$  \\
\texttt{30} & \texttt{5 -1 3} & \texttt{0} & \texttt{1 6} & \texttt{0} & $x_1 \land \obar{x}_3 \rightarrow p_5$ \\
\texttt{31} & \texttt{6 -1 -3} & \texttt{0} & \texttt{2 9} & \texttt{0} & $x_1 \land x_3 \rightarrow p_6$ \\
\texttt{32} & \texttt{3 7 -1} & \texttt{0} & \texttt{13 30} & \texttt{0} & $\obar{x}_3 \land x_1 \rightarrow s_7$  \\
\texttt{33} & \texttt{7 -1} & \texttt{0} & \texttt{32 14 31} & \texttt{0} & $x_1 \rightarrow s_7$  \\
\texttt{34} & \texttt{9 -1} & \texttt{0} & \texttt{5 33 18} & \texttt{0} & $x_1 \rightarrow p_9$  \\
\texttt{35} & \texttt{1 10} & \texttt{0} & \texttt{23 29} & \texttt{0} & $\obar{x}_1 \rightarrow s_{10}$  \\
\texttt{36} & \texttt{10} & \texttt{0} & \texttt{35 24 34} & \texttt{0} & $s_{10}$ \\
\bottomrule
\end{tabular}
\\[1.2ex]
(C) Input clause deletions\\[1.2ex]
\begin{tabular}{lllll}
  \toprule
 \multicolumn{3}{l}{Proof line} & Explanation\\
\midrule
 \texttt{dc 1} & \texttt{36 8 10 12 16 21 22} & \texttt{0} & Delete clause 1 \\
 \texttt{dc 2} & \texttt{36 7 11 12 16 21 22} & \texttt{0} & Delete clause 2 \\
 \texttt{dc 3} & \texttt{36 8 10 12 17 19 22} & \texttt{0} & Delete clause 3 \\
 \texttt{dc 4} & \texttt{36 7 11 12 17 19 22} & \texttt{0} & Delete clause 4 \\
 \texttt{dc 5} & \texttt{36 16 20 22} & \texttt{0} &  Delete clause 5 \\
\end{tabular}
\end{minipage}
\caption{Defining Clauses (A), Assertions (B), and Input Clause Deletions (C) for Example Formula}
\label{fig:eg4:proof}
\end{figure}

We step through a detailed example to provide a better understanding of the CRAT proof framework.
Figure
\ref{fig:eg4:proof}(A) shows the defining clauses that are implicitly
defined by the POG operation declarations.  These do not appear in the
CRAT file.  Referring back to the declarations of the sum nodes in
Figure \ref{fig:eg4:pog}, we can see that the declaration of node
$s_7$ had clause IDs 7 and 10 as hints.  We can see in Figure
\ref{fig:eg4:proof}(A) that these two clauses resolve to the clause
$\obar{p}_5 \lor \obar{p}_6$, showing that the two children of $s_7$
have disjoint models.  Similarly, node $s_{10}$ is declared as having
clause IDs 16 and 19 as hints.  These resolve to the clause
$\obar{p}_8 \lor \obar{p}_9$, showing that the two children of
$s_{10}$ have disjoint models.

Figure \ref{fig:eg4:proof}(B) provides the sequence of assertions
leading to unit clause 36, consisting of the literal $s_{10}$.  This clause indicates that $s_{10}$ is implied by the input clauses, i.e.,
any assignment $\assign$
satisfying the input clauses must extend to one with $\assign(s_{10}) = 1$.
Working backward, we can see that
clause 29 indicates that node $p_8$ will be implied by the input
clauses when $\assign(x_1) = 0$, while clause 34 indicates that node $p_9$ will
be implied by the input clauses when $\assign(x_1) = 1$.  These serve as the
hints for clause 36.

Figure \ref{fig:eg4:proof}(C) shows the RUP proof steps required to
delete the input clauses.  Consider the first of these, deleting
input clause $\obar{x}_1 \lor x_3 \lor \obar{x}_4$.  The requirement is to show
that there is no assignment $\assign$ that falsifies this clause, but assigns $\assign(s_{10}) = 1$.
proof proceeds by first assuming that the clause is false, requiring
$\assign(x_1) = 1$, $\assign(x_3) = 0$, and $\assign(x_4) = 1$.  The hints then consist of unit
clauses (e.g., clause 36 asserting that $\alpha(p_{10}) = 1$) or
clauses that cause unit propagation.  Hint clauses 8 and 10 force the
assignments $\assign(p_5) = \assign(p_6) = 0$.  These, plus hint clause 12 forces
$\assign(s_7) = 0$.  This, plus hint clauses 16 and 21 force $\assign(p_8) = \assign(p_9) = 0$, leading
via clause 22 to $\assign(s_10) = 0$.  But this contradicts clause 22,
completing the RUP proof.  The deletion hints for the other input
clauses follow similar patterns---they work from the bottom nodes of
the POG upward, showing that any assignment that falsifies the clause
must assign $s_{10} = 0$.

Deleting the asserted clauses so simple that we do not show it.  It
involves simply deleting the clauses from clause number 35 down to
clause number 25, with each deletion using the same hints as were used
to add that clause.  In the end, therefore, only the defining clauses
for the POG nodes and the unit clause asserting $s_{10}$ remain,
completing a proof that the POG is logically equivalent to the input
formula.

Observe that this proof must ``visit'' nodes
$p_5$ and $p_6$ twice, separately considering assignments where $\assign(x_1) = 0$
(clauses 25 and 26) and where $\assign(x_1) = 1$ (clauses 30 and 31).  In
general, a naive proof must expand the POG into a tree.
We address this shortcoming by introducing {\em lemmas}, as is described in Section~\ref{sect:lemma}



\section{Generating CRAT from dDNNF}

\label{sect:lemma}

\section{A Formally Verified CRAT Checker}

\section{Experimental Evaluation}

\section{Future Work}
\label{sect:future}


Each node $u$ in an $SDD$ has children $p_1, p_2, \ldots, p_k$ and
$s_1, s_2, \ldots p_k$, where the Boolean formula $\phi(u)$ associated
with the node is defined recursively as $\phi(u) = \bigwedge_{i=1,k}
[\phi(p_i) \land \phi(s_i)]$.  Moreover, each children $p_i$ and $p_j$
such that $1 \leq i < j \leq k$ must satisfy $\modelset(\phi(p_i))
\cap \modelset(\phi(p_j)) = \emptyset$, and for every $p_i$ and $s_j$
such that $1 \leq i,j \leq k$, we must have $\dependencyset(\phi(p_i))
\cap \dependencyset(\phi(s_j)) = \emptyset$.  The POG representation
of such a node could have $k$ conjunction nodes: $t_i = p_i \pand s_i$
for $1 \leq i \leq k$, and then $k-1$ binary disjunction nodes to form
the disjunction of $t_1, \ldots, t_k$.

\bibliography{references}


\appendix

\section{Experimental Methodology}

\section{Detailed Experimental Results}



\end{document}
