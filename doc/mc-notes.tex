\documentclass{llncs}

\usepackage{latexsym}
\usepackage{times}
\usepackage{amsmath}
\usepackage{graphicx}
\usepackage{cite}
\usepackage{booktabs}
\usepackage{adjustbox}
\usepackage{marvosym}
%% Colored hyperlink 
\newcommand{\cref}[2]{\href{#1}{\color{blue}#2}}
%% Colored hyperlink showing link in TT font
% \newcommand{\chref}[1]{\href{#1}{\small\tt \color{blue}#1}}
\newcommand{\hcref}[1]{\cref{#1}{\small\tt #1}}

\bibliographystyle{splncs04}

\newcommand{\one}{\mbox{\bf 1}}
\newcommand{\zero}{\mbox{\bf 0}}
\newcommand{\leafone}{L_1}
\newcommand{\leafzero}{L_0}
\newcommand{\booland}{\land}
\newcommand{\boolor}{\lor}
\newcommand{\boolxor}{\oplus}
\newcommand{\boolnot}{\neg}
\newcommand{\tautology}{\top}
\newcommand{\nil}{\bot}
\newcommand{\obar}[1]{\overline{#1}}
\newcommand{\ite}{\mbox{\it ITE}}

\newcommand{\opname}[1]{\mbox{\sc #1}}
\newcommand{\andop}{\opname{And}}
\newcommand{\implyop}{\opname{Imply}}

\newcommand{\turnstile}{\vdash}

\newcommand{\fname}[1]{\mbox{\small\sf #1}}

\newcommand{\lo}{\fname{Lo}}
\newcommand{\hi}{\fname{Hi}}
\newcommand{\var}{\fname{Var}}
\newcommand{\val}{\fname{Val}}

\newcommand{\interp}{\sigma}
\newcommand{\interpset}[1]{\Sigma_{#1}}
\newcommand{\mcount}{\mu}

\newcommand{\ifarg}{\textbf{I}}
\newcommand{\thenarg}{\textbf{T}}
\newcommand{\elsearg}{\textbf{E}}
\newcommand{\depend}{{\it D}}

\newcommand{\subs}[2]{[#1/#2]}
\newcommand{\substrue}[1]{\subs{#1}{\tautology}}
\newcommand{\subsfalse}[1]{\subs{#1}{\false}}
\newcommand{\subsflip}[1]{\subs{#1}{\obar{#1}}}

\title{Notes on Validated Model Counting}

\author{Randal E. Bryant}

\institute{
Computer Science Department \\
Carnegie Mellon University, Pittsburgh, PA, United States
}

\date{\today}

\begin{document}

\maketitle

\section{Notation}

Consider Boolean formulas over a set of variables $X$.  An
assignment $\interp$ is a function mapping each variable to 
truth value $\tautology$ (true) or $\nil$ (false).  We can extend
$\interp$ to Boolean formulas in the normal way, such that
$\interp(F)$ will be $\tautology$ (respectively, $\nil$) if the
evaluation of formula $F$ yields $\tautology$ (resp., $\nil$) when its
variables are assigned values according to $\interp$.

For Boolean formula $F$, we define its set of {\em models} $\interpset{F}$ as
\begin{eqnarray}
\interpset{F} & = & \{ \interp | \interp(F) = \tautology \}
\end{eqnarray}
The task of model counting is, given formula $F$, to determine the size of its set of models
$\mcount(F) = |\interpset{F}|$.  Ideally, this should be done without actually enumerating the set.

For interpretation $\interp$ and a Boolean formula $E$ over $X$, we
use the notation $\interp\subs{x}{E}$ to denote the intepretation
$\interp'$, such that $\interp'(y) = \interp(y)$ for all $y \not = x$
and $\interp'(x) = \interp(E)$.  In particular, the notation
$\interp\subsflip{x}$ indicates the interpretation in which the value
assigned to $x$ is complemented, while others remain unchanged.

A Boolean formula $F$ is said to be {\em independent} of variable $x$
if every $\interp \in \interpset{F}$ has $\interp\subsflip{x} \in
\interpset{F}$.

\begin{lemma}
\label{lemma:independent:split}
if Boolean formula $F$ is independent of variable $x$, then:
\begin{equation}
|\interpset{F}| \; = \; 2\cdot |\interpset{F\land x}| = \; 2\cdot |\interpset{F\land \obar{x}}| \label{eqn:independent}
\end{equation}  
\end{lemma}  
That is, the set of interpretations of $F$ are split evenly between
those assigning $\tautology$ to $x$ and those assigning $\nil$ to $x$.

\section{ITE Operation}

We consider a single Boolean operation, known as
``If-Then-Else,'' or simply ``ITE.''  For Boolean values $a, b, c$,
the operation is defined as $\ite(a,b,c) = (a \land b) \lor (\obar{a}
\land c)$.  This single operation can be used to express several common Boolean operations:
\begin{eqnarray*}
\obar{a} & = & \ite(a, \nil, \tautology) \\
a \land b & = & \ite(a, b, \nil) \\
a \lor b & = & \ite(a, \tautology, b) \\
a \rightarrow b & = & \ite(a, b, \tautology) \\
\end{eqnarray*}  

For variable $x$ and Boolean formulas $F$ and $G$, the set of models
for $\ite(x, F, G)$ is given by the formula:
\begin{eqnarray}
  \interpset{\ite(x, F, G)} & = & \{ \interp \in \interpset{F} | \interp(x) = \tautology \}
  \cup \{ \interp \in \interpset{G} | \interp(x) = \nil \} \label{eqn:ite:interp}
\end{eqnarray}

\section{ITE Graphs}
Model counting is especially simple for a class of formulas we
call ``free ITE graphs.''  This representation can be obtained
directly from an Ordered Binary Decision Diagram (OBDD) representation
of a Boolean function~\cite{Bryant:1986}.

An {\em ITE graph} is defined to be a directed acyclic graph,
consisting of three node types:
\begin{description}
\item[Constant:] Corresponds to value $\tautology$ or $\nil$.  A constant node has no incoming arcs.
\item[Input:] Corresponds to one of the variables in $X$.   A variable node has no incoming arcs.
\item[Operator:] Represents an application of the $\ite$ operation.  An operator node has three incoming arcs, labeled \ifarg{}, \thenarg{}, and \elsearg{}, corresponding to the three arguments of the $\ite$ operation.
\end{description}  

A {\em proper} ITE graph is one for which the incoming \ifarg{} arc for every operator node is
from an input node.  That is, the result of one $\ite$ operation can
serve as the ``then'' or ``else'' argument to another $\ite$
operation, but not for the ``if'' argument.

When describing proper ITE graphs, we refer to constant and input
nodes by their associated value or variable.  An operator node is described by an expression of the form
$\ite(x, v_t, v_e)$, where $v_t$ and $v_e$ are the
nodes corresponding to the \thenarg{} and \elsearg{} inputs, respectively.

We can define the function $\depend$ mapping each node
$v$ in a proper ITE graph to the set of variables on which it logically depends.  This can be expressed recursively as
\begin{eqnarray}
\depend(\tautology) & = & \emptyset \label{eqn:depend:true}\\
\depend(\nil) & = & \emptyset \label{eqn:depend:false}\\
\depend(x) & = & \{x \} \label{eqn:depend:variable}\\
\depend(\ite(x, v_t, v_e)) & = & \{ x \} \cup \depend(v_t) \cup \depend(v_e) \label{eqn:depend:ite}
\end{eqnarray}  


\begin{lemma}
\label{lemma:independent:dset}
Any node $v$ in an ITE graph is independent of any variable $y \in X$ such that $y \not \in \depend(v)$.
\end{lemma}  
\noindent
{\em Proof:} Follows by the recursive definition of $\depend$ (\ref{eqn:depend:true}--\ref{eqn:depend:ite}) and by (\ref{eqn:ite:interp}).


A {\em free} ITE graph is a proper ITE graph satisfying the added
property that for every operator node $v = \ite(x, v_t, v_e)$, its
incoming \thenarg{} and \elsearg{} arcs must be from nodes such that
$x \not \in \depend(v_t)$ and $x \not \in \depend(v_e)$.

Free ITE graphs correspond directly to a class of binary decision diagrams known as {\em free} BDDs (FBDDs)~\cite{gergov:ieeetc:1994}.  Ordered BDDs are a subclass of FBDDs.

\section{Model Counting Free ITE Graphs}

While model counting is a difficult problem for arbitrary
formulas, it readily be computed when $F$ takes the form of a free
ITE graph.

\begin{theorem}
\label{theorem:count}
Letting $n = |X|$, the following recursive formula holds for any node in a free ITE graph
\begin{eqnarray}
\mcount(\tautology) & = & 2^{n} \label{eqn:count:true}\\
\mcount(\nil) & = & 0 \label{eqn:count:false}\\
\mcount(x) & = & 2^{n-1} \label{eqn:count:variable}\\
\mcount(\ite(x, v_t, v_e)) & = & \frac{\mcount(v_t) + \mcount(v_e)}{2} \label{eqn:count:ite}
\end{eqnarray}  
\end{theorem}
\noindent
{\em Proof:} The key result here is (\ref{eqn:count:ite}).  It follows
from Lemma~\ref{lemma:independent:split}, (\ref{eqn:ite:interp}),
Lemma~\ref{lemma:independent:dset}, and (\ref{eqn:depend:ite}).  That is, both
$v_t$ and $v_e$ are independent of $x$.  That implies that half of the
satisfying assignments $\interp$ for $v_t$ have have $\interp(x) =
\tautology$.  Similarly, half of the satisfying assignments $\interp$
for $v_e$ have $\interp(x) = \nil$.  Equation (\ref{eqn:count:ite}) then follows by (\ref{eqn:ite:interp}).

\bibliography{references}

\end{document}
