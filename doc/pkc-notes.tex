\documentclass[letterpaper,USenglish,cleveref, autoref, thm-restate]{lipics-v2021}
%This is a template for producing LIPIcs articles.
%See lipics-v2021-authors-guidelines.pdf for further information.
%for A4 paper format use option "a4paper", for US-letter use option "letterpaper"
%for british hyphenation rules use option "UKenglish", for american hyphenation rules use option "USenglish"
%for section-numbered lemmas etc., use "numberwithinsect"
%for enabling cleveref support, use "cleveref"
%for enabling autoref support, use "autoref"
%for anonymousing the authors (e.g. for double-blind review), add "anonymous"
%for enabling thm-restate support, use "thm-restate"
%for enabling a two-column layout for the author/affilation part (only applicable for > 6 authors), use "authorcolumns"
%for producing a PDF according the PDF/A standard, add "pdfa"

%\pdfoutput=1 %uncomment to ensure pdflatex processing (mandatatory e.g. to submit to arXiv)
%\hideLIPIcs  %uncomment to remove references to LIPIcs series (logo, DOI, ...), e.g. when preparing a pre-final version to be uploaded to arXiv or another public repository

%\graphicspath{{./graphics/}}%helpful if your graphic files are in another directory

\nolinenumbers %uncomment to disable line numbering


\usepackage{amsmath}
\usepackage{tikz}
\usepackage{pgfplots}
\usepackage{booktabs}
\usepackage{hyperref}

\newcommand{\pand}{\mathbin{\land^{\sf p}}}
\newcommand{\por}{\mathbin{\lor^{\sf p}}}
\DeclareMathOperator*{\Pand}{\bigwedge^{\sf p}}
\DeclareMathOperator*{\Por}{\bigvee^{\sf p}}
\newcommand{\boolnot}{\neg}
\newcommand{\tautology}{\top}
\newcommand{\nil}{\bot}
\newcommand{\obar}[1]{\overline{#1}}
\newcommand{\oneminus}{{\sim}}
\newcommand{\lit}{\ell}

\newcommand{\dvarset}{D}
\newcommand{\varset}{X}
\newcommand{\pvarset}{P}
\newcommand{\dependencyset}{{\cal V}}
\newcommand{\litset}{{\cal L}}
\newcommand{\ring}{{\cal R}}
\newcommand{\dset}{{\cal A}}
\newcommand{\rep}{\textbf{R}}
\newcommand{\drep}{\textbf{R}_D}
\newcommand{\radd}{+}
\newcommand{\rmul}{\times}
\newcommand{\addident}{\textbf{0}}
\newcommand{\mulident}{\textbf{1}}
\newcommand{\imply}{\Rightarrow}
\newcommand{\ifandonlyif}{\Leftrightarrow}
%\newcommand{\drational}{\mathbb{Q}_{2,5}}
\newcommand{\drational}{\textbf{Q}_{2,5}}

\newcommand{\assign}{\rho}
\newcommand{\uassign}{{\cal U}}
\newcommand{\modelset}{{\cal M}}
\newcommand{\dassign}{\alpha}
\newcommand{\passign}{\beta}
\newcommand{\acombine}{\!\cdot\!}
\newcommand{\dmodelset}{{\cal M_{D}}}
\newcommand{\pure}{\mathit{Pure}}

\newcommand{\simplify}[2]{#1|_{#2}}
\newcommand{\ureduce}[2]{\forall #2\, #1}
\newcommand{\partition}[2]{\mathit{Part}(#1, #2)}


\newcommand{\makenode}[1]{\mathbf{#1}}
\newcommand{\nodeu}{\makenode{u}}
\newcommand{\nodev}{\makenode{v}}
\newcommand{\nodes}{\makenode{s}}
\newcommand{\nodep}{\makenode{p}}
\newcommand{\noder}{\makenode{r}}

\newcommand{\progname}[1]{\textsc{#1}}
\newcommand{\dfour}{\progname{D4}}
\newcommand{\Dfour}{\progname{D4}}
\newcommand{\cadical}{\progname{CaDiCal}}
\newcommand{\dtrim}{\progname{drat-trim}}
\newcommand{\lean}{Lean~4}

\newcommand{\algo}[1]{\textsf{#1}}
\newcommand{\kc}{\algo{KC}}
\newcommand{\pkc}{\algo{PKC}}
\newcommand{\traverse}{\algo{Traverse}}
\newcommand{\posop}{\algo{Pos}}
\newcommand{\negop}{\algo{Neg}}
\newcommand{\bcp}{\algo{BCP}}
\newcommand{\uprop}{\algo{Uprop}}
\newcommand{\simpl}{\also{Simpl}}

\definecolor{redorange}{rgb}{0.878431, 0.235294, 0.192157}
\definecolor{lightblue}{rgb}{0.552941, 0.72549, 0.792157}
\definecolor{clearyellow}{rgb}{0.964706, 0.745098, 0}
\definecolor{clearorange}{rgb}{0.917647, 0.462745, 0}
\definecolor{mildgray}{rgb}{0.54902, 0.509804, 0.47451}
\definecolor{softblue}{rgb}{0.643137, 0.858824, 0.909804}
\definecolor{bluegray}{rgb}{0.141176, 0.313725, 0.603922}
\definecolor{lightgreen}{rgb}{0.709804, 0.741176, 0}
\definecolor{darkgreen}{rgb}{0.152941, 0.576471, 0.172549}
\definecolor{redpurple}{rgb}{0.835294, 0, 0.196078}
\definecolor{midblue}{rgb}{0, 0.592157, 0.662745}
\definecolor{clearpurple}{rgb}{0.67451, 0.0784314, 0.352941}
\definecolor{browngreen}{rgb}{0.333333, 0.313725, 0.145098}
\definecolor{darkestpurple}{rgb}{0.396078, 0.113725, 0.196078}
\definecolor{greypurple}{rgb}{0.294118, 0.219608, 0.298039}
\definecolor{darkturquoise}{rgb}{0, 0.239216, 0.298039}
\definecolor{darkbrown}{rgb}{0.305882, 0.211765, 0.160784}
\definecolor{midgreen}{rgb}{0.560784, 0.6, 0.243137}
\definecolor{darkred}{rgb}{0.576471, 0.152941, 0.172549}
\definecolor{darkpurple}{rgb}{0.313725, 0.027451, 0.470588}
\definecolor{darkestblue}{rgb}{0, 0.156863, 0.333333}
\definecolor{lightpurple}{rgb}{0.776471, 0.690196, 0.737255}
\definecolor{softgreen}{rgb}{0.733333, 0.772549, 0.572549}
\definecolor{offwhite}{rgb}{0.839216, 0.823529, 0.768627}
\definecolor{medgreen}{rgb}{0.15, 0.6, 0.15}

% Lean code:
\usepackage{listings}
%\definecolor{keywordcolor}{rgb}{0.7, 0.1, 0.1}   % red
\definecolor{keywordcolor}{rgb}{0.0, 0.1, 0.6}   % blue
\definecolor{tacticcolor}{rgb}{0.0, 0.1, 0.6}    % blue
\definecolor{commentcolor}{rgb}{0.4, 0.4, 0.4}   % grey
\definecolor{symbolcolor}{rgb}{0.0, 0.1, 0.6}    % blue
\definecolor{sortcolor}{rgb}{0.1, 0.5, 0.1}      % green
\definecolor{attributecolor}{rgb}{0.7, 0.1, 0.1} % red
\def\lstlanguagefiles{lstlean.tex}
% set default language
\lstset{language=lean, xleftmargin=1em}
\lstset{backgroundcolor=\color{white}}

\bibliographystyle{plainurl}% the mandatory bibstyle

\title{Notes on \\ Projected Knowledge Compilation \\ \today }

\titlerunning{Projected Knowledge Compilation}

\author{Randal E. Bryant}{Computer Science Department, Carnegie Mellon University, Pittsburgh, PA 15213 USA}{Randy.Bryant@cs.cmu.edu}{https://orcid.org/0000-0001-5024-6613}{Supported by NSF grant CCF-2108521}

\authorrunning{R. E. Bryant} %TODO mandatory. First: Use abbreviated first/middle names. Second (only in severe cases): Use first author plus 'et al.'

%\Copyright{Jane Open Access and Joan R. Public} %TODO mandatory, please use full first names. LIPIcs license is "CC-BY";  http://creativecommons.org/licenses/by/3.0/

%\ccsdesc[100]{\textcolor{red}{Replace ccsdesc macro with valid one}} %TODO mandatory: Please choose ACM 2012 classifications from https://dl.acm.org/ccs/ccs_flat.cfm
\begin{CCSXML}
<ccs2012>
   <concept>
       <concept_id>10003752.10003790.10003794</concept_id>
       <concept_desc>Theory of computation~Automated reasoning</concept_desc>
       <concept_significance>500</concept_significance>
       </concept>
 </ccs2012>
\end{CCSXML}

\ccsdesc[500]{Theory of computation~Automated reasoning}


%\keywords{Dummy keyword} %TODO mandatory; please add comma-separated list of keywords
\keywords{knowledge compilation, model counting, projected model counting}

%\category{} %optional, e.g. invited paper

%\funding{(Optional) general funding statement \dots}%optional, to capture a funding statement, which applies to all authors. Please enter author specific funding statements as fifth argument of the \author macro.

%\nolinenumbers %uncomment to disable line numbering

%Editor-only macros:: begin (do not touch as author)%%%%%%%%%%%%%%%%%%%%%%%%%%%%%%%%%%
%\EventEditors{John Q. Open and Joan R. Access}
%\EventNoEds{2}
%\EventLongTitle{42nd Conference on Very Important Topics (CVIT 2016)}
%\EventShortTitle{CVIT 2016}
%\EventAcronym{CVIT}
%\EventYear{2016}
%\EventDate{December 24--27, 2016}
%\EventLocation{Little Whinging, United Kingdom}
%\EventLogo{}
%\SeriesVolume{42}
%\ArticleNo{23}
%%%%%%%%%%%%%%%%%%%%%%%%%%%%%%%%%%%%%%%%%%%%%%%%%%%%%%


\begin{document}

\maketitle

\section{Logical Foundations}

  Let $\varset$ denote a set of Boolean variables, and let $\assign$
  be an \emph{assignment} of truth values to some subset of the
  variables, where $0$ denotes false and $1$ denotes true, i.e.,
  $\assign \colon Y \rightarrow \{0,1\}$ for some $Y  \subseteq \varset$.
We say the assignment is \emph{total with respect to a set of variables} $Y$, when it assigns a value to every $x \in Y$.
  The set of all possible total assignments with respect to a set of variables $Y$ is denoted $\uassign(Y)$.


For each variable $x \in \varset$,
  we define the \emph{literals} $x$ and $\obar{x}$, where $\obar{x}$ is the
  negation of $x$. An
  assignment $\assign$ can be viewed as a set of literals, with
   $\lit \in \assign$ when $\lit = x$ and $\assign(x) = 1$ or when
  $\lit = \obar{x}$ and $\assign(x) = 0$.  We write the negation of literal $\lit$ as $\obar{\lit}$.  That is, $\obar{\lit} = \obar{x}$ when $\lit = x$ and
$\obar{\lit} = x$ when $\lit = \obar{x}$.

\begin{definition}[Boolean Formulas]
  The set of Boolean formulas is defined recursively.  Each
  formula $\theta$ has an associated \emph{dependency set}
  $\dependencyset(\theta)  \subseteq \varset$, and a set of \emph{models} $\modelset(\theta)$,
  consisting of assignments that are total with respect to $\varset$ and that satisfy the formula:
  \begin{enumerate}
  \item Constant formulas $\nil$ (logical falsehood) and $\tautology$ (tautology) are Boolean formulas,
    with $\dependencyset(\nil) = \dependencyset(\tautology) = \emptyset$, with $\modelset(\nil) = \emptyset$, and with $\modelset(\tautology) = \uassign(\varset)$.
  \item Variable $x$ is a Boolean formula, with $\dependencyset(x) = \{x\}$
    and $\modelset(x) = \{\assign \in \uassign(\varset) \mid  \assign(x)=1\}$.
  \item For formula $\theta$, its \emph{negation}, written $\boolnot \theta$ is a Boolean formula,
    with $\dependencyset(\boolnot \theta) = \dependencyset(\theta)$ and $\modelset(\boolnot \theta) = \uassign(\varset) - \modelset(\theta)$.
  \item For formulas $\theta_1, \theta_2, \ldots, \theta_k$, their \emph{product} $\theta = \bigwedge_{1 \leq i \leq k} \theta_i$ is a Boolean formula, with
      $\dependencyset(\theta) = \bigcup_{1 \leq i \leq k} \dependencyset(\theta_i)$ and
      $\modelset(\theta) = \bigcap_{1 \leq i \leq k} \modelset(\theta_i)$.
  \item For formulas $\theta_1, \theta_2, \ldots, \theta_k$, their \emph{sum} $\theta = \bigvee_{1 \leq i \leq k} \theta_i$ is a Boolean formula, with
      $\dependencyset(\theta) = \bigcup_{1 \leq i \leq k} \dependencyset(\theta_i)$ and
      $\modelset(\theta) = \bigcup_{1 \leq i \leq k} \modelset(\theta_i)$.
  \end{enumerate}
\label{def:boolean}
\end{definition}

\subsection{Conjunctive Normal Form (CNF)}

  Many tools for reasoning about Boolean formulas operate on formulas in \emph{conjunctive normal form}, defined as follows.
  A formula is in  \emph{negation normal form} when negation is applied only to variables.  A
  formula is in \emph{conjunctive normal form} (CNF) when i) it is in
  negation normal form, and ii) sum is applied only to literals.  We use symbol $\phi$ to denote CNF formulas.

  A CNF
  formula can be represented as a set of \emph{clauses}, each of which is a
  set of literals.  Each clause represents the sum of the
  literals, and the formula is the product of its clauses.  We use
  set notation to reference the clauses within a formula and the
  literals within a clause.  A clause consisting of a single literal is referred to as a \emph{unit} clause and the literal as a \emph{unit} literal.
This literal must be assigned value $1$ by any satisfying assignment of the formula.  We use symbol $\phi$ to denote a CNF formula.

\subsection{Partitioned-Operation Formulas}

Computing many important properties of Boolean formulas, including
whether they are satisfiable or how many models that have, is
intractable for general Boolean formulas, as well as for those in
conjunctive normal form.  By imposing sufficient restrictions, on the
other hand, these properties can be computed in polynomial time (with
respect to the formula size).

The set of partitioned-operation operation formulas (POFs) is defined to be those Boolean formulas $\psi$ obeying the following recursively defined restrictions.
\begin{description}
\item[Constants:] 
$\psi = \tautology$ and $\psi = \nil$.
 \item[Literals:]
   $\psi = x$ and $\psi = \obar{x}$ for $x \in \varset$
 \item[Product operations:] $\psi = P(\psi_1, \psi_2, \ldots, \psi_k)$ where each $\psi_i$ is a POF\@.
   This product represents the Boolean formula
   $\psi = \bigwedge_{1 \leq i \leq k} \psi_i$.
 The operation arguments must be defined over disjoint sets of variables.
 That is, $\dependencyset(\psi_i) \cap \dependencyset(\psi_j) = \emptyset$ for all $1 \leq i < j \leq k$.
 \item[Sum operations:] $\psi = S(\psi_1, \psi_2, \ldots, \psi_k)$ where each $\psi_i$ is a POF\@.
   This sum represents the Boolean formula
   $\psi = \bigvee_{1 \leq i \leq k} \psi_i$.
The operation arguments must have disjoint models.
That is, $\modelset(\psi_i) \cap \modelset(\psi_j) = \emptyset$ for $1 \leq i < j \leq k$.
\item[Negations:] $\neg \psi'$, where $\psi'$ is a sum or product operation.
\end{description}

\subsection{Projected Formulas}

The set of Boolean variables $\varset$ is divided into two sets: {\em data} variables
$\dvarset$ and {\em auxiliary} variables $\pvarset$.  We use separate
assignments for the two sets of variables:
$\dassign \colon \dvarset \rightarrow \{0,1\}$
 and
 $\passign \colon \pvarset \rightarrow \{0,1\}$, and the symbol $\acombine$ to combine the two assignments:
 $\dassign\acombine\passign \colon \varset \rightarrow \{0,1\}$.
Sets $\uassign(\dvarset)$ and $\uassign(\pvarset)$ therefore denote the set of all assignments that are total with respect to $\dvarset$ and $\pvarset$, respectively.

For Boolean formula $\theta$,
its set of {\em data models}, written $\dmodelset(\theta)$ is defined as:
\begin{eqnarray}
\dmodelset(\theta) & = & \{ \dassign \in \uassign(\dvarset) \mid  \exists \passign \in \uassign(\pvarset) \; \dassign\acombine\passign \in \modelset(\theta) \} \label{eqn:dmodelset}
\end{eqnarray}

\subsection{Data Ring Evaluation}

  For commutative ring $\ring$, a {\em ring weight function} associates a value $w(x) \in \dset$ with
  every data variable $x \in \dvarset$.  We then define $w(\obar{x}) \doteq \mulident-w(x)$.

  For Boolean formula $\theta$ and ring weight function $w$, its {\em data ring evaluation} computes
  \begin{eqnarray}
    \drep(\theta, w) & = & \sum_{\alpha \in \dmodelset(\theta)} \;\; \prod_{\lit \in \alpha} w(\lit) \label{eqn:rep}
  \end{eqnarray}
where $\sum$ and $\prod$ represent the ring sum $\radd$ and ring product $\rmul$ operations, respectively.

For a partitioned-operation formula $\psi$ such that $\dependencyset(\psi) \subseteq \dvarset$, its data ring evaluation can be computed based on the recursive structure of the formula:
\begin{itemize}
\item Constants: $\drep(\tautology) = \mulident$ and $\drep(\nil) = \addident$
\item Literals: $\drep(x) = w(x)$ and $\drep(\obar{x}) = \mulident - x$
\item Product operations: 
$\drep(P(\psi_1, \psi_2, \ldots, \psi_k)) = \prod_{1 \leq i \leq k} \drep(\psi_i)$
\item Sum Operations
$\drep(S(\psi_1, \psi_2, \ldots, \psi_k)) = \sum_{1 \leq i \leq k} \drep(\psi_i)$
\item Negations: $\drep(\neg \psi) = \mulident -\psi$
\end{itemize}

The task of \emph{projected knowledge compilation} is then defined as follows: Given clausal formula $\phi$, generate a partitioned-operation formula $\psi$ having $\dependencyset(\psi) \subseteq \dvarset$, such that
$\dmodelset(\psi) = \dmodelset(\phi)$.

\section{Compilation Algorithm}

We can build on existing knowledge compilers to implement a projected knowledge compiler.

\subsection{Operations on Clausal Representations}

Let $U(\phi)$ denote the set of \emph{unit literals} in $\phi$, i.e., the set of all literals $\lit$ such that clause $\{ \lit \} \in \phi$.  In the following $L$ denotes a set of literals.

We define the following operations for transforming clausal formula $\phi$ into a new clausal formula:
\begin{description}
\item[$\posop(\phi, L)$:] Remove tautologies, as well as clauses satisfied by the literals: $\posop(\phi, L) = \{ C \in \phi \mid  C \cap L = \emptyset\}$.

\item[$\negop(\phi, L)$:] Remove the negations of the literals from all clauses.  Let $\negop(C, L) = \{ \lit \in C \mid  \obar{\lit} \not \in L\}$.
  Then $\negop(\phi, L) = \{ \negop(C, L) \mid  C \in \phi \}$.  If this process generates an empty clause, then the result is $\nil$.

\item[$\simplify{\phi}{L}$:] Simplify the clauses by assigning the literals, removing all occurrences of any literal for the variables in $\dependencyset(L)$:
$\simplify{\phi}{L} = \posop(\negop(\phi, L), L)$. 

\item[$\uprop(\phi)$:]  Perform unit propagation based on the unit literals of $\phi$.  This involves simplifying the clauses according to the unit literals and then adding back the unit clauses.
$\uprop(\phi) = \simplify{\phi}{U(\phi)} \cup \{ \{\lit \} \mid  \lit \in U(\phi) \}$.

\item[$\bcp(\phi)$:] Perform Boolean constraint propagation (BCP).
  Repeatedly compute $\phi \leftarrow \uprop(\phi)$ until either $\phi$ becomes $\nil$ or a fixed point is reached.

\item[$\ureduce{\phi}{x}$:] Express the universal quantification of $\phi$ with respect to variable $x$ in clausal form:
$\ureduce{\phi}{x} = \negop(\phi, \{x, \obar{x}\})$.

\item[$\partition{\phi}{Y}$:] Find the subset of clauses that depend only on the variables in $Y$:
$\partition{\phi}{Y} = \{ C \in \phi \mid \dependencyset(C) \subseteq Y\}$.

\end{description}

\subsection{Operations on POFs}

When constructing POFs via a bottom-up process, we can perform a set of simplifications, as follows:

\begin{itemize}
\item
The use of negation can be optimized as follows:
\begin{itemize}
\item Constants: $\neg \tautology = \nil$ and $\neg \nil = \tautology$
\item Literals: $\neg x = \obar{x}$ and $\neg \obar{x} = x$
\item Double negations: $\neg \neg \psi = \psi$
\end{itemize}

\item 
When forming a product $P(\psi_1, \psi_2, \ldots, \psi_k)$ we can perform the following simplifications:
\begin{itemize}
\item If $\psi_i = \nil$ for any argument $\psi_i$, then the result is $\nil$
\item Eliminate any argument $\psi_i$ for which $\psi_i = \tautology$.
\item If the reduced set of arguments consists of a single formula, then this formula becomes the result.
\item Otherwise, form a product operation over the reduced set of arguments.
\end{itemize}

\item
Similarly, when forming a sum $S(\psi_1, \psi_2, \ldots, \psi_k)$ we can perform the following simplifications:
\begin{itemize}
\item If $\psi_i = \tautology$ for any argument $\psi_i$, then the result is $\tautology$
\item Eliminate any argument $\psi_i$ for which $\psi_i = \nil$.
\item If the reduced set of arguments consists of a single formula, then this formula becomes the result.
\item Otherwise, form a sum operation over the reduced set of arguments.
\end{itemize}
\end{itemize}

With these simplifications, we can assume that the POFs we generate will satisfy the following:
\begin{itemize}
\item Constant values $\tautology$ and $\nil$ only occur when the POF is itself
  $\tautology$ or $\nil$.
\item Negation is only applied to product and sum operations.
\item Every product and sum operation will have at least two arguments.
\end{itemize}

\subsection{The Supporting Knowledge Compiler}

We assume that we have a knowledge compiler that can convert a clausal formula $\phi$ into a logically equivalent POF $\psi$.  Furthermore, we assume that the generated POF obeys the following restrictions:
\begin{itemize}
\item The POF is in normal form, i.e., there are no negated product or sum operations.
\item The sum operations are \emph{decision operations}.  That is, each sum operation is of the form $S(\psi_1, \psi_2)$, and it has an associated decision variable $x$,
  such that 1) either $\psi_1 = x$ or $\psi_1$ is a product operation having $x$ as an argument, and 2)
either $\psi_2 = \obar{x}$ or $\psi_2$ is a product operation having $\obar{x}$ as an argument.
\end{itemize}
We let $\kc(\phi)$ denote the output of the knowledge compiler for input formula $\phi$.

\subsection{Algorithm}

Our goal is to implement a projecting knowledge compiler that produces a POF $\pkc(\phi)$ when given input formula $\phi$.
We define our algorithm in terms of a recursive function $\traverse(\psi, \phi)$ that simultaneously operates on a POF $\psi$ and a clausal formula $\phi$.  Projected knowledge compilation is then defined
as $\pkc(\phi) = \traverse(\kc(\phi), \bcp(\phi))$.

The traversal is defined based on the recursive structure of its first argument:
\begin{description}
\item[Constants:]
  $\traverse(\tautology, \phi) = \tautology$ and 
  $\traverse(\nil, \phi) = \nil$.

\item[Literals:]
  $\traverse(\lit, \phi) = \lit$ if $\dependencyset(\lit) \subseteq \dvarset$, and $\traverse(\lit, \phi) = \tautology$ if $\dependencyset(\lit) \in \pvarset$.
\item[Product Operations:]
  Let the product operation be of the form $P(\lit_1, \lit_2, \ldots, \lit_k, \psi_1, \psi_2, \ldots, \psi_m)$, where each $\psi_i$ is either a sum or a  product operation.
  \begin{enumerate}
  \item Let $L = \{\lit_1, \lit_2, \ldots, \lit_k\}$.
  \item Define clausal formula $\phi'$ to be $\bcp(\simplify{\phi}{L})$
  \item For $1 \leq i \leq m$, recursively compute $\psi'_i = \traverse(\psi_i, \partition{\phi'}{\dependencyset(\psi_i)})$.  (Partitioning is only required when $m > 1$.)
  \item Return the product (with simplifications) for the set of arguments $\{\lit \in L \mid  \dependencyset(\lit) \subseteq \dvarset\} \cup \{ \psi'_i \mid  1 \leq i \leq m \}$.
  \end{enumerate}
\item[Sum Operations:]
  Let the sum operation be of the form $S(\lit_1, \lit_2)$ and having decision variable $x$.
  \begin{enumerate}
  \item For $1 \leq i \leq 2$, recursively compute $\psi'_i = \traverse(\psi_i, \phi)$.
  \item If $x \in \dvarset$, then return the sum (with simplifications) $S(\psi'_1, \psi'_2)$.
  \item Otherwise, let $\psi' = \pkc(\ureduce{\phi}{x})$
    \begin{enumerate}
    \item If $\psi' = \nil$, then return the sum (with simplifications) $S(\psi'_1, \psi'_2)$
    \item Otherwise return nested formula $S(\neg S(\neg \psi'_1, \psi'), \psi'_2)$.
    \end{enumerate}
  \end{enumerate}
\end{description}

Of these steps, the most interesting ones involve the handling of sum
operations when the decision variable $x$ is to be projected.  The
recursive calls yield subformulas $\psi'_1$ and $\psi'_2$, representing the projected compilation of $\phi$ when $x$ is assigned $1$ and $0$, respectively.
These
formulas potentially have overlapping models and so care must be taken
in forming their sum.  Formula $\ureduce{\phi}{x}$ is the clausal
representation of the universal quantification of $\phi$ with respect
to $x$, and so $\psi' = \pkc(\ureduce{\phi}{x})$ will be a POF whose
data models are the intersection of those for $\psi_1'$ and $\psi_2'$.  In
the event that $\psi' = \nil$, the two formulas are disjoint.  This
will occur, for example, when $x$ is a Tseitin variable encoding some
Boolean formula over the data variables.  In that case, the sum can be
formed directly.  If $\psi' \not = \tautology$, we want to
``subtract'' the overlapping models from one of the arguments before
forming the sum with the other.  We can do so using DeMorgan's Laws.  
That is formula $\neg S(\neg \psi'_1, \psi')$ will have as data models $\dmodelset(\psi'_1) - \dmodelset(\psi')$.  Furthermore, the arguments to this sum operation will have disjoint models
since $\dmodelset(\psi') \subseteq \dmodelset(\psi'_1)$, and therefore
$\dmodelset(\psi') \cap \dmodelset(\neg \psi'_1) = \emptyset$.
As can be seen, we could also reverse the role of the two arguments, computing the result
$S(\psi'_1, \neg S(\neg \psi'_2, \psi'))$.


\subsection{Optimizations}

We can improve the efficiency of the traversal by applying the following optimizations:
\begin{itemize}
\item Only data variables in $\psi$.  If $\dependencyset(\psi) \subseteq \dvarset$, then $\traverse(\psi, \phi) = \psi$.
\item Only data variables in $\phi$.  If $\dependencyset(\phi) \subseteq \dvarset$, then $\traverse(\psi, \phi) = \kc(\phi)$.
\item Only projection variables in $\phi$.  If $\dependencyset(\phi) \subseteq \pvarset$, then $\traverse(\psi, \phi) = \tautology$ if formula $\phi$ is satisfiable and $\nil$ otherwise.
\item Cached result. If $\psi$ is a sum operation and some previous traversal had the same clausal formula $\phi$ as its second argument, then the previously computed result can be reused.\footnote{
Caching for product operations is not, in general valid, since the two recursive calls made when processing a sum operation can be for product operations with the same clausal argument.}
\item Pure literal reduction.
Literal $\lit$ is said to be \emph{pure} in $\phi$ if some clause in $\phi$ contains $\lit$, and no clause in $\phi$ contains  $\obar{\lit}$.
Let $\pure(\phi)$ denote the set of all pure literals in $\phi$, and $\pure_{\pvarset}(\phi) = \{ \lit \in \pure(\phi) \mid  \dependencyset(\lit) \subseteq \pvarset \}$.
If $\pure_{\pvarset}(\phi) \not = \emptyset$, then $\phi$ can be reduced as
$\phi \leftarrow \bcp(\simplify{\phi}{\pure_{\pvarset}(\phi)})$.
\end{itemize}

\section{Partitioned-Operation Graphs (POGs)}

To allow a sharing of subformulas, we represent partitioned-operation
formulas as a class of directed acyclic graphs, known as POGs.  These
graphs have nodes denoting product and sum operations and edges
denoting arguments, where an edge can be directed to a variable, to
$\tautology$, or to another POG node.  Each edge has a \emph{phase}
indicating whether  the argument is to be used directly
(phase = $1$) or negated (phase = $0$).  The top-level POF is represented by a \emph{root edge}.


\section{Example Formula}

Consider an encoding of the formula $(x_1 \land x_2) \lor x_3$ using
encoding variable $y$.  A full {\em Tseitin} encoding of $y
\leftrightarrow (x_1 \land x_2)$ requires three clauses, but we can
use a Plaisted-Greenbaum encoding in this context, encoding just $y
\rightarrow (x_1 \land x_2)$, represented by the clauses
$x_1 \lor \obar{y}$ and $x_2 \lor \obar{y}$.  We then encode the disjunction
with the clause $x_3 \lor y$.  The DIMACS representation of the input formula $\phi$ is then as follows (using variable 4 to represent $y$):

\begin{center}
\begin{tabular}{ll}
\toprule
\makebox[10mm]{Clause} & \\
\midrule
 \texttt{1 -4} & \texttt{0} \\
 \texttt{2 -4} & \texttt{0} \\
 \texttt{3  4} & \texttt{0}\\
\bottomrule
\end{tabular}
\end{center}

\begin{figure}
  \begin{minipage}{0.48\textwidth}
    (A) POG representation of input formula \\[1.2ex]
    \input{dd/eg-preproj}
  \end{minipage}
  \begin{minipage}{0.48\textwidth}
  \end{minipage}
  \begin{minipage}{0.48\textwidth}
    (B) After projecting variable $y$ \\[1.2ex]
    \input{dd/eg-proj}
  \end{minipage}
  \caption{POG Representation of formula (A), and its projection (B)
  A black dot on an edge denotes Boolean negation.}
  \label{fig:eg-proj}
\end{figure}


Figure \ref{fig:eg-proj}(A) shows the POG representation of the
example formula, derived from the dec-DNNF representation produced by
\dfour{}.  In this graph, product nodes are labeled with the operation $\pand$, and sum nodes are labeled with the operation $\por$, emphasizing that these operations satisfy their respective partitioning constraints.
We can see that \dfour{} used $y$ as the decision variable at the top level, creating sum node $\nodes_7$ having edges to
node $\nodep_5$ representing product operation $P(x_3, \obar{y})$ on the left and  node $\nodep_6$
representing product operation $P(x_1, x_2, y)$ on the right.

Figure \ref{fig:eg-proj}(B) shows the result of our algorithm, based on the following recursive steps.
Computing $\traverse(P(x_3, \obar{y}), \phi)$ yields the literal $x_3$, while
computing $\traverse(P(x_1, x_2, y), \phi)$ yields node $\nodep_{10}$ representing the product operation $P(x_1, x_2)$.
Computing $\ureduce{\phi}{y}$ yields the clausal formula $\phi'$ given as
\begin{center}
\begin{tabular}{ll}
\toprule
\makebox[10mm]{Clause} & \\
\midrule
\texttt{1} & \texttt{0} \\
\texttt{2} & \texttt{0} \\
 \texttt{3} & \texttt{0}\\
\bottomrule
\end{tabular}
\end{center}
and this is compiled to node $\nodep_{8}$ representing product operation $P(x_1, x_2, x_3)$.
Finally, the nodes are assembled using our DeMorgan's Law transformations as shown with root node $\nodes_{11}$.

We can see in the POG of (B) that node $\nodep_{8}$ encodes $x_1 \land x_2
\land x_3$, and this formula's only model assigns value $1$ to
variables $x_1$, $x_2$, and $x_3$.  This is the only assignment
satisfying both $x_3$ and $x_1 \land x_2$, and so our construction
subtracts this model from those of $x_3$ before forming a sum with
$x_1 \land x_2$.


\end{document}
